\chapter{Efficient Implementation of the SVD Estimator}
\label{appB}
As noted in Section~\ref{sing_val_formalism}, taking the SVD of the 
transformation matrix $\widetilde{M}_{\gamma\delta} \equiv 
\mathcal{N}_{\gamma\gamma}^{-1/2}M_{\gamma\delta}$
is not trivial for large fields.  This appendix will first give a rough
outline of the form of $M_{\gamma\delta}$, then describe our tensor 
decomposition method which enables quick calculation of the singular
value decomposition.  For a more thorough review of the lensing results, 
see e.g.~\citet{Bartelmann01}.

Our goal is to speed the computation of the SVD by writing \
$\widetilde{M}_{\gamma\delta}$
as a tensor product $\mymat{A} \otimes \mymat{B}$.  Here ``$\otimes$''
is the Kronecker product, defined such that, if $\mymat{A}$ is a matrix
of size $n \times m$, $B$ is a matrix of arbitrary size,
\begin{equation}
  \mymat{A}\otimes\mymat{B} \equiv \left(
  \begin{array}{cccc}
    A_{11}B & A_{12}B & \cdots & A_{1m}B \\
    A_{21}B & A_{22}B & \cdots & A_{2m}B \\
    \vdots  & \vdots & \ddots & \vdots  \\
    A_{n1}B & A_{n2}B & \cdots & A_{nm}B 
  \end{array}\right)
\end{equation}
In this case, the singular value decomposition
$A\otimes B = U_{AB}\Sigma_{AB}V^\dagger_{AB}$
satisfies
\begin{eqnarray}
  \label{AB_SVD}
  U_{AB} &=& U_A\otimes U_B \nonumber\\
  \Sigma_{AB} &=& \Sigma_A \otimes \Sigma_B\nonumber\\
  V_{AB} &=& V_A \otimes V_B
\end{eqnarray}
where $U_A\Sigma_AV^\dagger_A$ is the SVD of $A$, 
and   $U_B\Sigma_BV^\dagger_B$ is the SVD of $B$.
Decomposing $\widetilde{M}_{\gamma\delta}$ in this way can 
greatly speed the SVD computation.

\subsection{Angular and Line-of-Sight Transformations}
The transformation from shear to density, encoded in $M_{\gamma\delta}$,
consists of two steps: an angular integral relating shear $\gamma$ to
convergence $\kappa$, and a line-of-sight integral relating the convergence
$\kappa$ to the density contrast $\delta$.

The relationship between $\gamma$ and $\kappa$ is a convolution over
all angular scales,
\begin{equation}
  \label{gamma_integral}
  \gamma(\myvec\theta,z_s) \equiv \gamma_1 + i\gamma_2 = \int \dd^2\theta^\prime
  \ \mathcal{D}(\myvec\theta^\prime-\myvec\theta)\kappa(\myvec\theta^\prime,z_s),
\end{equation}
where $\mathcal{D}(\myvec\theta)$ 
is the Kaiser-Squires kernel \citep{Kaiser93}.  
This has a particularly simple form in Fourier space:
\begin{equation}
  \label{gamma_fourier}
  \hat\gamma(\myvec\ell,z_s) 
  = \frac{\ell_1 + i\ell_2}{\ell_1 - i\ell_2}\hat\kappa(\myvec\ell,z_s).
\end{equation}
where $\hat\gamma$ and $\hat\kappa$ are the Fourier transforms of $\gamma$
and $\kappa$ and $\myvec{\ell}\equiv(\ell_1,\ell_2)$ is the angular wavenumber.

The relationship between $\kappa$ and $\delta$ is an integral along each
line of sight:
\begin{equation}
  \label{kappa_integral}
  \kappa(\myvec\theta,z_s) = 
  \int_0^{z_s}\dd z\ W(z,z_s)\delta(\myvec\theta,z)
\end{equation}
where $W(z,z_s)$ is the lensing efficiency function at redshift $z$ 
for a source located at redshift $z_s$ 
(refer to STH09 for the form of this function).

Upon discretization of the quantities $\gamma$, $\kappa$, and $\delta$
(described in Section~\ref{LinearMapping}), 
the integrals in Equations~\ref{gamma_integral}-\ref{kappa_integral} 
become matrix operations.  The relationship between the data vectors
$\myvec{\gamma}$ and $\myvec{\kappa}$ can be written
\begin{equation}
  \label{P_gk}
  \myvec\gamma = [\mymat{P}_{\gamma\kappa} \otimes \mathbf{1}_s]\myvec\kappa 
  + \myvec{n}_\gamma
\end{equation}
where $\mathbf{1}_s$ is the $N_s \times N_s$ identity matrix and 
$\mymat{P}_{\gamma\kappa}$ is the matrix representing the linear 
transformation in Equations~\ref{gamma_integral}-\ref{gamma_fourier}.  
The quantity $[\mymat{P}_{\gamma\kappa} \otimes \mathbf{1}_s]$ 
simply denotes that $\mymat{P}_{\gamma\kappa}$ operates on each of the $N_s$ 
source-planes represented within the vector $\myvec\kappa$.
Similarly, the relationship between the vectors $\myvec{\kappa}$ and
$\myvec{\delta}$ can be written
\begin{equation}
  \label{Q_kd}
  \myvec\kappa = [\mathbf{1}_{xy} \otimes \mymat{Q}_{\kappa\delta}]\myvec\delta
\end{equation}
where $\mathbf{1}_{xy}$ is the $N_{xy} \times N_{xy}$ 
identity matrix, and the tensor product signifies that the operator 
$Q_{\kappa\delta}$ operates on each of the $N_{xy}$ lines-of-sight in
$\myvec\delta$.  $Q_{\kappa\delta}$ is the $N_s \times N_l$ matrix which
represents the discretized version of equation \ref{kappa_integral}.
Combining these representations allows us to decompose the matrix 
$\mymat{M}_{\gamma\delta}$ in Equation~\ref{M_gd} into a tensor product:
\begin{equation}
  \mymat{M}_{\gamma\delta} = 
  \mymat{P}_{\gamma\kappa} \otimes \mymat{Q}_{\kappa\delta}.
\end{equation}

\subsection{Tensor Decomposition of the Transformation}
We now make an approximation that the noise covariance 
$\mymat{\mathcal{N}}_{\gamma\gamma}$ can be written as a
tensor product between its angular part $\mymat{\mathcal{N}_P}$ 
and its line of sight part $\mymat{\mathcal{N}_Q}$:
\begin{equation}
  \label{noise_decomp}
  \mymat{\mathcal{N}}_{\gamma\gamma} 
  = \mymat{\mathcal{N}_P} \otimes \mymat{\mathcal{N}_Q}.
\end{equation}
Because shear measurement error comes primarily from shot noise, this 
approximation is equivalent to the statement that source galaxies are drawn 
from a single redshift distribution, with a different normalization along 
each line-of-sight.  For realistic data, this approximation will break down
as the size of the pixels becomes very small.  We will assume here for 
simplicity that the noise covariance is diagonal, but the following results
can be generalized for non-diagonal noise.  
Using this noise covariance approximation, we can compute the 
SVDs of the components of $\widetilde{M}_{\gamma\delta}$:
\begin{eqnarray}
  \mymat{U}_P\mymat{\Sigma}_P\mymat{V}_P^\dagger = \mathcal{N}_P^{-1/2} \mymat{P}_{\gamma\kappa}\nonumber\\
  \mymat{U}_Q\mymat{\Sigma}_Q\mymat{V}_Q^\dagger = \mathcal{N}_Q^{-1/2} \mymat{Q}_{\kappa\delta}
\end{eqnarray}

In practice the SVD of the matrix $\mymat{P}_{\gamma\kappa}$ 
need not be computed explicitly.  
$\mymat{P}_{\gamma\kappa}$ encodes the discrete linear operation expressed
by Equations~\ref{gamma_integral}-\ref{gamma_fourier}: 
as pointed out by STH09, in the large-field limit $P_{\gamma\kappa}$ 
can be equivalently computed in either real or Fourier space.
Thus to operate with $P_{\gamma\kappa}$ on a shear vector, 
we first take the 2D Fast Fourier Transform (FFT) of each
source-plane, multiply by the kernel $(\ell_1+i\ell_2)/(\ell_1-i\ell_2)$,
then take the inverse FFT of the result.  This is orders-of-magnitude
faster than a discrete implementation of the real-space convolution.
Furthermore, the conjugate transpose of this operation can be computed
by transforming $\ell \to -\ell^*$, so that
\begin{equation}
  \mymat{P}_{\gamma\kappa}^\dagger\mymat{P}_{\gamma\kappa} = \mymat{I}
\end{equation}
and we see that $P_{\gamma\kappa}$ is unitary in the wide-field limit.  This
fact, along with the tensor product properties of the SVD, allows us to
write $\widetilde{M}_{\gamma\delta} = U\Sigma V^\dagger$ where
\begin{eqnarray}
  U &\approx& \mathbf{1}_{xy} \otimes U_Q \nonumber\\
  \Sigma &\approx& \mathcal{N}_P^{-1/2} \otimes \Sigma_Q \nonumber\\
  V^\dagger &\approx& P_{\gamma\kappa} \otimes  V_Q^\dagger
\end{eqnarray}
The only explicit SVD we need to calculate is that of 
$\mathcal{N}_Q^{-1/2}\mymat{Q}_{\kappa\delta}$,
which is trivial in cases of interest.  
The two approximations we have made are the
applicability of the Fourier-space form of the $\gamma\to\kappa$ mapping
(Eqn.~\ref{gamma_fourier}), and the tensor
decomposition of the noise covariance (Eqn.~\ref{noise_decomp}).
