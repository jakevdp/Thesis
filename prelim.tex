%
% ----- copyright and title pages
%
\Title{Karhunen-Lo\`{e}ve Analysis for Gravitational Lensing}
\Author{Jacob Vanderplas}
\Year{2012}
\Program{UW Astronomy}

\Chair{Andrew Connolly}{Professor}{Department of Astronomy}
\Signature{Bhuvnesh Jain}
\Signature{Andrew Becker}

%\copyrightpage
\titlepage

\setcounter{page}{-1}
\abstract{%
In the past decade, weak gravitational lensing has become an important tool
in the study of the universe at the largest scale, giving insights into the
distribution of dark matter, the expansion of the universe, and the nature
of dark energy. This thesis research explores several applications
of Karhunen-Lo\`{e}ve (KL) analysis to speed and improve the comparison of
weak lensing shear catalogs to theory in order to constrain cosmological 
parameters in current and future lensing
surveys.% This work addresses three related aspects of weak lensing analysis:

%\begin{enumerate}

%\item {\bf Three-dimensional tomographic mapping.}
%(based on work published in Vanderplas et al. 2011). We develop a fast,
%KL-based approach to the three-dimensional reconstruction of matter
%maps from shear surveys. This KL approach solves the lensing inversion 
%problem using truncation of eigenvalues within the context of generalized 
%least squares estimation, without
%a priori assumptions about the statistical nature of the signal. 
%It also allows a quantitative
%comparison between different filtering methods: we evaluate our 
%method beside the previously
%explored Wiener-filter approaches. Our method yields near-optimal 
%angular resolution
%of the lensing reconstruction and allows cluster sized halos to be 
%de-blended robustly.
%It allows for mass reconstructions which are two to three orders of 
%magnitude faster than
%the Wiener-filter approach; in particular, we estimate that an 
%all-sky reconstruction with arcminute
%resolution could be performed on a timescale of hours. We find 
%however that linear,
%non-parametric reconstructions have a fundamental limitation in 
%the resolution achieved in the redshift direction.

%\item {\bf Mapping and shear peak statistics with noisy and incomplete data.}
% (based on work published
%in Vanderplas et al 2012). We use KL analysis to solve a 
%practical problem in the analysis
%of gravitational shear surveys. Shear catalogs from 
%large-field weak-lensing surveys
%will be subject to many systematic limitations, notably 
%incomplete coverage and pixel-level
%masking due to foreground sources. We develop amethod to 
%use two-dimensional KL eigenmodes
%of shear to interpolate noisy shear measurements across 
%masked regions. We explore
%the results of thismethod with simulated shear catalogs, 
%using statistics of high-convergence
%regions in the resulting map. We find that the KL 
%procedure not only minimizes the bias due
%to masked regions in the field, it also reduces spurious 
%peak counts from shape noise by a
%factor of 3 in the cosmologically sensitive regime. This 
%indicates that KL reconstructions of
%masked shear are not only useful for creating robust 
%convergence maps from masked shear
%catalogs, but also offer promise of improved parameter 
%constraints within studies of shear peak statistics.

%\item {\bf Two-point parameter estimation with noisy and incomplete data.}
%The two point statistics
%of shear, namely the power spectrumand correlation function, 
%scale with cosmology in ways
%that are predictable using a combination of analytic models and 
%simulations of the universe.
%As such, these measures can give us fundamental insight into the 
%nature of the universe.
%When measuring the power spectrum directly, gaps in the data can 
%lead to a situation where
%fourier modes are no longer orthogonal. This can lead to mode 
%mixing which is complicated
%to correct. We use KL analysis to construct an alternative 
%orthonormal basis in which
%to record this cosmological information. In addition to addressing 
%the mode mixing problem,
%KL allows for robust noise filtration and direct comparison to 
%cosmological models, all
%within a very fast matrix-based framework.

%\end{enumerate}
}
 
%
% ----- contents & etc.
%
\tableofcontents
%\listoffigures
%\listoftables
 
%
% ----- glossary 
%
%\chapter*{Glossary}      % starred form omits the `chapter x'
%\addcontentsline{toc}{chapter}{Glossary}
%\thispagestyle{plain}
%
%\begin{glossary}
%\item[item1] description description description description description
%  description description description description description description
%  description description description description description description
%\item[item2] description description description description description
%  description description description description description description
%  description description description description description description
%\item[item3] description description description description description
%  description description description description description description
%  description description description description description description 
%\end{glossary}
 
%
% ----- acknowledgments
%
\acknowledgments{% \vskip2pc
  % {\narrower\noindent
  This is an acknowledgment, in which the acknowledged are acknowledged.
  % \par}
}

%
% ----- dedication
%
\dedication{\begin{center}This is the dedication.\end{center}}

%
% end of the preliminary pages
 
 