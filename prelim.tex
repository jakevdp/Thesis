%
% ----- copyright and title pages
%
\Title{Karhunen-Lo\`{e}ve Analysis for Weak Gravitational Lensing}
\Author{Jacob T. Vanderplas}
\Year{2012}
\Program{Department of Astronomy}

\Chair{Andrew Connolly}{Professor}{Department of Astronomy}
\Signature{Bhuvnesh Jain}
\Signature{Andrew Becker}

\copyrightpage
\titlepage

\setcounter{page}{-1}

\abstract{
In the past decade, weak gravitational lensing has become an important tool
in the study of the universe at the largest scale, giving insights into the
distribution of dark matter, the expansion of the universe, and the nature
of dark energy. This thesis research explores several applications
of Karhunen-Lo\`{e}ve (KL) analysis to speed and improve the comparison of
weak lensing shear catalogs to theory in order to constrain cosmological 
parameters in current and future lensing
surveys. This work addresses three related aspects of weak lensing analysis:

\begin{description}
   \item[Three-dimensional Tomographic Mapping:]
        \citep[Based on work published in][]{Vanderplas2011}
        We explore a new fast approach to three-dimensional mass mapping in
        weak lensing surveys.  The KL approach uses a KL-based filtering of
        the shear signal to reconstruct mass structures on the line-of-sight,
        and provides a unified framework to evaluate the efficacy of linear
        reconstruction techniques.  We find that the KL-based filtering leads
        to near-optimal angular resolution, and computation times which are
        faster than previous approaches.  We also use the KL formalism to
        show that linear non-parametric reconstruction methods are
        fundamentally limited in their ability to resolve lens redshifts.
   \item[Shear Peak Statistics with Incomplete Data]
        \citep[Based on work published in][]{Vanderplas2012}
        We explore the use of KL eigenmodes for interpolation across masked     
        regions in observed shear maps.  Mass mapping is an inherently
        non-local calculation, meaning gaps in the data can have a significant
        effect on the properties of the derived mass map.  Our KL mapping
        procedure leads to improvements in the recovery of detailed statistics
        of peaks in the mass map, which holds promise of improved cosmological
        constraints based on such studies.
   \item[Two-point parameter estimation with KL modes]
        The power spectrum of the observed shear can yield powerful cosmological
        constraints.  Incomplete survey sky coverage, however, can lead to
        mixing of power between Fourier modes, and obfuscate the cosmologically
        sensitive signal.  We show that KL can be used to derive an alternate
        orthonormal basis for the problem which avoids mode-mixing and allows
        a convenient formalism for cosmological likelihood computations.
        Cosmological constraints derived using this method are shown to be
        competitive with those from the more conventional correlation function
        approach.  We also discuss several aspects of the KL approach which
        will allow improved handling of correlated errors and redshift
        information in future surveys.
\end{description}










%\begin{enumerate}
%\item {\bf Three-dimensional tomographic mapping.}
%(based on work published in \citet{Vanderplas2011}). We develop a fast,
%KL-based approach to the three-dimensional reconstruction of matter
%maps from shear surveys. This KL approach solves the lensing inversion 
%problem using truncation of eigenvalues within the context of generalized 
%least squares estimation, without
%a priori assumptions about the statistical nature of the signal. 
%It also allows a quantitative
%comparison between different filtering methods: we evaluate our 
%method beside the previously
%explored Wiener-filter approaches.
%Our method yields near-optimal angular resolution
%of the lensing reconstruction and allows cluster sized halos to be 
%de-blended robustly.
%It allows for mass reconstructions which are two to three orders of 
%magnitude faster than
%the Wiener-filter approach; in particular, we estimate that an 
%all-sky reconstruction with arcminute
%resolution could be performed on a timescale of hours. We find 
%however that linear,
%non-parametric reconstructions have a fundamental limitation in 
%the resolution achieved in the redshift direction.

%\item {\bf Mapping and shear peak statistics with noisy and incomplete data.}
% (based on work published
%in \citet{Vanderplas2012}). We use KL analysis to solve a 
%practical problem in the analysis
%of gravitational shear surveys. Shear catalogs from 
%large-field weak-lensing surveys
%will be subject to many systematic limitations, notably 
%incomplete coverage and pixel-level
%masking due to foreground sources. We develop a method to 
%use two-dimensional KL eigenmodes
%of shear to interpolate noisy shear measurements across 
%masked regions. We explore
%the results of thismethod with simulated shear catalogs, 
%using statistics of high-convergence
%regions in the resulting map. We find that the KL 
%procedure not only minimizes the bias due
%to masked regions in the field, it also reduces spurious 
%peak counts from shape noise by a
%factor of 3 in the cosmologically sensitive regime. This 
%indicates that KL reconstructions of
%masked shear are not only useful for creating robust 
%convergence maps from masked shear
%catalogs, but also offer promise of improved parameter 
%constraints within studies of shear peak statistics.

%\item {\bf Two-point parameter estimation with noisy and incomplete data.}
%The two point statistics
%of shear, namely the power spectrumand correlation function, 
%scale with cosmology in ways
%that are predictable using a combination of analytic models and 
%simulations of the universe.
%As such, these measures can give us fundamental insight into the 
%nature of the universe.
%When measuring the power spectrum directly, gaps in the data can 
%lead to a situation where
%Fourier modes are no longer orthogonal. This can lead to mode 
%mixing which is complicated
%to correct. We use KL analysis to construct an alternative 
%orthonormal basis in which
%to record this cosmological information. In addition to addressing 
%the mode mixing problem,
%KL allows for robust noise filtration and direct comparison to 
%cosmological models, all
%within a very fast matrix-based framework.
%\end{enumerate}
}
 
%
% ----- contents & etc.
%
\tableofcontents
\listoffigures
\listoftables
 
%
% ----- glossary 
%
%\chapter*{Glossary}      % starred form omits the `chapter x'
%\addcontentsline{toc}{chapter}{Glossary}
%\thispagestyle{plain}
%
%\begin{glossary}
%\item[item1] description description description description description
%  description description description description description description
%  description description description description description description
%\item[item2] description description description description description
%  description description description description description description
%  description description description description description description
%\item[item3] description description description description description
%  description description description description description description
%  description description description description description description 
%\end{glossary}
 
%
% ----- acknowledgments
%
\acknowledgments{% \vskip2pc
   {\narrower\noindent
   Thanks first to my wife Cristin for supporting me during the last five
   years, and understanding my late nights of working and writing.  Her
   genuine and deep love for the people in her life is a daily inspiration.

   Thanks to my family: especially to my mother Gretchen for her constant
   encouragement and support.  From collecting butterflies to growing   
   carnivorous plants to counting the stars, she instilled in me a desire
   to know the world around me.

   Thanks to my advisors, Andrew Connolly and Bhuvnesh Jain, for their
   patient mentorship.  Their thoughtfulness, humor, and expectation of
   excellence has given me a graduate school experience that exceeded even
   my highest expectations.

   Thanks to Andrew Becker for offering detailed and insightful comments
   on a draft of this work.

   I am indebted to several colleagues for helpful discussions through the
   course of this research, including Debbie Bard, Gary Bernstein, Anna Cabre,
   Joerg Dietrich, Mike Jarvis, Jan Kratochvil, Tim Schrabback, Patrick Simon,
   Andy Taylor, Vinu Vikram, Risa Wechsler, and many others.

   Support for this research was provided by DOE Grant DESC0002607,
   NSF Grant AST-0709394, and NASA Grant NNX07-AH07G.
   \par}
}

%
% ----- dedication
%

\dedication{
\begin{center}
To my father, Hugh Vander Plas, who lost his battle with cancer just weeks
before I presented this dissertation.  I will always be thankful for his
ceaseless support, encouragement, and willingness to listen.  His loyal,
loving, and faithful character will remain my highest example of how to live.
\end{center}
}

%
% end of the preliminary pages
 
 
