% tutorial.tex  -  a short descriptive example of a LaTeX document
%
% For additional information see  Tim Love's ``Text Processing using LaTeX''
% http://www-h.eng.cam.ac.uk/help/tpl/textprocessing/
%
% You may also post questions to the newsgroup <b> comp.text.tex </b> 

\documentclass[12pt,preprint]{aastex}			% For LaTeX 2e
						% other documentclass options:
						% draft, fleqn, openbib, 12pt

\usepackage{graphicx}	 			% insert PostScript figures
\usepackage{amsfonts}
%% \usepackage{setspace}   % controllabel line spacing
%% If an increased spacing different from one-and-a-half or double spacing is
%% required then the spacing environment can be used.  The spacing environment 
%% takes one argument which is the baselinestretch to use,
%%         e.g., \begin{spacing}{2.5}  ...  \end{spacing}


% the following produces 1 inch margins all around with no header or footer
%\topmargin	=10.mm		% beyond 25.mm
%\oddsidemargin	=0.mm		% beyond 25.mm
%\evensidemargin	=0.mm		% beyond 25.mm
%\headheight	=0.mm
%\headsep	=0.mm
%\textheight	=220.mm
%\textwidth	=165.mm
					% SOME USEFUL OPTIONS:
% \pagestyle{empty}			% no page numbers
% \parindent  15.mm			% indent paragraph by this much
% \parskip     2.mm			% space between paragraphs
% \mathindent 20.mm			% indent math equations by this much

\newcommand{\MyTabs}{ \hspace*{25.mm} \= \hspace*{25.mm} \= \hspace*{25.mm} \= \hspace*{25.mm} \= \hspace*{25.mm} \= \hspace*{25.mm} \kill }

\graphicspath{{../Figures/}{../data/:}}  % post-script figures here or in /.

					% Helps LaTeX put figures where YOU want
 \renewcommand{\topfraction}{0.9}	% 90% of page top can be a float
 \renewcommand{\bottomfraction}{0.9}	% 90% of page bottom can be a float
 \renewcommand{\textfraction}{0.1}	% only 10% of page must to be text

\newcommand{\rcom}{\chi}     %comoving distance
\newcommand{\dd}{\mathrm{d}} %differential for integrals

\alph{footnote}				% make title footnotes alpha-numeric

\title{3D Lensing Tomography}	% the document title

\author{Jake VanderPlas\\
  \texttt{vanderplas@astro.washington.edu}}

\date{\today}				% your own text, a date, or \today

% --------------------- end of the preamble ---------------------------

\begin{document}			% REQUIRED
\bibliographystyle{apj}

%\maketitle				% you need to define \title{..}

\section{Introduction}
This is a summary of weak lensing theory, starting from first principles, in 
order to solidify the theoretical/statistical basis for the reconstruction of
3D mass distributions based on propagation of PDFs for lensing observables.

\section{Cosmological Lensing: Simplifying Assumptions}
\label{simplifying_assumptions}
The propagation of light through a region of gravitational potential $\Phi(\vec{r})$ is, in general, a very complicated problem, only analytically solvable for potentials with various symmetries.  In cosmological contexts, however, it is safe to assume that the universe is described by a Robertson-Walker metric, with only small perturbations due to the density fluctuations described by the potential $\Phi$.  In this case, the gravitational deflection of a photon can be described by an effective index of refraction given by \citep[see][and references therein]{Narayan96}:
\begin{equation}
  n = 1-\frac{2}{c^2}\Phi 
\end{equation}
As in conventional optics, light rays are deflected in proportion to the perpendicular gradient of the refraction index, such that the deflection $\hat{\alpha}$ is given by
\begin{equation}
  \label{alpha-def}
  \hat{\alpha} = -\int_0^{D_S} \vec{\nabla}_\perp n \dd D = \frac{2}{c^2}\int_0^{D_S} \vec{\nabla}_\perp\Phi \dd D
\end{equation}
where $D_S$ is the distance from the observer to the photon source.  

For a point-mass located at a distance $D_L$ and an impact parameter $b$, with $D_L \gg b$ and $D_S \approx 2D_L$, equation \ref{alpha-def} can be integrated to give
\begin{equation}
  \hat{\alpha} = \frac{4GM}{bc^2}\Bigg[1 - \frac{1}{2}\bigg(\frac{b}{D_L}\bigg)^2 + \mathcal{O}\bigg[\frac{b}{D_L}\bigg]^3 \Bigg]
\end{equation}
The first-order approximation is twice the deflection predicted by Newtonian gravity for a particle of arbitrary mass moving at a speed $c$.  It is important to note here that to first order, the deflection does not depend on the distance to the lens or source.  That is, for a mass distribution located at a distance $D_L$, equation \ref{alpha-def} can be approximated
\begin{equation}
  \label{alpha-def-approx}
  \hat{\alpha} \approx \frac{2}{c^2}\int_{D_L-\delta D}^{D_L+\delta D} \vec{\nabla}_\perp\Phi\,\dd D
\end{equation}
for $\delta D$ sufficiently greater than the size scale of the mass-distribution in question.  Also note that for a large impact parameter $b$, the contribution to $\hat{\alpha}$ becomes vanishingly small, and can be neglected.  So, to a very good approximation, the incremental deflection $\delta \hat{\alpha}$ of a photon at a given point along its trajectory is entirely due to a sheet of matter with a width $2\delta D$, oriented perpendicular to the unperturbed photon trajectory.  

\section{Lensing Geometry}
For a mass-sheet located at a distance $D_L$, and a photon source located at a distance $D_S$ (with $D_{LS} = D_S - D_L$) geometric considerations in the small-angle approximation yield the relation
\begin{equation}
  \vec{\theta} = \vec{\beta} + \frac{D_{LS}}{D_S}\hat{\vec{\alpha}}
\end{equation}
where $\vec{\theta}$ and $\vec{\beta}$ are the observed and true positions of the source, respectively.  Rescaling $\hat{\vec{\alpha}}$ in more convenient units gives
\begin{equation}
  \label{mapping}
  \vec{\theta} = \vec{\beta} + \vec{\alpha}
\end{equation}
where we have defined
\begin{equation}
  \label{alpha-def2}
  \vec{\alpha} \equiv \frac{D_{LS}}{D_S}\hat{\vec{\alpha}}
\end{equation}

\section{Continuous Mass Distribution}
In the case of a continuous mass distribution, we can recall the remarks of section \ref{simplifying_assumptions}, and define a surface-mass density for a mass-sheet located at a redshift $z_L$:
\begin{equation}
  \label{sigma-def}
  \Sigma(\vec{\theta},z_L) 
  = \int_{D_L-\delta D}^{D_L+\delta D}\rho_M(\vec{\theta},D)\dd D 
  = \frac{1}{c^2}\int_{z_L - \delta z}^{z_L + \delta z} 
  \varepsilon_M(\vec{\theta},z)\frac{dD}{dz}\dd z
\end{equation}
where $\varepsilon_M \equiv \rho_M c^2$ is the energy density of matter,
$\vec{\theta}$ is the apparent angular position, and $z$ and $D(z)$ are the 
redshift and line-of-sight distance, respectively, with $D_S = D(z_s)$.  

A matter distribution $\rho_M(\vec{\theta},z)$, and its newtonian potential $\Phi(\vec{\theta},z)$ are related by Poisson's equation:
\begin{equation}
  \label{poisson}
  \nabla^2 \Phi(\vec{\theta},z) = 4\pi G \rho_M(\vec{\theta},z)
\end{equation}

It is convenient to define the unscaled lensing potential $\hat{\psi}$, 
given by
\begin{equation}
  \hat{\psi}(\vec{\theta},z_s) 
  = \int_{0}^{D(z_s)} \Phi(\vec{\theta},z(D))\dd D 
  = \int_{0}^{z_s}\Phi(\vec{\theta},z)\frac{dD}{dz}\dd z
\end{equation}
Using the approximation in equation \ref{alpha-def-approx}, we can write this in terms of multiple mass-sheets, such that
\begin{equation}
  \hat{\psi} = \sum_i \delta \hat{\psi}_i 
  = \sum_i \int_{D_i-\delta D}^{D_i+\delta D}\Phi \dd D
\end{equation}
with $D_{i+1} = D_i + 2*\delta D$.

The gradient of $\hat{\psi}$ with respect to 
$\vec{\xi} \equiv D_L\vec{\theta}$ is
\begin{equation}
  \vec{\nabla}_\xi\hat{\psi} 
  = \sum_i \vec{\nabla}_\xi\big(\delta\hat{\psi}_i\big) 
  = \sum_i \int_{D_i - \delta D}^{D_i+\delta D}\vec{\nabla}_\xi \Phi \dd D
\end{equation}
Comparing this with (\ref{alpha-def-approx}) and (\ref{alpha-def2}) gives the incremental deflection angle in terms of the lensing potential of a mass-sheet:
\begin{equation}
  \delta\vec{\alpha}_i = \frac{2}{c^2}\frac{D_{LS}}{D_S}\vec{\nabla}_\xi (\delta\hat{\psi}_i)
\end{equation}
Further simplification can be made by using the rescaling $\vec{\xi} = D_L \vec{\theta}$ and
\begin{equation}
  \delta\psi_i = \frac{2}{c^2}\frac{D_{LS}}{D_L D_S} \delta\hat{\psi}_i
\end{equation}
so that we are left with
\begin{equation}
  \label{alpha-psi}
  \delta\vec{\alpha}_i(\vec{\theta},z_L) = \vec{\nabla}_\theta \Big( \delta\psi_i(\vec{\theta},z_L)\Big)
\end{equation}
Defining the total scaled lensing potential $\psi = \sum_i \delta\psi_i$, and the total deflection $\vec{\alpha} = \sum_i\delta\vec{\alpha}_i$, we obtain
\begin{equation}
  \vec{\alpha}(\vec{\theta},z_s) = \vec{\nabla}_\theta \psi(\vec{\theta},z_s)
\end{equation}

The Laplacian of $\delta\psi_i$ with respect to theta is given by
\begin{equation}
  \nabla_\theta^2 (\delta\psi_i) 
  = \frac{2}{c^2}\frac{D_{LS}D_L}{D_S}\int_{D_L-\delta D}^{D_L+\delta D} 
  \nabla_\xi^2\Phi \dd D
\end{equation}
Using (\ref{sigma-def}) and (\ref{poisson}) this becomes
\begin{equation}
  \label{psi-sigma_init}
  \nabla_\theta^2(\delta\psi_i) = \frac{8\pi G}{c^2}\frac{D_{LS}D_L}{D_S}\Sigma(\vec{\theta},z_i)
\end{equation}

We now define the critical surface density,
\begin{equation}
  \Sigma_{c}(z) \equiv\frac{c^2 D_S}{4\pi G D_L(z) D_{LS}(z)}
\end{equation}
and the convergence
\begin{equation}
  \label{kappa-sigma}
  \kappa(\vec{\theta},z_s) \equiv \sum \frac{\Sigma(\vec{\theta},z_i)}{\Sigma_c(z_i)}\ \forall\ z_i < z_s
\end{equation}
Now summing all the mass-sheets in (\ref{psi-sigma_init}) gives the relation between the scaled lensing potential and the convergence
\begin{equation}
  \label{psi-kappa-1}
  \nabla_\theta^2\psi(\vec{\theta},z_s) = 2\kappa(\vec{\theta},z_s)
\end{equation}

Solving this two-dimensional differential equation gives the effective potential in terms of the convergence:
\begin{equation}
  \label{psi-kappa}
  \psi(\vec{\theta},z) 
  = \frac{1}{\pi}\int_{\mathbb{R}^2} \kappa(\vec{\theta}^\prime,z) 
  \ln|\vec{\theta} - \vec{\theta}^\prime|\dd^2\theta^\prime
\end{equation}

\section{Weak Lensing Formalism}
The local properties of the mapping in (\ref{mapping}) are contained in its Jacobian matrix, given by
\begin{equation}
  \label{Jacobian_def}
  \mathcal{A} \equiv \frac{\partial \vec{\beta}}{\partial \vec{\theta}} = \Big(\delta_{ij} - \frac{\partial \alpha_i}{\partial \theta_j} \Big) = \Big(\delta_{ij} - \frac{\partial^2 \psi}{\partial \theta_i \partial \theta_j} \Big)
\end{equation}

Introducing the abbreviation
\begin{equation}
  \label{psi_ij}
  \psi_{ij} = \frac{\partial^2 \psi}{\partial \theta_i \partial \theta_j}
\end{equation}
We can then rewrite the convergence $\kappa$ (eqn \ref{psi-kappa-1}) and define the complex shear $\gamma \equiv \gamma_1 + i\gamma_2$ of the mapping:
\begin{equation}
  \label{gamma-def}
  \begin{array}{lcl}
    \kappa & = & (\psi_{11} + \psi_{22})/2\\
    \gamma_1 & = & (\psi_{11} - \psi_{22})/2\\
    \gamma_2 & = & \psi_{21} = \psi_{12}
  \end{array}
\end{equation}
The local Jacobian matrix (\ref{Jacobian_def}) of the lens mapping can then be written
\begin{equation}
  \mathcal{A} = \left(
  \begin{array}{cc}
    1 - \kappa - \gamma_1 & -\gamma_2\\
    -\gamma_2             & 1-\kappa+\gamma_1
  \end{array}\right)
\end{equation}

Equations \ref{psi-kappa}, \ref{psi_ij} and \ref{gamma-def} can be combined and simplified to yield the following relationship between the convergence and the shear, where for simplicity we define the complex angle $\theta = \theta_1 + i\theta_2$:
\begin{equation}
  \label{gamma-kappa}
  \gamma(\theta) 
  = \frac{-1}{\pi}\int_{\mathbb{R}^2} \mathcal{D}(\theta - 
  \theta^\prime)\kappa(\theta^\prime) \dd^2\theta^\prime
\end{equation}
where 
\begin{equation}
  \label{scriptD}
  \mathcal{D}(\theta \equiv \theta_1+i\theta_2) 
  = \frac{\theta_1^2 - \theta_2^2 + 2i\theta_1\theta_2}{(\theta_1^2+\theta_2^2)^2}
  = \frac{\theta^2}{|\theta|^4}
\end{equation}

In weak lensing, the ellipticities of source galaxies are measured, 
giving a noisy estimate of the shear $\gamma(\theta)$.  The problem becomes, 
given this shear estimate, to recover a local estimate of 
$\kappa(\theta)$ using equation \ref{gamma-kappa}, and then an estimate 
of the projected mass $\Sigma(\theta)$ via equation \ref{kappa-sigma}.  
For 3D tomography, this procedure is repeated at various redshifts to recover 
an estimate of the 3D mass density, $\rho(\theta,z)$. It helps to write 
the expression for $\kappa(\theta,z)$ in terms of $\rho(\theta,z)$
 explicitly.  From (\ref{sigma-def}) and (\ref{kappa-sigma}), approximating 
the sum as an integral, we find
\begin{equation}
  \kappa(\vec{\theta},z_s) 
  = 4\pi G \int_0^{z_s} 
  \frac{D^{(A)}(z)[D^{(A)}(z_s)-D^{(A)}(z)]}{D^{(A)}(z_s)} 
  \rho_M(\vec{\theta},z) \frac{dD^{(A)}(z)}{dz} \dd z
\end{equation}
The notation has been changed here to make clear that the distances in 
question are in fact angular diameter distance, the relevant distance 
in the context of lensing calculations.  Recall that angular diameter 
distance $D^{(A)}$ is related to the comoving distance $D$ by
\begin{equation}
  D^{(A)}(z) = a S_\kappa (D)
\end{equation}
where $S_\kappa(D) = D$ for a flat universe.  Converting the angular 
diameter distances to comoving distances, and writing this in terms 
of $\varepsilon = \rho c^2$, we find
\begin{equation}
  \label{kappa-epsilon-1}
  \kappa(\vec{\theta},z_s) 
  = \frac{4\pi G}{c^2} \int_0^{z_s} \dd z\frac{dD}{dz} 
  a^2\frac{D(D_S-D)}{D_S} \varepsilon_M(\vec{\theta},z) 
\end{equation}
here we've used the shorthand $D \equiv D(z)$ and 
$D_S \equiv D(z_s)$.

It is instructive to expand the energy density in equation 
\ref{kappa-epsilon-1} about the background matter density.  
We define $\delta(z)$, the density contrast, such that
\begin{equation}
  \label{delta-def}
  \varepsilon_M(\vec{\theta},z) = \Omega_M(z) \varepsilon_c(z)\Big[1+\delta(\vec{\theta},z)\Big]
\end{equation}
where $\varepsilon_c(z)$ is the critical density, described by the Friedmann equation:
\begin{equation}
  \label{H-def}
  \varepsilon_c(z) = \frac{3 \left[H(z)\right]^2 c^2}{8\pi G}
\end{equation}
Note also that the change of comoving distance $D$ with redshift $z$ is given by
\begin{equation}
  \label{dDdz}
  \frac{dD}{dz} = \frac{c}{H(z)}
\end{equation}
and that the matter density fraction can be written
\begin{equation}
  \Omega_M(z) = \frac{H_0^2\Omega_{M,0}(1+z)^3}{[H(z)]^2}
\end{equation}


Combining these equations gives
\begin{equation}
  \kappa(z_s) = \frac{3cH_0^2\Omega_{M,0}}{2}\int_0^{z_s} \dd z \frac{(1+z)}{H(z)} \frac{D(D_S-D)}{D_S}\big[1+\delta(z)\big]
\end{equation}

Because of the mass-sheet degeneracy, $\kappa(z_s)$ can only be determined up to an additive constant across a given redshift bin \citep[see][for discussion]{Seitz_Schneider96}.  Defining $\bar{\kappa}(z_s)$ to be the convergence due to matter-dominated growth and $\Delta(z) \equiv \delta(z)/a$ (see Appendix \ref{matter_dominated_growth} for motivation behind this change of variables) we find
\begin{equation}
  \label{kappa-delta}
  \kappa(z_s) \equiv \hat{\kappa}(z_s)-\bar\kappa(z_s) = 
  \frac{3cH_0^2\Omega_{M,0}}{2}\int_0^{z_s} \dd z 
  \frac{1}{H(z)} \frac{D(D_S-D)}{D_S}\Delta(z)
\end{equation}
where, to be explicit,
\begin{equation}
  \bar\kappa(z_s) = \frac{3cH_0^2\Omega_{M,0}}{2}\int_0^{z_s} \dd z 
  \frac{(1+z)}{H(z)} \frac{D(D_S-D)}{D_S}
\end{equation}
To be clear, here, $D$ is the comoving distance to a redshift $z$, and $D_S$ is the comoving distance to the redshift $z_s$ of the photon source.  Equation \ref{kappa-delta} defines the mapping from $\kappa(z_s)$ to $\Delta(z)$ for $z<z_s$.  

\section{Inversion}
Equations \ref{gamma-kappa} and \ref{kappa-delta}:
\begin{equation}
  \gamma(\theta) 
  = \frac{-1}{\pi}\int_{\mathbb{R}^2} \mathcal{D}(\theta - 
  \theta^\prime)\kappa(\theta^\prime) \dd^2\theta^\prime
\end{equation}

\begin{equation}
  \kappa(z_s) \equiv \hat{\kappa}(z_s)-\bar\kappa(z_s) = 
  \frac{3cH_0^2\Omega_{M,0}}{2}\int_0^{z_s} \dd z 
  \frac{1}{H(z)} \frac{D(D_S-D)}{D_S}\Delta(z)
\end{equation}
must be inverted in order to determine the mass distribution based on shear measurements.  \citet{Kaiser_Squires93} give an analytical inversion of (\ref{gamma-kappa}).  Due to edge-effects for a finite observation field, however, this inversion is not exact.  \citet{Seitz_Schneider96} present a few strategies to deal with this explicitly.  Another strategy, developed in \citet{Hu02} and brought further in \citet{Simon09}, makes use of the fact that observations are discrete rather than continuous, so that the integrals in equations \ref{gamma-kappa} and \ref{kappa-delta} are approximated by discrete sums.  The two equations then become linear mappings, which can be inverted by well-known linear algebra techniques.  This is the strategy we will pursue here.
\subsection{Fourier Space Inversion}
Equation \ref{gamma-kappa} shows that $\gamma(\theta)$ is simply a 2D
convolution of $\kappa(\theta)$ with $\mathcal D(\theta)$.  With this in mind,
we may proceed using the convolution theorem.  Given the $n$-dimensional
fourier transform pair of a given function,
\begin{eqnarray}
  \label{ft}
  g(\vec x) = \int \frac{\dd^nk}{(2\pi)^n} \tilde{g}(\vec k)
  e^{-i\vec k\cdot\vec x}\\
  \label{ift}
  \tilde{g}(\vec k) = \int \dd^nx g(\vec x)e^{i\vec x\cdot\vec k}
\end{eqnarray}
it can be shown that equation \ref{gamma-kappa} can be equivalently expressed
\begin{equation}
  \tilde \gamma(\ell) 
  = \frac{-1}{\pi} \widetilde{\mathcal{D}}(\ell)\tilde\kappa(\ell)
\end{equation}

\subsection{Real Space Inversion}
Following \citet{Hu02}, we define $\phi = \arctan (\theta_2/\theta_1)$ so that equations \ref{gamma-kappa}-\ref{scriptD} can be rewritten in a more suggestive way:
\begin{equation}
  \label{gamma-kappa-new}
  [\gamma_1 \pm i\gamma_2](\vec{\theta}) = -\frac{1}{\pi}\int_{\mathbb{R}^2} d^2\theta^\prime\frac{\cos(2i\phi) \pm i\sin(2i\phi)}{|\vec{\theta}-\vec{\theta}^\prime|^2}\kappa(\vec{\theta}^\prime)
\end{equation}
We discretize this equation into $N$ pixels, such that pixel $n$ has an area $A^{(n)}$, shear $\gamma^{(n)} = \gamma_1^{(n)} + i\gamma_2^{(n)}$, and convergence $\kappa^{(n)}$.  We define the data vector $\vec{d}_\gamma$ such that
\begin{equation}
  \left.
  \begin{array}{ll}
    \{d_\gamma\}_{2i-1} &= \gamma_1^{(i)}\\  
    \{d_\gamma\}_{2i}   &= \gamma_2^{(i)}
  \end{array}
  \right\}1\le i\le N
\end{equation}
and the projection matrix $\mathbf{P_{\gamma\kappa}}$ such that
\begin{equation}
  \label{P-gamma-kappa}
  \left.
  \begin{array}{ll}
    \{\mathbf{P_{\gamma\kappa}}\}_{2i-1,j} = -\frac{A_j}{\pi}\frac{\cos{2\phi_{ij}}}{\theta_{ij}^2}\\
    \{\mathbf{P_{\gamma\kappa}}\}_{2i,j} = -\frac{A_j}{\pi}\frac{\sin{2\phi_{ij}}}{\theta_{ij}^2}
  \end{array}
  \right\}1\le i,j\le N
\end{equation}
where $(\theta_{ij},\phi_{ij})$ are the polar coordinites separating bins $i$ and $j$ in the plane of the sky.  The discretized version of the mapping equation (\ref{gamma-kappa-new}) becomes
\begin{equation}
  \label{mapping-gamma-kappa}
  \vec{d}_\gamma = \mathbf{P_{\gamma\kappa}}\cdot\vec{s}_\kappa + \vec{n}_\gamma
\end{equation}
which can be straightforwardly solved using, e.g. QR-decomposition (see \citet{Hu02} for the details on inversions of linear maps in this context).

\section{Utilization of redshift information}
\label{3D-formalism}
If the source galaxies which are used as shear estimators have measured constraints on their redshift $z$, then the mapping inversion in the previous section can be repeated for each source redshift bin.  This yields a collection of estimators $\vec{s}_\kappa (z_i)$ at different redshifts.  Let the convergence estimates along a given line of sight (i.e. within each of the $N$ pixels defined above) be given by $\vec{d}_\kappa$.  Equation \ref{kappa-delta} gives the continuous mapping from this new data vector to the signal vector of density contrasts, $\vec{s}_\Delta$.  Following a procedure similar to above, we define the projection matrix characterizing the discretization of (\ref{kappa-delta}):
\begin{equation}
  \label{P-kappa-delta}
  [\mathbf{P_{\kappa\Delta}}]_{jk} =  \frac{3cH_0^2\Omega_{M,0}}{2}
  \times\left\{
  \begin{array}{ll}
   \frac{1}{H(z_k)} \frac{(D_j-D_k)D_k}{D_j}\delta z_k, & z_k < z_j\\
    0, & z_k \ge z_j
  \end{array}
  \right.
\end{equation}
and the linear mapping is given by
\begin{equation}
  \label{mapping-kappa-delta}
  \vec{d}_\kappa = \mathbf{P_{\kappa\Delta}}\cdot\vec{s}_\Delta + \vec{n}_\kappa
\end{equation}
This, as above, can be inverted using the techniques in \citet{Hu02}.  In particular, for a well-behaved $\mathbf{P_{\kappa\Delta}}$, the solution is given by the estimator
\begin{equation}
  \hat{\vec{s}}_\Delta = \mathbf{R_{\Delta\kappa}} \vec{d}_\kappa
\end{equation}
where 
\begin{equation}
  \mathbf{R_{\Delta\kappa}} = 
	 [\mathbf{P}^T_{\kappa\Delta} \mathbf{N}^{-1}_{\kappa\kappa}
	   \mathbf{P}_{\kappa\Delta}]^{-1}
	 \mathbf{P}^T_{\kappa\Delta}\mathbf{N}^{-1}_{\kappa\kappa}
\end{equation}
where $\mathbf{N}_{\kappa\kappa}$ is the noise covariance:
\begin{equation}
  \mathbf{N}_{\kappa\kappa} \equiv \langle \vec{n}_\kappa \vec{n}^T_\kappa \rangle .
\end{equation}
Note the importance of choosing well-behaved redshift bins: if $\max(z_\kappa) 
\le \max(z_\Delta)$, then $\mathbf{P_{\kappa\Delta}}$ will be singular and 
the inversion will not be well-posed.  

In case of large noise, and where the 
statistical properties of the signal matrix are known, a ``Wiener Filter''
can be used by adding a penalty function to the $\chi^2$ minimization: $\chi^2 \to \chi^2 + H$, where
\begin{equation}
  H = \left(\hat{\vec{s}}^{(w)}_\Delta\right)^T \mathbf{S}^{-1}_{\Delta\Delta} \hat{\vec{s}}^{(w)}_\Delta
\end{equation}
with
\begin{equation}
  \label{SDD}
  \mathbf{S}_{\Delta\Delta} \equiv \langle \vec{s}_\Delta \vec{s}_\Delta^T \rangle
\end{equation}
such that the minimization leads to
\begin{equation}
  \label{Wiener_R}
  \mathbf{R}^{(w)}_{\Delta\kappa} = \mathbf{S}_{\Delta\Delta}\mathbf{P}^T_{\kappa\Delta} \left[\mathbf{P}_{\kappa\Delta}\mathbf{S}_{\Delta\Delta}\mathbf{P}^T_{\kappa\Delta} + \mathbf{N}_{\kappa\kappa} \right]^{-1}
\end{equation}
with the Wienter-filtered estimator
\begin{equation}
  \label{Wiener_s}
  \hat{\vec{s}}^{(w)}_\Delta = \mathbf{R^{(w)}_{\Delta\kappa}} \vec{d}_\kappa.
\end{equation}
This form of the solution is much more robust to noise in the measurements.
 
\section{Generalization to Arbitrary Redshift PDFs}
The above method is equivalent to that outlined in \citet{Hu02}.  The weakness of this procedure is that it assumes a known redshift for each galaxy, i.e. a probability distribution for galaxy $g$ given by a delta-function
\begin{equation}
  P_g(z) = \delta_D(z-\bar{z}_g)
\end{equation}
or its discretely-binned equivalent.  We would like to extend this analysis 
to make use of the full observed reshift information for an arbitrary $P_g(z)$.

The shape of redshift PDFs for various galaxies is determined primarily by 
their spectral type, their redshift, the colors observed, and the noise 
characteristics of the observation.  For a homogeneous survey using a single 
photometric redshift technique, the characteristics of a redshift PDF (i.e. 
tails, bimodalism, etc) across the observed field will be determined entirely 
by galaxy redshift and spectral type, regardless of location on the plane of 
the sky.  This suggests a generalization of the concept of a redshift ``bin'' 
to a set of galaxies across the sky with similar PDFs.  The degree of 
similarity can be determined based on the amount of data, desired resolution, 
noise constraints, etc.  (This will be an area for future work: perhaps we 
can categorize pdfs based on a PCA approach)

Consider now a survey divided into  $n_p$ pixels, indexed by $1\le i \le n_p$, 
and $n_g$ redshift bins, indexed by $1\le j \le n_g$.  Each bin at pixel $i$ 
and redshift bin $j$ contains $N_{ij}$ galaxies, with an average observed 
complex shear estimate $\gamma_{ij}$, and an average redshift probablility 
distribution $p_{ij}(z)$ given by the mean of the individual galaxy PDFs 
$p_g(z)$ within the bin:
\begin{equation}
  p_{ij}(z) = \frac{1}{N_{ij}}\sum_{g=1}^{N_{ij}}p_g(z)
\end{equation}
The density contrast $\Delta$ (eqn. \ref{kappa-delta}) will be determined in $n_z$ discrete bins per pixel, indexed by $1\le k\le n_z$.  Bin $k$ spans the redshifts $z_k \le z < z_{k+1}$.

The $\gamma$ to $\kappa$ mapping in equation \ref{mapping-gamma-kappa} is 
performed for each of the $n_g$ galaxy bins for a given line-of-sight.  This 
gives an estimate of $\vec{d}_{\kappa,i}$ for each pixel, where each 
convergence estimate $\kappa_{ij}$ has a corresponding redshift probability 
distribution $p_{ij}(z)$, normalized such that $\int p_{ij}(z)dz=1$.  
The line-of-sight mapping matrix $\mathbf{P_{\kappa\Delta}}$ 
(eqn. \ref{P-kappa-delta}) for pixel $i$, then becomes
\begin{equation}
  \label{P-kappa-delta-pdf}
  [\mathbf{P_{\kappa\Delta}}^{(i)}]_{jk} = 
  \frac{3cH_0^2\Omega_{M,0}}{2}\delta z_k
  \int_{z_k}^{\infty} \frac{p_{ij}(z)}{H(z)}\frac{(D(z)-D_k)D_k}{D(z)}dz
\end{equation}
This formula for $\mathbf{P_{\kappa\Delta}}^{(i)}$ reflects the fact that a 
given galaxy bin is ``smeared'' in redshift by its empirical redshift 
PDF.  The matter at redshift $z_k$, in a statistical sense, affects the shear
of all galaxies with a nonzero probability of having a redshift 
$z_{\rm gal} > z_k$.  Setting $p_{ij}(z) = \delta_D(z-z_j)$ recovers the 
form of equation \ref{P-kappa-delta}.

\section{A Toy Model}
\subsection{Density Distribution}
To show the utility of a procedure, we may create a toy model along a particular line-of-sight, assuming that $\kappa$ has been accurately reconstructed based on non-local shear data.  

As shown in \citet{Hu02}, an accurate 3D reconstruction can be performed only
if the noise and covariance properties of the distribution are known.  We can
draw our simulated data from the gaussian distribution defined by the mass
power spectrum, given by 
\begin{equation}
  \label{power-spec}
  P_k(z) = P_0 \cdot k [T_kG(z)]^2
\end{equation}
where $P_0$ is the normalization constant which will be determined
empirically, $G(z)$ is the linear growth factor defined in Appendix 
\ref{matter_dominated_growth}, and we have assumed that the 
primordial power law follows the Harrison-Zel'dovich spectrum
$P_k \sim k$, in agreement with most inflationary scenarios.  The transfer
function $T_k$ is defined empirically by the following functional form, 
taken from Equation  15.82-1 in \citet{Peacock} 
\citep[The form is due to][]{Bardeen86}:
\begin{equation}
  \label{transfer-function}
  T_k = \frac{\ln(1+2.34q)}{2.34q}\left[ 1 + 3.89q + (16.1q)^2 + 
    (5.46q)^3 + (6.71q)^4 \right]^{-1/4}
\end{equation}
where $q = k/(\Omega_M h^2)$, with $k$ measured in Mpc$^{-1}$. 
For historical reasons, the normalization constant $P_0$ is commonly 
defined in terms of the parameter $\sigma_8$, which is defined as the density 
fluctuation within a sphere of radius 8 Mpc.  To compute this, we use a 
top-hat window function:
\begin{equation}
\label{top-hat}
  W_T(\mathbf{x}/R) = \left\{
    \begin{array}{ll}
      1, & x/R \le 1 \\
      0, & x/R > 1
    \end{array}
    \right.
\end{equation} 
The fourier transform of an arbitrary window function is given by
\begin{equation}
  \label{window-transform}
  \widetilde{W}(\mathbf{k}R) \equiv \frac{\int d^3\mathbf{x} W(\mathbf{x}/R) e^{i\mathbf{k}\cdot\mathbf{x}}}{\int d^3\mathbf{x} W(\mathbf{x}/R)}
\end{equation}
The density fluctuation $\sigma_W$ within a given window $W$ is given by
\begin{equation}
  \label{powerspec-sigma-def}
  \sigma_W^2 = \frac{1}{(2\pi)^3}\int d^3\mathbf{k} P_k W^2(\mathbf{k}R)
\end{equation}
where the window function is assumed to be shallow enough that there is no
cosmological evolution of the signal.

For the top-hat window function of equation \ref{top-hat}, 
the fourier transform given by Equation \ref{window-transform}
is (using $k = |\mathbf{k}|$)
\begin{equation}
  \label{top-hat-f}
  \widetilde{W}_T(kR) = \frac{3}{(kR)^3}\left[\sin(kR) - kR\cos(kR) \right]
\end{equation}
Putting together equations \ref{power-spec}, \ref{transfer-function},
\ref{powerspec-sigma-def}, and \ref{top-hat-f} and integrating numerically
allows us to determine $P_0$ as a function of $\sigma_8$:
\begin{equation}
  P_0 = \frac{8.24 \times 10^6}{[G(0)]^2}\sigma_8^2
\end{equation}
Using the WMAP5 measurement, $\sigma_8 = 0.812 \pm 0.026$ \citep{Hinshaw09}, 
we find the power-spectrum normalization $P_0 = 5.44 \times 10^6 [G(0)]^{-2}$.

This prepares us to simulate a matter distribution within an arbitrarily 
defined window.  Because we are doing radial maps, it is convenient to find 
an expression for the window function given a small pixel size, and two 
bounding radii $r_1$ and $r_2$.  Performing the integral in 
(\ref{window-transform}) for a square pixel of angular width
$\Delta \theta$, and a radial distance range $r_1$ to $r_2$ gives
\begin{equation}
  \widetilde{W}(k) = \frac{3}{kS_\theta(r_2^3 - r_1^3)} \int_{r_1}^{r_2}
    r \sin \left( kS_\theta r \right)dr
\end{equation} 
where $S_\theta = \sin(\Delta\theta/2)$.  Using this expression, the matter
density fluctuation within a given pixel and distance range can be determined 
using equation \ref{powerspec-sigma-def}.


\subsection{Statistical Properties of the Lensing Signal}
To solve the mapping using the Wiener-estimator of Equation \ref{Wiener_R}, we
need to determine the signal covariance $\mathbf{S}_{\Delta\Delta}$ defined
in equation \ref{SDD}.  The measurement within a bin $i$ is given by
\begin{equation}
  \vec{s}_{\Delta,i} = a_i^{-1}\int d^3x W_i(\vec{x})\delta(\vec{x}),
\end{equation}
where $W(\vec{x})$ is the window function associated with the bin.  Then we have
\begin{equation}
  \label{SDDij}
  \begin{array}{lll}
      [S_{\Delta\Delta}]_{ij} &=& \langle \vec{s}_{\Delta,i}\vec{s}_{\Delta,j}\rangle \\
                              &=& (a_i a_j)^{-1} \int d^3x_i \int d^3x_j W_i(\vec{x}_i)W_j(\vec{x}_j) \langle \delta(\vec{x}_i)\delta(\vec{x}_j)\rangle \\
                              &=& (a_ia_j)^{-1} \int \frac{d^3k}{(2\pi)^3}\widetilde{W}_i(\vec{k})\widetilde{W}^*_j(\vec{k})G_iG_jP_k(0)\\
                              &=& G_iG_j/(2 a_ia_j \pi^2) \int dk k^2 \widetilde{W}_i(k)\widetilde{W}^*_j(k)P_k(0)
  \end{array}
\end{equation}
where $P_k(0)$ is the linear power spectrum today, $W_i(\vec{k})$ is the 
fourier transform of the window function $W_i(\vec{x})$, and $G_i = G(z_i)$ 
is the linear growth factor (see Appendix \ref{matter_dominated_growth}).  
The statistical properties of the signal given in Equation \ref{SDDij} can 
be used to determine the Wiener-filtered solution, in Equations 
\ref{Wiener_R}-\ref{Wiener_s}.


\subsection{Simplification}
For simplicity, let's assume a matter dominated flat universe, $\Omega_M = 1$,
so that $G(a) = a$ and the comoving distance is given by
\begin{equation}
  D(z) = \int_0^z \frac{c}{H(z^\prime)}dz^\prime = \frac{2c}{H_0} \left(1-\frac{1}{\sqrt{1+z}} \right).
\end{equation}
Assume the line-of-sight in question has $N$ bins of 
equal width in redshift $\delta z$, and a pixel width of $2^\circ$ (above the 
$\sim 1^\circ$ threshold where the power spectrum becomes nonlinear).  Then
define a density field $\mathbf{s_\Delta}$ based on the gaussian-field 
distribution with dispersion given by equation \ref{powerspec-sigma-def}.
The observed $\kappa$ distribution is given (in the delta-function redshift
case) by $\mathbf{s_\kappa} = \mathbf{P_{\kappa\Delta}}\mathbf{s_\Delta} 
+ \mathbf{n_\kappa}$.  The methods of section \ref{3D-formalism} can then
be used to determine the reconstruction given various noise levels.  The
result of this is shown in figure \ref{simple-reconstruction}.  This shows
equivalent results to figure 1 in \citet{Hu02}

\begin{figure}
 \centering
 \includegraphics[width=0.8\textwidth]{simple.eps}
 \caption{Simple Reconstruction}
 \label{simple-reconstruction}
\end{figure}

 






%Equation \ref{kappa-epsilon-1} gives the mapping from density $\varepsilon(z)$ to the convergence $\kappa(z_s)$.  We will rewrite it here:
%\begin{equation}
%  \kappa(z_s) = \frac{4\pi G}{c^2} \int_0^{z_s} dz\frac{dD}{dz} a^2\frac{D(D_S-D)}{D_S} \varepsilon(z) 
%\end{equation}

%We will construct mass distributions along this line-of-sight from a number of isothermal spheres, defined by
%\begin{equation}
%  \rho(\vec{r}) = \frac{\rho_0}{|\vec{r}-\vec{r_0}|^2}
%\end{equation}
%There is a singularity at $\vec{r} = \vec{r_0}$, so it is desirable that our line of sight not pass through this point.  With this in mind, let $b$ be the impact parameter of the line of sight, and $x$ be the distance along this line of sight from $\vec{r}=0$.  Then the density along this line of sight is given by
%\begin{equation}
%  \rho_{\rm los}(D) = \frac{\rho_0}{a^2[(D-D_0)^2+b^2]}
%\end{equation}
%where we have converted to comoving distance.  Here, $D_0$ is the comoving distance from the observer to the point of closest approach to $\vec{r_0}$.  the convergence is then given by
%\begin{equation}
%  \kappa(z_s) = \frac{4\pi G\rho_0}{c^2} \int_0^{D_S} dD \frac{D(D_S-D)}{D_S[(D_0-D)^2+b^2]}
%\end{equation}
%This can be numerically integrated to find $\kappa(z_s)$.


%%%%%%%%%%%%%%%%%%%%%%%%%%%%%%%%%%%%%%%%%%%%%%%%%%%%%%%%%%%%%%%%%%%%%%%%%%%
%             begin bibliography

\bibliography{Lens3D}


%%%%%%%%%%%%%%%%%%%%%%%%%%%%%%%%%%%%%%%%%%%%%%%%%%%%%%%%%%%%%%%%%%%%%%%%%%%
%             begin Appendix  

\newpage
\appendix
\section{Matter-Dominated Growth}
\label{matter_dominated_growth}
\subsection{Intuitive Argument}
Consider a flat universe with energy density $\varepsilon(\vec{r},t)$ at a location $\vec{r}$ and time $t$.  We define the density contrast $\delta(\vec{r},t)$ such that
\begin{equation}
  \varepsilon(\vec{r},t) = \bar{\varepsilon}(t)\big[1+\delta(\vec{r},t)\big]
\end{equation}
where $\bar{\varepsilon}(t)$ is the average energy density at time $t$:
\begin{equation}
  \bar{\varepsilon}(t) = \frac{\int\varepsilon(\vec{r},t)d^3r }{\int d^3r}
\end{equation}

Recall for a flat single-component universe, the first Friedmann equation gives
\begin{equation}
  H^2 = \frac{8\pi G}{3c^2}\bar{\varepsilon}
\end{equation}
Within a region of overdensity $\delta$, there will be positive curvature so that
\begin{equation}
  H^2 = \frac{8\pi G}{3c^2}\bar{\varepsilon}(1+\delta) - \frac{\kappa c^2}{a^2 R_0^2}
\end{equation}
Equating the hubble-parameter $H$ at the boundary gives
\begin{equation}
  \delta = \frac{3\kappa c^4}{8\pi G a^2 R_0^2 \bar{\varepsilon}}
\end{equation}
For a matter-only universe, $\bar{\varepsilon} \propto a^{-3}$ so that to first-order,
\begin{equation}
  \delta \propto a
\end{equation}
Thus, we define the \textit{comoving density contrast} 
$\Delta \equiv \delta/a$.  In matter-dominated linear growth, $\Delta$ should 
be approximately constant with redshift.  This makes it an ideal choice as an 
observable in large-scale structure contexts.  
Note that in a more general calculation, this type of analysis gives 
the relation 
\begin{equation}
  \label{G_def}
  \delta(a) \propto G(a)
\end{equation}
where $G(a)$ the linear growth factor of the density field, so that the 
power spectrum at a given epoch is $P_k(z) = G(z)^2 P_k(0)$.

\subsection{Detailed Derivation}
A full General Relativistic treatment of the growth of density fluctuations 
gives \citep[see, e.g.][]{Peacock}
\begin{equation}
  \label{delta_growth}
  \ddot{\delta}_k + 2H\dot{\delta}_k - \frac{3}{2}\Omega_m H^2 \delta_k
\end{equation}
Make the substitution $\delta_k(z) = G(z)\delta_k(z=0)$, and using
\begin{equation} 
  H^2 = \left( \frac{\dot{a}}{a}\right)^2 = H_0^2 \sum_X \Omega_X a^{n_X},
\end{equation}
where the sum is over the components of the universe for an arbitrary
cosmology (with non-evolving dark energy equation of state $w$), 
equation \ref{delta_growth} can 
be written:
\begin{equation}
  a^2 \frac{d^2G}{da^2} + \left(3+\frac{\sum_X n_X \Omega_X a^{n_X}}
  {2\sum_X \Omega_X a^{n_X}} \right)a\frac{dG}{da} - \frac{3}{2}\Omega_m G = 0
\end{equation}
It is easy to check that a matter-dominated solution gives $G(a) \propto a$
as found above.  For a general cosmology, this equation can be numerically 
solved given appropriate boundary conditions.  We follow the convention of
\citet{Takada04} and let $G(a_0)=a_0$ and $dG/da|_{a_0} = 1$ at $a_0=1/1100$, the approximate
scale factor at the formation of the CMB.

\end{document}				% REQUIRED

