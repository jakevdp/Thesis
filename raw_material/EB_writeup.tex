% tutorial.tex  -  a short descriptive example of a LaTeX document
%
% For additional information see  Tim Love's ``Text Processing using LaTeX''
% http://www-h.eng.cam.ac.uk/help/tpl/textprocessing/
%
% You may also post questions to the newsgroup <b> comp.text.tex </b> 

\documentclass[12pt]{article}			% For LaTeX 2e
						% other documentclass options:
						% draft, fleqn, openbib, 12pt

\usepackage{graphicx}	 			% insert PostScript figures
%% \usepackage{setspace}   % controllabel line spacing
%% If an increased spacing different from one-and-a-half or double spacing is
%% required then the spacing environment can be used.  The spacing environment 
%% takes one argument which is the baselinestretch to use,
%%         e.g., \begin{spacing}{2.5}  ...  \end{spacing}


% the following produces 1 inch margins all around with no header or footer
\topmargin	=10.mm		% beyond 25.mm
\oddsidemargin	=0.mm		% beyond 25.mm
\evensidemargin	=0.mm		% beyond 25.mm
\headheight	=0.mm
\headsep	=0.mm
\textheight	=220.mm
\textwidth	=165.mm
					% SOME USEFUL OPTIONS:
% \pagestyle{empty}			% no page numbers
 \parindent  15.mm			% indent paragraph by this much
 \parskip     2.mm			% space between paragraphs
% \mathindent 20.mm			% indent math equations by this much

\newcommand{\MyTabs}{ \hspace*{25.mm} \= \hspace*{25.mm} \= \hspace*{25.mm} \= \hspace*{25.mm} \= \hspace*{25.mm} \= \hspace*{25.mm} \kill }

\graphicspath{{../Figures/}{../data/:}}  % post-script figures here or in /.

					% Helps LaTeX put figures where YOU want
 \renewcommand{\topfraction}{0.9}	% 90% of page top can be a float
 \renewcommand{\bottomfraction}{0.9}	% 90% of page bottom can be a float
 \renewcommand{\textfraction}{0.1}	% only 10% of page must to be text

\alph{footnote}				% make title footnotes alpha-numeric

% --------------------- end of the preamble ---------------------------

\begin{document}			% REQUIRED
\section*{E and B modes}
\subsection{Defining E/B Fourier Coefficients}
Here we'll work from the E/B potential formalism of Crittendon et al. (2001)
toward the definition of E/B Fourier coefficients.
We'll express the observed ellipticity field 
$\epsilon = (\epsilon_1,\epsilon_2)$ in terms of E-mode and B-mode potentials:
\begin{eqnarray}
  \epsilon_1 = \frac{1}{2}(\partial_1^2 - \partial_2^2)\phi_E - \partial_1\partial_2\phi_B \nonumber \\
  \epsilon_2 = \partial_1\partial_2\phi_E + \frac{1}{2}(\partial_1^2 - \partial_2^2)\phi_B 
\end{eqnarray}
Taking the fourier transform of this equation, we find
$\partial_{1,2} \to i\ell_{1,2}$ and can write
\begin{equation}
  \hat\epsilon_1 \pm i\hat\epsilon_2 = 
  -\frac{1}{2}(\ell_1\pm i\ell_2)^2(\hat\phi_E \pm \hat\phi_B)
\end{equation}
Now we'll define the E/B fourier modes:
\begin{eqnarray}
  E_\ell = -\frac{1}{2}|\ell_1 + i\ell_2|^2\hat\phi_E \nonumber\\
  B_\ell = -\frac{1}{2}|\ell_1 + i\ell_2|^2\hat\phi_B 
\end{eqnarray}
Then we can write
\begin{eqnarray}
  \label{EB_fourier}
  E_\ell \pm iB_\ell = (\hat\epsilon_1 \pm i\hat\epsilon_2)e^{\mp 2i\phi_\ell}
\end{eqnarray}
where $\phi_\ell$ is the rotation angle of the angular component, which enters
via the identity 
$\exp(\mp 2i\phi_\ell) = |\ell_1 + i\ell_2|^2/(\ell_1\pm i\ell_2)^2$.

Writing the Fourier transform explicitly, now, we recover the flat-sky
definition of E/B shear modes used in Hikage et al. (2010):
\begin{equation}
  \label{EB_full}
  E_\ell \pm iB_\ell = \int \mathrm{d}^2\theta 
  [\epsilon_1(\theta) \pm i\epsilon_2(\theta) ]
  e^{-i(\ell\cdot\theta \pm 2i\phi_\ell)}
\end{equation}
Note that $E_\ell$ and $B_\ell$ are, in general, complex: the two equations
given by the $\pm$ are necessary to fully characterize them.

\subsection{E/B components of KL modes}
We express our observed ellipticity in terms of the derived KL basis:
\begin{equation}
  \epsilon(\theta) \equiv \epsilon_1(\theta) + i\epsilon_2(\theta) 
  = \sum_n a_n\psi_n(\theta)
\end{equation}
where $\psi_n(\theta)$ are the KL basis for the shear field, and
$a_n = a_{1,n} + ia_{2,n}$ are the complex KL coefficients.  Because
$\psi_n$ are real, the orthogonality relation has no mixing between real
and imaginary parts:
\begin{eqnarray}
  a_{j,n} = \sum_\theta \psi_n(\theta)\epsilon_j(\theta) \nonumber\\
  \epsilon_j(\theta) = \sum_n a_{j,n}\psi_n(\theta)
\end{eqnarray}
with $j\in\{1,2\}$.

Putting this result into equation \ref{EB_fourier} we can write
\begin{equation}
  E_\ell \pm iB_\ell = \sum_n (a_{1,n} \pm ia_{2,n})\hat\psi_n(\ell) e^{\mp 2i\phi_\ell}
\end{equation}
The E and B components can be separated and written
\begin{eqnarray}
  E_\ell = \sum_n\hat\psi_n(\ell)
  \mathrm{Re}\left[ a_n e^{-2i\phi_\ell} \right] \nonumber\\
  B_\ell = \sum_n\hat\psi_n(\ell)
  \mathrm{Im}\left[ a_n e^{-2i\phi_\ell} \right] 
\end{eqnarray}
For a finite field and pixelization, the Fourier transform $\hat\psi_n(\ell)$
is only determined over a limited range in $\ell$, 
bounded by the field window scale for small $\ell$, 
and the pixel window scale for large $\ell$.  
In the intermediate range, the E/B properties of the field 
can be well-understood, and for a given KL basis
are encoded in the real and imaginary
parts of $a_n \exp(-2i\phi_\ell)$.

It is easy to see here why Fourier modes are ``pure'' in the E/B sense.
The $\ell$-space representation of a Fourier mode is 
given by a delta function in $\ell$, so that
a careful choice of $a_n$ can lead to either $E_\ell$ or $B_\ell$ being
identically zero.  KL modes that are very close to Fourier modes (i.e. those
that have little ``leakage'' in $\ell$-space) can come close to this property.

\end{document}				% REQUIRED

