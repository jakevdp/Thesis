% tutorial.tex  -  a short descriptive example of a LaTeX document
%
% For additional information see  Tim Love's ``Text Processing using LaTeX''
% http://www-h.eng.cam.ac.uk/help/tpl/textprocessing/
%
% You may also post questions to the newsgroup <b> comp.text.tex </b> 

\documentclass[12pt,preprint]{aastex}			% For LaTeX 2e
						% other documentclass options:
						% draft, fleqn, openbib, 12pt

\usepackage{graphicx}	 			% insert PostScript figures
\usepackage{amsfonts}
%% \usepackage{setspace}   % controllabel line spacing
%% If an increased spacing different from one-and-a-half or double spacing is
%% required then the spacing environment can be used.  The spacing environment 
%% takes one argument which is the baselinestretch to use,
%%         e.g., \begin{spacing}{2.5}  ...  \end{spacing}


% the following produces 1 inch margins all around with no header or footer
%\topmargin	=10.mm		% beyond 25.mm
%\oddsidemargin	=0.mm		% beyond 25.mm
%\evensidemargin	=0.mm		% beyond 25.mm
%\headheight	=0.mm
%\headsep	=0.mm
%\textheight	=220.mm
%\textwidth	=165.mm
					% SOME USEFUL OPTIONS:
% \pagestyle{empty}			% no page numbers
% \parindent  15.mm			% indent paragraph by this much
% \parskip     2.mm			% space between paragraphs
% \mathindent 20.mm			% indent math equations by this much

\newcommand{\MyTabs}{ \hspace*{25.mm} \= \hspace*{25.mm} \= \hspace*{25.mm} \= \hspace*{25.mm} \= \hspace*{25.mm} \= \hspace*{25.mm} \kill }

\graphicspath{{../Figures/}{../data/:}}  % post-script figures here or in /.

					% Helps LaTeX put figures where YOU want
 \renewcommand{\topfraction}{0.9}	% 90% of page top can be a float
 \renewcommand{\bottomfraction}{0.9}	% 90% of page bottom can be a float
 \renewcommand{\textfraction}{0.1}	% only 10% of page must to be text

\newcommand{\rcom}{\chi}     %comoving distance
\newcommand{\dd}{\mathrm{d}} %differential for integrals

\alph{footnote}				% make title footnotes alpha-numeric

\title{Fourier Transforms: Discrete and Continuous}	% the document title

\author{Jake VanderPlas\\
  \texttt{vanderplas@astro.washington.edu}}

\date{\today}				% your own text, a date, or \today

% --------------------- end of the preamble ---------------------------

\begin{document}			% REQUIRED
\section{Continuous Fourier Transforms}
For an $n$-dimensional function $g(\vec x)$, we can define the Fourier 
Transform\footnote{for mathematica users, this fourier transform
convention corresponds to the option \texttt{FourierParameters->\{1,1\}}.}
\begin{equation}
  \label{ft_continuous}
  \hat{g}(\vec k) = \int \dd^nx g(\vec x)e^{-i\vec x\cdot\vec k}
\end{equation}
The corresponding inverse Fourier Transform is
\begin{equation}
  \label{ift_continuous}
  g(\vec x) = \int \frac{\dd^nk}{(2\pi)^n} \hat{g}(\vec k)
  e^{i\vec k\cdot\vec x}.
\end{equation}
From these, we can see that the n-dimensional Dirac delta 
function can be written
\begin{equation}
  \label{ddelta_form}
  \delta^n_D(\vec x-\vec x^\prime) 
  = \frac{1}{(2\pi)^n}\int \dd^nke^{-i\vec k\cdot(\vec x-\vec x^\prime)}
\end{equation}
such that 
\begin{equation}
  \label{ddelta_def}
  \int \dd^nx f(\vec x)\delta^n_D(\vec x-\vec x^\prime) = f(\vec x^\prime)
\end{equation}
Note that the fourier transform convention in eqns 
\ref{ft_continuous}-\ref{ift_continuous}
is useful in that it leads to a particularly simple form of the 
convolution theorem, without any gratuitous factors of $\sqrt{2\pi}$:
\begin{equation}
  h(\vec x) = \int\dd^nx^\prime 
  f(\vec x^\prime)g(\vec x-\vec x^\prime)
  \ \ \ \ \ \Longleftrightarrow\ \ \ \ \ 
  \hat h(\vec k) = \hat f(\vec k)\hat g(\vec k)
\end{equation}

\section{Discrete Fourier Transform}
For computational purposes, it is more useful to consider the discrete
fourier transform. It is helpful to define some notation to make the 
following expressions
more compact.  First let an element of the $n$-dimensional array
$G_{j_1,j_2\cdots j_n}$ be written $G_\mathbf{j}$, where $\mathbf j$ is
understood to be an $n$ dimensional vector of integers, with 
$0 \le j_i < N_i$.  We'll also define the multiple sum
\begin{equation}
  \sum_{\mathbf{j}} \equiv
  \sum_{j_1=0}^{N_1-1}\sum_{j_2=0}^{N_2-1}\cdots\sum_{j_n=0}^{N_n-1}
\end{equation}
Using this notation, we can write the $n$-dimensional discrete fourier
transform as
\begin{equation}
  \label{ft_discrete}
  \hat{G}_{\mathbf{k}} 
  = \sum_{\mathbf{j}}
  G_{\mathbf{j}} e^{(-2\pi i/N)\mathbf{j}\cdot\mathbf{k}},\ 0\le k_m< N_m
\end{equation}
The corresponding inverse discrete fourier transform is given by
\begin{equation}
  \label{ift_discrete}
  G_{\mathbf{j}}
  = \left[\prod_m\frac{1}{\left(N_m\right)^n}\right]
  \sum_{\mathbf{k}}
  \hat{G}_{\mathbf{k}} e^{(2\pi i/N)\mathbf{j}\cdot\mathbf{k}},\ 0\le j_\ell< N_\ell
\end{equation}
From this, we see that the Kronecker delta function can be written
\begin{equation}
  \delta^n_{\mathbf{j}\mathbf{j^\prime}} 
  = \left[\prod_m\frac{1}{\left(N_m\right)^n}\right]
  \sum_{\mathbf{k}}
  e^{(2\pi i/N)\mathbf{k}\cdot(\mathbf{j}-\mathbf{j^\prime})}
\end{equation}
such that
\begin{equation}
  G_{\mathbf{j}} 
  = \sum_{\mathbf{j^\prime}}
  \delta^n_{\mathbf{j}\mathbf{j^\prime}}
  G_{\mathbf{j^\prime}} 
\end{equation}

The expressions in equations \ref{ft_discrete} and \ref{ift_discrete} can
be quickly computed using the Fast Fourier Transform algorithm.

\section{The Relationship Between Discrete and Continuous Fourier Transforms}
Often it is desirable to approximate a continuous fourier transform given 
a discrete sampling of the function.  Here we will work in one dimension,
though the generalization to multiple dimensions is straightforward.

Consider a continuous function $g(x)$, which is sampled at $N$ equal intervals
$G_j = g(x_j)$ with $x_j \equiv x_0 + j\Delta x$, $0\le j<N$.  Assuming that
$g(x)\approx 0$ outside the range $x_0\le x \le (x_0+N\Delta x)$, we can
approximate the integral in equation \ref{ft_continuous} as
\begin{eqnarray}
  \hat{g}(t) &=& \int \dd x g(x)e^{-ixt}\nonumber\\
  &\approx& \Delta x\sum_{j=0}^{N-1}g(x_j)e^{-ixt}\nonumber\\
  &\approx& \Delta x\sum_{j=0}^{N-1}g(x_j)e^{-it(x_0+j\Delta x)}
\end{eqnarray}
We would like to sample the fourier transform $\hat g(t)$ at $N$ equally
spaced intervals in $t$.  To this end, let $t_k = t_0+k\Delta t$ such that,
\begin{equation}
  \hat{g}(t_k) 
  \approx \Delta x\sum_{j=0}^{N-1}g(x_j)
  e^{-i(t_0+k\Delta t)(x_0+j\Delta x)}
\end{equation}
Now to make this look like equation \ref{ft_discrete}, we let 
$\Delta t = 2\pi/(N\Delta x)$ and rearrange to find
\begin{equation}
  \frac{1}{\Delta x}\hat{g}(t_k) e^{i (t_k-t_0) x_0}
  \approx \sum_{j=0}^{N-1}g(x_j)e^{-it_0x_j}e^{-2\pi ijk/N}
\end{equation}
We see that this matches equation \ref{ft_discrete} with
\begin{eqnarray}
  G_j &\equiv& g(x_j)e^{-it_0x_j}\nonumber\\
  \hat{G}_k &\equiv& \frac{e^{-ix_0t_0}}{\Delta x}\hat{g}(t_k) e^{i x_0t_k}
\end{eqnarray}
Thus the continuous fourier transform can be approximated using a discrete
sampling and an FFT, by using the appropriate exponential weighting of
the sampled functions.  The extension of this to multiple dimensions 
follows by applying the correct transformation along each dimension.

Note that when using the FFT and IFFT routines in, e.g. fftpack, the choice
of $t_0$ is not necessarily free (though by exploiting the periodic boundary
conditions implicit in an FFT, any range in $t$ can be calculated).  
The fiducial $t$ range
is symmetric about $t=0$, so that $t_0 = (-N/2)\Delta t = -\pi/\Delta x$.

\section{A More Correct Form}
In many cases, the discrete sampling of the previous section is not simply
the value of $g(x)$ at each point $x_j$, but an average over a range 
$x_j-\Delta x/2 < x < x_j+\Delta x/2$.  To be correct in this situation, we
need to recognize that our sampled values at position $x_j$ are an estimator
of a different quantity, namely
\begin{equation}
  g_W(x_j) = \int \dd x g(x)W_j(x_j-x)
\end{equation}
with 
\begin{equation}
  W_j(x) = \left\{
\begin{array}{ll}
  1/\Delta x,& |x-x_j| < \Delta x/2 \\
  0,& \mathrm{otherwise}
\end{array}\right.
\end{equation}
By the convolution theorem, we can write
\begin{equation}
  \hat{g}(t) = \hat{g}_W(t) / \hat{W}_j(t)
\end{equation}
Where the fourier transform of the window function is given by
\begin{equation}
  \label{window_ft}
  \hat{W}_j(t) = \frac{\sin(t\Delta x/2)}{t\Delta x/2}
\end{equation}
Thus we can compute $\hat g(t)$ correctly by applying the FFT approach from the
previous section to compute $\hat g_W(t)$ from the sampled values, 
then dividing by the Fourier
transorm of the pixel window in equation \ref{window_ft}.

At first glance, it seems that there might be a problem, because $1/W_j(t)$
diverges for $t\Delta x = 2\pi n$, with $n$ a non-zero integer.  
It turns out that
the limits of the fft save us here: we showed above that 
$t_k = (k-N/2)\Delta t = \pi(2k/N-1)/\Delta x$, with $0\le k<N$, 
so that that we are limited to the domain $-\pi \le t\Delta x < \pi$,
which lies safely within the finite range.  The result does not change 
the smaller wave numbers, but weights the larger wave-numbers by 
an increasing factor of up to $\sim 1.5$.

\end{document}		