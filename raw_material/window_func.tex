% tutorial.tex  -  a short descriptive example of a LaTeX document
%
% For additional information see  Tim Love's ``Text Processing using LaTeX''
% http://www-h.eng.cam.ac.uk/help/tpl/textprocessing/
%
% You may also post questions to the newsgroup <b> comp.text.tex </b> 

\documentclass[12pt,preprint]{aastex}			% For LaTeX 2e
						% other documentclass options:
						% draft, fleqn, openbib, 12pt
 
\usepackage{graphicx}	 			% insert PostScript figures
\usepackage{amsfonts}
\usepackage[stable]{footmisc} %allows footnotes on section titles

%% \usepackage{setspace}   % controllabel line spacing
%% If an increased spacing different from one-and-a-half or double spacing is
%% required then the spacing environment can be used.  The spacing environment 
%% takes one argument which is the baselinestretch to use,
%%         e.g., \begin{spacing}{2.5}  ...  \end{spacing}


% the following produces 1 inch margins all around with no header or footer
%\topmargin	=10.mm		% beyond 25.mm
%\oddsidemargin	=0.mm		% beyond 25.mm
%\evensidemargin	=0.mm		% beyond 25.mm
%\headheight	=0.mm
%\headsep	=0.mm
%\textheight	=220.mm
%\textwidth	=165.mm
					% SOME USEFUL OPTIONS:
% \pagestyle{empty}			% no page numbers
% \parindent  15.mm			% indent paragraph by this much
% \parskip     2.mm			% space between paragraphs
% \mathindent 20.mm			% indent math equations by this much

%\newcommand{\MyTabs}{ \hspace*{25.mm} \= \hspace*{25.mm} \= \hspace*{25.mm} \= \hspace*{25.mm} \= \hspace*{25.mm} \= \hspace*{25.mm} \kill }

%\graphicspath{{../Figures/}{../data/:}}  % post-script figures here or in /.

					% Helps LaTeX put figures where YOU want
 \renewcommand{\topfraction}{0.9}	% 90% of page top can be a float
 \renewcommand{\bottomfraction}{0.9}	% 90% of page bottom can be a float
 \renewcommand{\textfraction}{0.1}	% only 10% of page must to be text

\alph{footnote}				% make title footnotes alpha-numeric

%\title{Cosmological Window Functions}	% the document title

%\author{Jake VanderPlas}

\newcommand{\rcom}{\chi}
\newcommand{\dd}{\mathrm{d}}

%\date{\today}				% your own text, a date, or \today

% --------------------- end of the preamble ---------------------------

\begin{document}			% REQUIRED
\bibliographystyle{apj}

\section{On Random Fields, Correlation Functions, and Power Spectra}
Consider a field $g(\vec x)$ in $n$ dimensions. We'll enforce a 
few restrictions on this field to make it easier to work with.
Note that $\langle\cdot\rangle$ denotes a volume-average:
\begin{enumerate}
  \item vanishing: $\langle g(\vec x)\rangle=0$ 
    for all $\vec x$.
  \item homogeneous: $g(\vec x + \vec y)$ is statistically equivalent to
    $g(\vec x)$ for all $\vec x$ and $\vec y$.
  \item isotropic: $g(\mathbf{R}\vec x)$ is statistically equivalent to
    $g(\vec x)$ for all $\vec x$ and any unitary rotation matrix $\mathbf{R}$.
\end{enumerate}
These conditions become very useful when we study the 
(auto) correlation function, defined as
\begin{equation}
  \label{Cgg}
  C_{gg}(\vec r) \equiv \left\langle g(\vec x) g^*(\vec x+\vec{r})\right\rangle
\end{equation}
which for a homogeneous and isotropic field depends only on the 
distance $r = |\vec r|$.  It becomes useful to decompose $g$ into
orthogonal Fourier components:\footnote{
  Note that the fourier transform convention in eqns \ref{ft}-\ref{ift}
  is useful in that it leads to a particularly simple form of the 
  convolution theorem, without any gratuitous factors of $\sqrt{2\pi}$:
  \begin{displaymath}
    h(\vec x) = \int\dd^nx^\prime 
    f(\vec x^\prime)g(\vec x-\vec x^\prime)
    \ \ \ \ \ \Longleftrightarrow\ \ \ \ \ 
    \hat h(\vec k) = \hat f(\vec k)\hat g(\vec k)
  \end{displaymath}
}
\begin{eqnarray}
  \label{ft}
  g(\vec x) = \int \frac{\dd^nk}{(2\pi)^n} \hat{g}(\vec k)
  e^{-i\vec k\cdot\vec x}\\
  \label{ift}
  \hat{g}(\vec k) = \int \dd^nx g(\vec x)e^{i\vec x\cdot\vec k}
\end{eqnarray}

From these, we can see that the n-dimensional dirac delta 
function can be written
\begin{equation}
  \label{ddelta_form}
  \delta^n_D(\vec x-\vec x^\prime) = \frac{1}{(2\pi)^n}\int \dd^nke^{-i\vec k\cdot(\vec x-\vec x^\prime)}
\end{equation}
such that 
\begin{equation}
  \label{ddelta_def}
  \int \dd^nx f(\vec x)\delta^n_D(\vec x-\vec x^\prime) = f(\vec x^\prime)
\end{equation}

We now define the Power Spectrum of $g$ to be the Fourier transform of the 
auto-correlation function, which, due to isotropy, depends only on the 
magnitude of $\vec k$:
\begin{eqnarray}
  \label{pspec}
  P_g(k) = \int \dd^nx e^{-i\vec x\cdot\vec k}C_{gg}(x)\nonumber\\
  C_{gg}(x) = \frac{1}{(2\pi)^n}\int\dd^nke^{i\vec x\cdot\vec k}P_g(k)
\end{eqnarray}
A bit of math shows that the Power Spectrum is proportional to the 
Fourier-space correlation function:
\begin{equation}
  \label{pspec_corr}
  \hat C_{gg}(\vec k-\vec k^\prime) 
  \equiv \left\langle\hat g(\vec k)\hat g^*(\vec k^\prime) \right\rangle 
  =  (2\pi)^n \delta^n_D(\vec k -\vec k^\prime) P_g(|\vec k|).
\end{equation}
Along with isotropy and homogeneity, this result implies
\begin{equation}
  P_g(k) \propto \left\langle | \hat g(k)|^2 \right\rangle.
\end{equation}
The proportionality constant is finite only for a discrete Fourier series 
(i.e. a finite averaging volume).

\subsection{Smoothing of Gaussian Fields}
\label{smoothing}
When measuring a realization of gaussian field, we often make the measurement
within a region defined by a window function $W(\vec x/R)$.
By convention, we express window functions in terms of $\vec x/R$,
where $R$ is a characteristic length scale of the window.
This window may reflect a sharp boundary in space (e.g. a spherical tophat 
function) or perhaps an observation efficiency in space (e.g. a 3D gaussian).
In either case, our observed overdensity is given by
\begin{equation}
  g_W(\vec x) = \int \dd^n x^\prime 
  W\left(\frac{\vec{x}^\prime-\vec{x}}{R}\right) g(\vec{x}^\prime)
\end{equation}
where $W(\vec x/R)$ is normalized such that
\begin{equation}
  \label{W_normalization}
  \int \dd^nx W(\vec{x}/R) = 1.
\end{equation}
This can be simplified if we can define the Fourier transform pair 
of a window function in the convention
of equations \ref{ft} and \ref{ift} (cf. Liddle \& Lyth 2000):
\begin{eqnarray}
  \label{W-transform}
  \widetilde{W}(\vec{k}R) = \int W(\vec{x}/R) e^{i\vec{k}\cdot\vec{x}}\dd^nx 
  \nonumber\\
  W(\vec{x}/R) = \frac{1}{(2\pi)^n}\int 
  \widetilde{W}(\vec{k}R) e^{-i\vec{k}\cdot\vec{x}}\dd^nk 
\end{eqnarray}
This definition is convenient because, when combined with equation \ref{ft},
some straightforward algebra leads to the convolution theorem:
\begin{equation}
  \hat g_W(\vec k) = \widetilde W(\vec k R)\hat g(\vec k)
\end{equation}
It is also useful to calculate the cross-correlation between two windows,
\begin{equation}
  \label{W_correlation}
  \langle g_{W_1}(\vec x_1)g^*_{W_2}(\vec x_2)\rangle = 
  \int \dd^n x 
  W_1\left(\frac{\vec{x}-\vec{x}_1}{R}\right) 
  \int \dd^n x^\prime 
  W_2\left(\frac{\vec{x}^\prime-\vec{x}_2}{R}\right) 
  \left\langle g(\vec{x})g^*(\vec{x}^\prime)  \right\rangle
\end{equation}
Using equations \ref{pspec} and \ref{W-transform}, we can re-express
equation \ref{W_correlation} as a single integral over the
wave number:
\begin{equation}
  \label{W_cov_simp}
  \langle g_{W_1}(\vec x_1)g^*_{W_2}(\vec x_2)\rangle = 
  \frac{1}{(2\pi)^n} \int \dd^nk
  P_g(k)\widetilde{W}_1(\vec{k}R)\widetilde{W}_2(\vec{k}R)
  e^{i\vec k\cdot(\vec x_1-\vec x_2)}
\end{equation}
Appendix \ref{Wtransforms} lists a few common window functions and
their fourier transforms.

\section{Cosmological Mass Power Spectrum}
In studies of the cosmological distribution of matter, we are interested
in the comoving matter density $\rho(\vec{x})$, which defines the mass 
density at every comoving point $\vec{x}$ in the universe.  In order
to take advantage of the preceding formalism, we can subtract the mean
cosmological density $\bar{\rho}(\vec{x})$, and definine a dimensionless
density contrast $\delta(\vec{x})$, such that
\begin{equation}
  \delta(\vec{x}) = \frac{\rho(\vec{x}) - \bar{\rho}(\vec{x})}{\bar{\rho}(\vec{x})}.
\end{equation}
By the assumptions of the Cosmological Principle, for small
deviations $\delta(\vec x)$ is an isotropic, homogeneous random field.  
We can better understand the distribution of $\delta(\vec x)$ by looking
at the mean square deviation
\begin{eqnarray}
  \label{Cdd0}
  \left\langle|\delta(\vec x)|^2\right\rangle 
  &=& C_{\delta\delta}(0)\nonumber\\
  &=& \frac{1}{(2\pi)^3}\int \dd^3 kP_\delta(k)\nonumber\\
  &=& \frac{1}{2\pi^2}\int k^2\dd k P_\delta(k)
\end{eqnarray}
The power spectrum of density contrast given by equation \ref{pspec}
can be an inconvenient quantity to work with, because it has dimensions
of volume.  We can take the lead from equation \ref{Cdd0} and define
a dimensionless form of the power spectrum
\begin{equation}
  \label{P_conversion}
 \Delta^2(k) = \frac{k^3}{2\pi^2} P_\delta(k)
\end{equation}
This is constructed so that equation \ref{Cdd0} can be written in a
simple form:
\begin{equation}
  \left\langle|\delta(\vec x)|^2\right\rangle 
  = \int_0^\infty \Delta^2(k)\dd(\ln k)
\end{equation}
This convention is due to Peacock (1999).
For mathematical convenience, we'll continue to work with 
the $P_\delta(k)$ convention,
with the understanding that we can switch back and forth any time using
equation \ref{P_conversion}. 

\subsection{Power Spectrum Normalization}
In practice, the functional form of the power spectrum is determined only
up to a proportionality constant, such that
\begin{equation}
  P_\delta(k) = P_0 P^\prime_\delta(k)
\end{equation}
where $P^\prime_\delta(k)$ is the unnormalized form.
For historical reasons, the normalization constant $P_0$ is commonly 
expressed in terms of the parameter $\sigma_8$, 
which is defined as the mean density 
fluctuation within a sphere of radius 8 Mpc.  To compute this, we use a 
top-hat window function:
\begin{equation}
\label{top-hat}
  W_T(\vec{x}/R) = \left\{
    \begin{array}{ll}
      1, & |\vec x|/R \le 1 \\
      0, & |\vec x|/R > 1
    \end{array}
    \right.
\end{equation} 
The density fluctuation within this window is found using equation \ref{W_cov_simp}:
\begin{equation}
  \label{powerspec-sigma-def}
  \sigma_R^2 = \frac{1}{(2\pi)^3}\int \dd^3\vec{k} 
  P_\delta(k) [\widetilde{W}_T(\vec{k}R)]^2
\end{equation}
where the window function is assumed to be shallow enough that there is no
cosmological evolution of the signal.

For the top-hat window function of equation \ref{top-hat}, with 
$k = |\vec{k}|$,
the fourier transform of equation \ref{top-hat} (cf. eqn. \ref{W-transform}) is
\begin{equation}
  \label{top-hat-f}
  \widetilde{W}_T(kR) = \frac{3}{(kR)^3}\left[\sin(kR) - kR\cos(kR) \right].
\end{equation}
$\sigma_8$ can be calculated using equation \ref{powerspec-sigma-def}
and \ref{top-hat-f} with $R=8$Mpc for a given $P_\delta(k)$. 
The WMAP 5-year measurement gives $\sigma_8 = 0.812 \pm 0.026$ 
\citep{Hinshaw09}.  Using this value, the correct normalization 
can be computed for any functional form of the power spectrum.

\subsection{Window functions and Measurement Covariance}
In a 3D lensing analysis, we are searching for a signal within a series
of windows defined as 
\begin{equation}
  W_{ij}(\vec{x}) = W_{ij}(\rcom,\vec{\theta}) 
  = q_i(\rcom) \cdot F_j(\vec\theta)
\end{equation}
where $\rcom$ is the radial comoving distance, and $\vec{\theta}$ is the
angular position on the sky.  To convert between angle on the sky and 
projected comoving separation, we multiply by the transverse comoving 
distance (eqn. 16 in Hogg 1999), given by
\begin{equation}
f_\kappa(\rcom) = \left\{
\begin{array}{ll}
  \kappa^{-1/2} \sin (\kappa^{1/2}\rcom) & (\kappa > 0) \\
  \rcom  & (\kappa = 0)\\
  (-\kappa)^{-1/2} \sinh [(-\kappa)^{1/2}\rcom] & (\kappa < 0)
\end{array}\right.
\end{equation}
Our observed overdensity in window $W_{ij}$ is given by
\begin{eqnarray}
  \delta_{ij} 
  & = & \int \dd^3x W_{ij}(\vec x)\delta(\vec x)\nonumber\\
  & = & \int \dd\rcom \int \dd^2\theta \left[ f_\kappa(\rcom)\right]^2F_j(\vec{\theta})   q_i(\rcom) \delta\left(f_\kappa(\rcom)\vec{\theta},\rcom\right)
\end{eqnarray}
$F_j$ is the function describing the shape of the $j^{th}$ pixel,
while $q_i$ is the function describing the $i^{th}$ redshift bin.
These window functions should be normalized as in equation \ref{W_normalization},
such that
\begin{equation}
  \label{q_normalization}
  \int \dd\rcom [f_\kappa(w)]^2 q_i(\rcom) = 1
\end{equation}
and
\begin{equation}
  \int \dd^2\theta F_j(\vec{\theta}) = 1
\end{equation}
We are usually concerned with the covariance matrix of the signal, given by
\begin{eqnarray}
  \label{Sdd}
  \left[S_{\delta\delta}\right]_{nm}
  & = & \left\langle \delta_{i_nj_n}\delta^*_{i_mj_m}\right\rangle \nonumber\\
  & = & \int \dd^2\theta F_{j_n}[\vec{\theta}] 
  \int \dd^2\theta^\prime F_{j_m}[\vec{\theta}^\prime] \nonumber\\
  & & \times
  \int \dd\rcom \left[f_\kappa(\rcom)\right]^2 q_{i_n}[\rcom]
  \int \dd\rcom^\prime \left[f_\kappa(\rcom^\prime)\right]^2 
  q_{i_m}[\rcom^\prime]\nonumber\\
  & &\times
  \left\langle
  \delta\left(f_\kappa(\rcom)\vec{\theta},
  \rcom\right)
  \delta^*\left(f_\kappa(\rcom^\prime)\vec{\theta^\prime},
  \rcom^\prime\right)
  \right\rangle
\end{eqnarray}
This can be simplified using the Limber approximation.

\subsection{The Limber Approximation\footnote{The following logic comes from \citet{Bartelmann01}.}}
Consider a projection of the density field along a certain radial direction
\begin{equation}
  g_i(\vec{\theta}) = \int \dd\rcom p_i(\rcom)\delta\left(f_\kappa(\rcom)\vec{\theta},\rcom\right).
\end{equation}
The cross correlation is
\begin{eqnarray}
  \label{Sgg}
  S_{g_ig_j}(\vec\theta-\vec\theta^\prime) 
  &=& \left\langle g_i(\vec\theta)g^*_j(\vec\theta^\prime)\right\rangle \nonumber\\
  &=& \int \dd\rcom p_i(\rcom)
  \int \dd\rcom^\prime p_j(\rcom^\prime)
  \left\langle\delta[f_\kappa(\rcom)\vec{\theta},\rcom]
  \delta^*[f_\kappa(\rcom^\prime)\vec{\theta}^\prime,\rcom^\prime]
  \right\rangle
\end{eqnarray}
Let's express $\delta(\vec x)$ in terms of the Fourier integral, equation \ref{ft}.  This gives
\begin{eqnarray}
  S_{g_ig_j}(\vec\theta-\vec\theta^\prime) 
  &=& \int \dd\rcom p_i(\rcom)
  \int \dd\rcom^\prime p_j(\rcom^\prime)
  \int \frac{\dd^3k}{(2\pi)^3}\int \frac{\dd^3k^\prime}{(2\pi)^3}
  \left\langle\hat\delta(\vec k,\rcom)
  \hat\delta^*(\vec k^\prime, \rcom^\prime)
  \right\rangle\nonumber\\
  &&\times
  \exp\left[-i f_\kappa(\rcom)\vec{\theta}\cdot \vec k_\perp -i k_\parallel w\right]
  \exp\left[i f_\kappa(\rcom^\prime)\vec{\theta}^\prime \cdot \vec k^\prime_\perp  + i k^\prime_\parallel w^\prime\right].
\end{eqnarray}
Here $\vec k_\perp$ is the 2-dimensional projection of $\vec k$ perpendicular to 
the line of sight, and $k_\parallel$ is the projection of $\vec k$ along 
the line of sight.
The second argument of $\hat\delta(\vec k,\rcom)$ parametrizes evolution with
time via $|c\cdot dt| = a\cdot d\rcom$.

Because the power spectrum $P_\delta(k)$ decreases linearly with $k$ as 
$k\to0$, there must be a coherence scale $L_c$ such that the
correlation is near zero for 
$|\rcom-\rcom^\prime| \equiv \Delta\rcom > L_c$.  
The first part of the Limber approximation is to assume that $S_{gg}$ vanishes
at these distances.
Next we make the assumption that $p_i(\rcom)$ and 
$p_j(\rcom^\prime)$ do not vary appreciably over the small range where
$S_{g_ig_j}$ is nonvanishing, and that this range is small enough that 
$f_\kappa(\rcom) \approx f_\kappa(\rcom^\prime)$. 
This allows us to rewrite the above expression in a simpler way:
\begin{eqnarray}
  S_{g_ig_j}(\vec\theta-\vec\theta^\prime) 
  &=& \int \dd\rcom p_i(\rcom)p_j(\rcom)
  \int \frac{\dd^3k}{(2\pi)^3}\int \frac{\dd^3k^\prime}{(2\pi)^3} 
  \left\langle\hat\delta(\vec k,\rcom)
  \hat\delta^*(\vec k^\prime, \rcom)
  \right\rangle\nonumber\\
  &&\times
  \exp\left[-i f_\kappa(\rcom) 
    \left(\vec{\theta} \cdot \vec k_\perp 
    - \vec{\theta}^\prime \cdot k^\prime_\perp \right) 
    -i \rcom k_\parallel \right]
  \int \dd\rcom^\prime\exp\left(i\rcom k^\prime_\parallel\right)
\end{eqnarray}
The integral over $\rcom^\prime$ is 
simply $2\pi\delta_D(k^\prime_\parallel)$
via equation \ref{ddelta_form}, and the Fourier space correlation function
is proportional to $P_\delta(k)\delta^3_D(\vec k-\vec k^\prime)$ 
via equation \ref{pspec_corr}:
\begin{eqnarray}
  S_{g_ig_j}(\vec\theta-\vec\theta^\prime) 
  &=& \int \dd\rcom p_i(\rcom)p_j(\rcom)
  \int \frac{\dd^3k}{(2\pi)^2}\int \dd^3k^\prime
  \delta_D^3(\vec k-\vec k^\prime)\delta_D(k_\parallel^\prime)P_\delta(k)\nonumber\\
  &&\times
  \exp\left[-i f_\kappa(\rcom) 
    \left(\vec{\theta} \cdot \vec k_\perp 
    - \vec{\theta}^\prime \cdot k^\prime_\perp \right) 
    -i \rcom k_\parallel \right]
\end{eqnarray}
Carrying out the integrals over the two delta functions we see
\begin{eqnarray}
  \label{Limber}
  S_{g_ig_j}(\vec\theta_{ij}) 
  &=& \int \dd\rcom p_i(\rcom)p_j(\rcom)
  \int \frac{\dd^2k_\perp}{(2\pi)^2}
  P_\delta(k_\perp,\rcom)
  \exp\left[-i f_\kappa(\rcom) 
    \vec \theta_{ij} \cdot \vec k_\perp 
    \right]\\
  \label{Limber2}
  &=& \int \dd\rcom p_i(\rcom)p_j(\rcom)
  \int \frac{k\dd k}{2\pi}P_\delta(k,\rcom)
  J_0\left[f_\kappa(\rcom)\theta k\right]
\end{eqnarray}
where we have defined $\vec \theta_{ij} = \vec{\theta} -\vec{\theta}^\prime$, 
and $J_0(x)$ is a Bessel function of the first kind, which comes from 
the angular integral via
\begin{equation}
  \label{bessel_j}
  J_n(x) = \frac{1}{2\pi}\int_{-\pi}^{\pi} e^{-i(n\tau-x sin\tau)}\dd\tau
\end{equation}
We see that all $k_\parallel$ terms have vanished, which leads to the main
result of the Limber approximation: there is no correlation between the 
density contrast in windows which do not overlap 
in redshift. This is quickly seen from the leading integral in equation 
\ref{Limber}.  If the windows $p_i(\rcom)$ and $p_j(\rcom)$ do 
not overlap, then the expression integrates to zero.  

Now that we have a simple expression for the 2-dimensional projected 
correlation functions, we can use equation \ref{pspec} to define the 
2-dimensional cross power spectrum of the measured covariance,
\begin{equation} 
  \label{cross_pspec}
  P_{g_ig_j}(\ell) = \int \dd^2\theta e^{i\vec\theta\cdot\vec\ell}S_{g_ig_j}(\vec\theta).
\end{equation}
Combining equations \ref{Limber} and \ref{cross_pspec}, we have
\begin{eqnarray}
  \label{cross_pspec2}
  P_{g_ig_j}(\ell) 
  &=& \int \dd^2\theta
  \int \dd\rcom p_i(\rcom)p_j(\rcom)
  \int \frac{\dd^2k_\perp}{(2\pi)^2}
  P_\delta(k_\perp,\rcom)
  \exp\left[i\vec\theta\cdot(\vec\ell- f_\kappa(\rcom) 
    \vec k_\perp)\right]\nonumber\\
  &=&\int \dd\rcom p_i(\rcom)p_j(\rcom)\int \dd^2k_\perp
  P_\delta(k_\perp,\rcom)\delta_D^2(\vec\ell- f_\kappa(\rcom)\vec k_\perp)
  \nonumber\\
  &=& \int \dd\rcom\frac{p_i(\rcom)p_j(\rcom)}{[f_\kappa(\rcom)]^2}P_\delta\left(\frac{\ell}{f_\kappa(\rcom)},\rcom\right).
\end{eqnarray}

\subsection{Applying the Limber Approximation}
Examining equation \ref{Sdd}, we see that the integrals over $\rcom$ and 
$\rcom^\prime$ appear in equation \ref{Sgg}, with 
$p_i(\rcom)\to q_{i_n}(\rcom)[f_\kappa(\rcom)]^2$ and 
$p_j(\rcom^\prime)\to 
q_{i_m}(\rcom^\prime)[f_\kappa(\rcom^\prime)]^2$.
Thus we can rewrite equation \ref{Sdd} using the Limber approximation:
\begin{equation}
  \label{Sdd_Limber}
  \left[S_{\delta\delta}\right]_{nm}
  = \int \dd^2\theta F_{j_n}[\vec{\theta}] 
  \int \dd^2\theta^\prime F_{j_m}[\vec{\theta}^\prime] 
  S_{nm}(\vec\theta -\vec\theta^\prime)
\end{equation}
Now using the procedure from section \ref{smoothing}, 
we can write this approximation as an integral in Fourier 
space over the cross power spectrum given by equation
\ref{cross_pspec2}.  We'll consider the simple case where 
each pixel has the same angular shape described by $F(\vec\theta)$,
so that $F_{j_n}(\vec\theta) = F(\vec\theta-\vec\theta_n)$ and
$F_{j_m}(\vec\theta) = F(\vec\theta-\vec\theta_m)$.  Defining
$\vec\theta_{nm} = |\vec\theta_n - \vec\theta_m|$,
\begin{eqnarray}
  \label{Sdd_final}
  \left[S_{\delta\delta}\right]_{nm} &=& 
  \frac{1}{(2\pi)^2}
  \int \dd^2\ell\, \widetilde{F}_{j_n}(\vec\ell)
  \widetilde{F}_{j_m}(\vec\ell)P_{nm}(\ell)
  e^{i\vec\ell\cdot\vec\theta_{nm}}\nonumber\\
  &=&\frac{1}{2\pi}
  \int \ell\dd\ell\, \widetilde{F}_{j_n}(\ell)
  \widetilde{F}_{j_m}(\ell)P_{nm}(\ell)J_0\left(\ell\theta_{nm}\right)
\end{eqnarray}
where the second line holds if the window functions are circularly symmetric, 
with the Bessel function given by equation \ref{bessel_j}.
The projected power spectrum $P_{nm}(\vec\ell)$ is given by 
equation \ref{cross_pspec2} 
with appropriate substitution for $p_{i,j}$:
\begin{equation}
  P_{nm}(\ell) = \int \dd\rcom[f_\kappa(\rcom)]^2 q_{i_n}(\rcom)q_{i_m}(\rcom)P_\delta\left(\frac{\ell}{f_\kappa(\rcom)},\rcom\right).
\end{equation}
Note the factor of $[f_\kappa(\rcom)]^2$ in the numerator, which has 
its root in the spherical coordinate differential 
$d^3x \to r^2dr \cdot d\Omega$.

Our main application of this formalism will involve discrete 
non-overlapping redshift bins with uniform weighting.  That is,
\begin{equation}
  q_i(\rcom) = \left\{
  \begin{array}{ll}
    A_i, & \rcom^{(i)}_{\mathrm{min}} 
    < \rcom < \rcom^{(i)}_{\mathrm{max}}\\
    0, & \mathrm{otherwise}
  \end{array}
  \right.
\end{equation}
with the normalization constant computed via the condition in equation 
\ref{q_normalization}:
\begin{equation}
  A_i = \left[\int_{\rcom^{(i)}_{\mathrm{min}}}^{\rcom^{(i)}_{\mathrm{max}}}
  [f_\kappa(\rcom)]^2 \dd\rcom\right]^{-1}.
\end{equation}
To summarize, the correlation between two windows $n$ 
and $m$ becomes, with $i_{n,m}$
indexing the redshift window, and $n,m$ indexing the angular window,
\begin{eqnarray}
  [S_{\delta\delta}]_{nm} 
  &=& \delta^K_{i_ni_m}\omega(|\theta_{nm}|)\nonumber\\
  \omega(\theta)
  &=&\frac{1}{2\pi}\int_0^\infty\ell\dd\ell\, 
  |\widetilde{F}(\ell)|^2
  P_{nm}(\ell)J_0(\ell\theta)\nonumber\\
  P_{nm}(\ell) &=& 
  \frac{1}{A^2}
  \int_{\rcom^{(i)}_\mathrm{min}}^{\rcom^{(i)}_\mathrm{max}} 
  \dd\rcom \left[f_\kappa(\rcom)\right]^2
  P_\delta\left(\frac{\ell}{f_\kappa(\rcom)},\rcom\right).\nonumber\\
  A &=& \int_{\rcom^{(i)}_\mathrm{min}}^{\rcom^{(i)}_\mathrm{max}} 
  \dd\rcom \left[f_\kappa(\rcom)\right]^2
\end{eqnarray} 
where $\delta^K_{ij}$ is the Kronecker delta. This result should be 
compared  to equations 39-41 of \citet{Simon09}.  The only difference is
that we have correctly accounted for the normalization of the redshift
bin.  Note that if we assume $f_\kappa(\rcom)$ is constant across
each redshift bin, the two formulations are equivalent.

For our analysis, we will use angular pixels with radius $\theta_s$, 
so that the fourier transform of the window functionis given by equation 
\ref{tophat2D} and the equation giving the signal covariance becomes
\begin{equation}
  \left[S_{\delta\delta}\right]_{nm}
  = \frac{2}{\pi\theta_s^2} 
  \int \dd\rcom \frac{q_{i_n}[\rcom]q_{i_m}[\rcom]}
  {f_\kappa(\rcom)^2} 
  \int  \frac{\dd\ell}{\ell}
  P_\delta\left(\frac{\ell}{f_\kappa(\rcom)},\rcom\right)
  [J_1(\theta_s\ell)]^2 J_0(\theta_{nm}\ell)
\end{equation}

\bibliography{window_func}
\newpage
\appendix
\section{Window Functions and their Fourier Transforms}
\label{Wtransforms}
Here we list a few common window functions and their Fourier transforms
\subsection{Gaussian Window Functions}
The $n$-dimenisional gaussian window function is defined as
\begin{equation}
  W(\vec x/R) = \frac{1}{(2\pi R^2)^{n/2}}
  \exp\left(\frac{-|\vec x|^2}{2R^2}\right)
\end{equation}
The fourier transform is straightforward because the dimensions decouple
and we're left with $n$ 1 dimensional gaussian integrals.  The resulting
window function is
\begin{equation}
  \widetilde{W}(\vec k R) 
  = \exp\left(\frac{-|\vec k|^2R^2}{2}\right)
\end{equation}
which itself is a gaussian.
\subsection{Top-hat Window Functions}
The $n$-dimensional tophat windown function is given by
\begin{equation}
W(\vec x/R) = A_n
\times\left\{
\begin{array}{ll}
  1, & |\vec x|\le R\\
  0, & |\vec x|>R
\end{array}
\right.
\end{equation}
with
\begin{equation}
  A_n = \frac{\Gamma(n/2+1)}{(R\sqrt\pi)^n}
\end{equation}
where $\Gamma(y)$ is the gamma function.  The normalization is simply the
inverse of the volume of an $n$-sphere of radius $R$.   
For $n=2$ and $n=3$, the normalizations are the familiar 
$1/(\pi R^2)$ and $3/(4\pi R^3)$, respectively.
For the top-hat window function, there is no
simple expression for the fourier transform for arbitrary $n$. 
Here we compute three special cases:
\begin{description}
  \item{$n=1$}:
    \begin{eqnarray}
      \label{tophat1D}
      A_1 &=& (2R)^{-1}\nonumber\\
      \widetilde W(\vec k R) &=& A_1\int_{-R}^{R}\dd r e^{ikr} \nonumber\\
      &=&\frac{\sin(kR)}{kR}
    \end{eqnarray}
  \item{$n=2$}:
    \begin{eqnarray}
      \label{tophat2D}
      A_2 &=& (\pi R^2)^{-1}\nonumber\\
      \widetilde W(\vec k R) &=& A_2\int_0^R r\dd r\int_0^{2\pi} \exp[ikr\cos\phi]\dd\phi\nonumber\\
      &=& \frac{2}{R^2}\int_0^R r J_0(kr)\dd r \nonumber\\
      &=& \frac{2J_1(kR)}{kR}
    \end{eqnarray}
  \item{$n=3$}: 
    \begin{eqnarray}
      \label{tophat3D}
      A_3 &=& (4\pi R^3/3)^{-1}\nonumber\\
      \widetilde W(\vec k R) &=& A_3\int_0^Rr^2\dd r\int_0^\pi \sin(\theta)\dd\theta\int_0^{2\pi}\exp[ikr\cos\phi]\dd\phi\nonumber\\
      &=& \frac{3}{R^3}\int_0^R r^2 J_0(kr)\dd r\nonumber\\
      &=& _1F_2(\frac{1}{2};\ 1,\frac{5}{2};\ \frac{-k^2R^2}{4})
    \end{eqnarray}
\end{description}
where, $J_n(x)$ are Bessel functions of the first kind (see eqn \ref{bessel_j})
and the last line is a generalized hypergeometric function, 
$_nF_m(a_0\cdots a_n;b_0\cdots b_m;x)$.  
\end{document}				% REQUIRED
  
