

Cluster searches in the x-ray surveys are a promising method 
to address these deficiencies.
The ionized intracluster medium (ICM) in high-mass clusters
emits at x-ray wavelengths via bremsstrahlung processes, 
and both the x-ray temperature and surface brightness correlate 
strongly with cluster mass \citep{Kravtsov06}.
Careful consideration must be given, however, to biases 
caused by deficiencies of physical models of the ICM.  Theoretical x-ray
mass-observable relations rely on a few simplifying assumptions, most notably
the assumption of hydrostatic equilibrium of the ICM.  This assumption
neglects important aspects of ICM physics, such as bulk gas flow
and thermal structure in the plasma, which are difficult to observe and 
model.  Because of this incomplete knowledge of the true physical 
state of the ICM, cluster mass can be systematically 
underestimated by as much as 20-40\% \citep{Rasia06,Nagai07,Jeltema08}.
Nevertheless, through judicious accounting for these biases, 
x-ray surveys, in conjunction with supernovae and BAO results, 
have led to competetive constraints on cosmological quantities 
\citep[e.g.][]{Vikhlinin09a,Vikhlinin09b}.