\chapter{Introduction to Cosmology}

By the early part of the 20th century, the general scientific concensus
was that the Universe comprised a collection of perhaps a few billion
stars in a ``bun-shaped'' distribution, occupying a static space
of perhaps infinite age \citep{smith2009expanding}.
That model was challenged by the work of many in the early years of the
twentieth century.
Harlow Shapley's observations of globular clusters showed that our solar
system is not near the center of our Galaxy. Edwin Hubble's discovery of
Henrietta Levitt's Cepheid variables in the so-called ``spiral nebulae''
showed their distances to be larger than we had ever imagined.  Hubble's
observational discovery of the seemingly uniform expansion of the
universe gave credence to the theoretical work of Einstein, Lemaitre,
Friedmann, Robinson, Walker, and other luminaries of Physical Cosmology.
Since then, a huge mass of theoretical and observational work has aided
our understanding of the dynamics of this expansion, as
well as it's implications on observations of Big Bang Nucleosynthesis
and the baryonic content of the universe,
recombination and the resulting cosmic background radiation,
gravitational instabilities and the formation of structure,
and the gravitational and dynamical effects postulated to be due to
previously unknown quantities dubbed
``dark matter'' and ``dark energy'', which together
make up over 95\% of the mass-energy content of the universe.

With such a wide and diverse field as Cosmology, we can't hope to offer
a complete introduction of the relevant theory.  For this purpose there
are several very well-written books available; much of the material
discussed below is taken from formalism developed more fully in these works
\citep[see, e.g.][]{ryden2003cosmology, peebles1993principles, peacock1999cosmological}

This chapter will cover the basic physical and mathematical background of
cosmology.  We will begin with a discussion of the
FLRW metric (named for Friedmann, Lemaitre,
Robinson, and Walker) which describes the geometry of space-time.
Next we'll move on to define the Friedmann Equations, which boil-down
the field equations of Einstein's General Relativity to the basic pieces
needed to describe the dynamics of a globally homogeneous and isotropic
universe.  We will briefly discuss the relevant theory behind gravitational
structure formation within this model, including the use of clustering
models and Fourier power spectra to relate observations to theory.
Finally, we will develop the equations describing gravitational lensing
in the weak limit, and show how weak lensing observations can be used
to gain insight into the parameters of our cosmological model.
Throughout, we'll point out the relevant observational work which supports
and constrains these theories.

\section{FLRW Metric}
\label{sec:FLRW}
The physical study of cosmology is based on the assumption of symmetry:
that the universe on the largest scales is homogeneous and isotropic.
Homogeneity is an expression of translational symmetry: the appearance of
the universe does not depend on the location of the observer.  Isotropy
is an expression of rotational symmetry: the appearance of the universe
does not change with respect to the orientation of the observer.
These assumptions are clearly incorrect at small scales -- our galaxy
has a much higher density of stars in the central bulge than in the
outer halo, for example -- but these assumptions appear to hold at
the largest scales.  At distance scales larger than the size of
typical galaxy clusters (about 10 Mpc or more), the distribution of
quasars and galaxies reflect the nearly homogeneous and isotropic
nature of large scale structure.  More importantly, the Cosmic Microwave
Background appears homogeneous and isotropic to within one part in
$10^5$, giving evidence that our assumptions of homogeneity and isotropy
are well-founded for the universe as a whole.

The most general metric for a homogeneous and isotropic space-time is due
to Howard Robertson and Arthur Walker, who showed that the space-time
distance $ds$ in spherical coordinates is given by
\begin{equation}
  \label{eq:FLRW_metric}
  ds^2 = -c^2 dt^2 + a(t)^2\left[dr^2 + S_\kappa^2(r)d\Phi^2\right]
\end{equation}
where $t$ is the time coordinate, $r$ and $\Phi$ are the spatial coordinate,
$a(t)$ describes the distance scale (which may be an arbitrary function
of $t$), and $S_\kappa^2(r)$ is the curvature term.  The curvature term
depends on the curvature, $\kappa$, which may be either positive, negative,
or zero:
\begin{equation}
  \label{eq:FLRW_curvature}
  S_\kappa(r) = \left\{
  \begin{array}{ll}
    R\,\sin(r/R) & \kappa = +1\\
    r & \kappa = 0\\
    R\,\sinh(r/R) & \kappa = -1
  \end{array}
  \right.
\end{equation}
where $R$ is the radius of curvature today.  Often, the curvature
sign $\kappa$ and radius $R$ are compactly expressed in a single curvature
parameter $k$, such that $\kappa = k/|k|$ and $R = |k|^{-1/2}$.

An interesting aspect of this metric is the scale factor $a(t)$.  A general
homogeneous and isotropic universe is not necessarily static: it can be
expanding or contracting with time.  The nature of this expansion cannot
be derived from purely geometric means.  The description of the dynamics
of cosmic expansion comes from the field equations of Einstein's theory
of General Relativity.

\section{The Friedmann Equations}
\label{sec:friedmann}
The Robertson-Walker metric shown above is a purely geometric result,
where the scale factor $a(t)$ is arbitrary and unspecified.
Friedmann and Lemaitre had earlier independently derived this expression
from Einstein's field equations, with the addition of certain dynamical
constraints on the scale factor.  For this reason, the Robertson-Walker
metric is often referred to as the Friedmann-Robertson-Walker metric
or the Friedmann-Lemaitre-Robertson-Walker (FLRW) metric.
The general relativistic constraints on the scale factor $a(t)$
are compactly expressed by the Friedmann
equations\footnote{For a GR-based derivation of the Friedmann
  equations, refer to \citet{peebles1993principles}}:
\begin{equation}
  \label{eq:friedmann_1}
  \left(\frac{\dot{a}}{a}\right)^2
  = \frac{8\pi G}{3c^2}\varepsilon
  + \frac{\Lambda}{3} - \frac{\kappa c^2}{a^2 R^2}
\end{equation}
\begin{equation}
  \label{eq:friedmann_2}
  \frac{\ddot{a}}{a}
  = -\,\frac{4\pi G}{3c^2}(\varepsilon + 3P) + \frac{\Lambda}{3}.
\end{equation}
Here the scale factor $a$ is understood to be a function of time, and the
dots represent derivatives with respect to time.
By convention, the scale factor at the present day is $a(t_0) = 1$.
$\varepsilon$ and $P$ are
the energy density and pressure of the mass-energy in the universe, and
$\Lambda$ represents the cosmological constant.
Equations~\ref{eq:friedmann_1} and \ref{eq:friedmann_2} are the first and
second Friedmann equations, respectively.  The third Friedmann equation
can be easily derived from the first two:
\begin{equation}
  \label{eq:friedmann_3}
  \dot{\varepsilon} = -3\,\frac{\dot{a}}{a}\,(\varepsilon + P).
\end{equation}
This expression is equivalent to the first law of thermodynamics
expressed for the universe as a whole.

We can simplify these further by writing the pressure $P$ and energy
density $\varepsilon$ in terms of an equation of state parameter
\begin{equation}
  \label{eq:w_EOS}
  w \equiv P / \varepsilon.
\end{equation}
Using this, the solution of eqn.~\ref{eq:friedmann_3} gives
\begin{equation}
  \varepsilon = \varepsilon_0\, a^{-3(1 + w)}
\end{equation}
for $w$ constant in time\footnote{the equivalent expression for
  non-constant $w(t)$ is slightly more complicated, but can be easily
  derived from eqn.~\ref{eq:friedmann_3}.}.
Here $\varepsilon_0 = \varepsilon(t_0)$ is the energy density today,
and we have used the standard convention $a(t_0) = 1$.
Given this parametrization, we can now
separate the various contributions to the mass-energy of the universe
and re-write eqn.~\ref{eq:friedmann_1} in terms of the equation of
state for each:
\begin{equation}
  \label{eq:friedmann_1_split}
  \left(\frac{\dot{a}}{a}\right)^2 = \frac{8\pi G}{3c^2}
  \sum_w \varepsilon_{w, 0} \,\, a^{-3(1 + w)}
\end{equation}
where $\varepsilon_{w,0}$ is the energy density of each species at present.
The various possible contributions are:
\begin{description}
  \item[Non-relativistic matter:] Non-relativistic matter (often known
    as {\it cold matter}) has kinetic energy much less than its rest mass;
    in other words $P \sim kT \ll \varepsilon$.  So non-relativistic matter
    has $w = 0$
  \item[Radiation:] Radiation has energy per particle
    proportional to the momentum times the speed of light.  From basic
    electrodynamics, one can show that for an ideal photon gas, each spatial
    degree of freedom contributes equally to the energy, so that the pressure
    is $P = dp/dt = \varepsilon / 3$.  So relativistic mass-energy has
    $w = 1/3$.
  \item[Vacuum energy:] The vacuum energy or cosmological constant has
    constant energy density.  So by Equation~\ref{eq:friedmann_3},
    $P = -\varepsilon$ and $w = -1$.
  \item[Curvature:] Though it may seem strange to think about the curvature
    of space as having an energy density, in General Relativity the curvature
    is in some sense a stand-in for gravitational potential energy.  Comparing
    eqns.~\ref{eq:friedmann_1} and \ref{eq:friedmann_1_split}, the dependence
    of the curvature term on scale factor means it has an effective equation
    of state parameter $w = -1/3$.  This makes it clear why curvature does
    not appear in eqn.~\ref{eq:friedmann_2}: for $w=-1/3$,
    $\varepsilon + 3P = 0$, and the presence of curvature cannot lead to
    a change in the expansion rate.
  \item[General Quintessence:] Quintessence is defined as any sort of matter
    which can balance the gravitational attraction, leading to accelerated
    expansion.  By Equation~\ref{eq:friedmann_2}, $\ddot{a}/a > 0$ only
    if $w < - 1/3$.  We see that the cosmological constant is a form of
    quintessence.
  \item[Relativistic Matter:] Relativistic matter has energy given by
    $E^2 = p^2c^2 + m^2 c^4$, where $p$ is the total momentum and $m$ is
    the rest-mass.  If $pc \ll mc^2$, we have the non-relativistic
    case above, and find $w \to 0$.  If $pc \gg mc^2$, we have the radiation
    case, and find $w \to 1/3$.  For general relativistic matter, then, we have
    $0 \le w \le 1/3$, with the exact value dependent on the energy density.
\end{description}

The first Friedmann equation (eqn.~\ref{eq:friedmann_1}) is commonly expressed
in terms of dimensionless parameters via the generalization in
eqn.~\ref{eq:friedmann_1_split}.  If we define the Hubble parameter
\begin{equation}
  \label{eq:hubble_parameter}
  H \equiv \frac{\dot{a}}{a},
\end{equation}
and let $H_0$ be the value of the hubble parameter today, then
eqn.~\ref{eq:friedmann_1_split} becomes
\begin{equation}
  \left(\frac{H}{H_0}\right)^2 = \frac{8\pi G}{3H_0^2c^2}
  \sum_w \varepsilon_{w, 0} \,\, a^{-3(1 + w)}.
\end{equation}
This motivates the definition of the critical density
\begin{equation}
  \label{eq:critical_density}
  \varepsilon_c \equiv \frac{3 H^2 c^2}{8\pi G},
\end{equation}
where, to be explicit, both the critical density $\varepsilon_c$ and
hubble parameter $H$ are functions of time.  With this definition,
and defining the dimensionless density parameter
\begin{equation}
  \label{eq:density_parameter}
  \Omega_w(t) \equiv \varepsilon_w(t) / \varepsilon_c(t)
\end{equation}
the Friedmann equation can be compactly expressed
\begin{equation}
  \left(\frac{H}{H_0}\right)^2
  = \sum_w \Omega_{w, 0}\,\, a^{-3(1 + w)},
\end{equation}
where the subscript $0$ indicates the value at present.
Alternatively, we can express the Friedmann equation as simply
\begin{equation}
  \sum_w \Omega_w(t) = 1.
\end{equation}
The most important contributors to the density of the universe are
dark energy $(\Omega_\Lambda)$, matter $(\Omega_M)$, radiation $(\Omega_R)$,
and curvature $(\Omega_\kappa)$.  Neglecting other components gives the
familiar dimensionless form of the Friedmann Equation:
\begin{equation}
  \label{eq:friedmann_dimensionless}
  \left(\frac{H}{H_0}\right)^2
  = \Omega_{M,0}\,\,a^{-3} + \Omega_{R,0}\,\,a^{-4}
  + \Omega_{\kappa,0}\,\,a^{-2} + \Omega_\Lambda
\end{equation}

\section{Redshift and distance measures}
\label{sec:redshift}
The FLRW metric of \S\ref{sec:FLRW} and the Friedmann equations of
\S\ref{sec:friedmann} lay the basic framework for the study of cosmology.
In many ways the history of $20^{\rm th}$ century cosmology surrounds various
attempts to understand the relative contributions of matter, radiation,
curvature, and dark energy to the hubble parameter, which measures
the expansion rate of the universe.  The exact nature of these various
contributions has far-reaching consequences,
determining how, when, and where galaxies, clusters and other structure
form and evolve; determining the age of the universe and its evolution
through time; determining cosmic abundances and 
the initial conditions of stellar evolution
and planet formation; and determining the nature of the universe's beginning,
and the possibility of its eventual end.

Eqn.~\ref{eq:friedmann_dimensionless} is simply a first-order differential
equation in $a$: For various choices of the density parameters $\Omega$, it
can be solved to yield a curve describing the scale factor $a$ as a function
of time $t$.  Placing observational constraints on the densities of
various components, then, would require simply measuring the value of $a$ at
several times $t$ and performing a multidimensional fit to these observed
data points.  But how can the scale factor $a$ be measured?
Conveniently, the nature of light allows straightforward determination
of the scale factor at the time that light was emitted.  

General Relativity tells us
that light always travels along null geodesics, that is, the space time
interval in eqn.~\ref{eq:FLRW_metric} satisfies $ds = 0$.  For a light
beam with no angular deflection $d\Omega$, this gives
\begin{equation}
  dr = \frac{c}{a(t)} dt.
\end{equation}
If a beam of light is emitted at time $t_e$ and travels
a comoving distance $r$, the
time $t_o$ that the light is observed can be found by solving
\begin{equation}
  r = \int_{t_e}^{t_o} \frac{c}{a(t)} dt
\end{equation}
If a second photon is emitted a short time later at time $t_e + \Delta t_e$,
and arrives at time $t_o + \Delta t_o$, this gives
\begin{eqnarray}
  r &=& 
  \int_{t_e + \Delta t_e}^{t_o + \Delta t_o} \frac{c}{a(t)} dt \nonumber\\
  &\approx& \int_{t_e}^{t_o} \frac{c}{a(t)} dt + \frac{c\Delta t_o}{a(t_o)}
  - \frac{c\Delta t_e}{a(t_e)},
\end{eqnarray}
where the second line is the first-order approximation.  Equating these
two expressions gives for small $\Delta t$:
\begin{equation}
  \label{eq:time_dialation}
  \Delta t_o = \Delta t_e \frac{a(t_o)}{a(t_e)}.
\end{equation}
If an atom emits light with a period 
$P_e = \Delta t_e = \lambda_e / c$, then the observed wavelength $\lambda_o$
and the emitted wavelength $\lambda_e$ are related by
\begin{equation}
  \lambda_o = \lambda_e \frac{a(t_o)}{a(t_e)}.
\end{equation}
The wavelength of light is lengthened due to the expansion of space.  For
historical reasons, this expansion is generally parametrized using the
redshift:
\begin{equation}
  1 + z \equiv \frac{a(t_o)}{a(t_e)}.
\end{equation}
Because we define $a(t_o) = 1$, we have
\begin{equation}
  a(t_e) = \frac{1}{1 + z}.
\end{equation}
Thus the redshift of a light source gives us a direct measurement of the
scale factor at the time that photon was emitted.  As such, it can be
substituted for $a$ as the dependent variable in the above equations
with the correct change-of-variables; we will switch between these two
conventions depending on what is convenient.

Measurement of the scale factor $a$ when the light was emitted
is only half the picture however: in order to constrain parameters in
eqn.~\ref{eq:friedmann_dimensionless}, we must be able to measure
information related to the time $t_e$ of emission.  This allows us to
constrain $a$ as a function of $t$, or the equivalent quantity in some
other parametrization.  To enable this, we'll introduce the concept
of distances in Cosmology.

Distance measures in cosmology are a potentially confusing subject.  An
excellent resource describing these can be found in \citet{hogg1999distance}.
We'll briefly define four relevant distance measures: the comoving distance,
the proper distance, the angular diameter distance, and the luminosity
distance.

\begin{description}
  \item[Comoving Distance:] The comoving distance is the distance $r$ which
    enters into the FLRW metric, eqn.~\ref{eq:FLRW_metric}.  This distance is
    constant for an object moving with the expansion of space.  Using the
    FLRW metric and setting $ds=0$, one can shown that the comoving
    distance for an object with redshift $z$ is given by
    \begin{equation}
      \label{eq:comoving_distance}
      r(z) = c\,\int_0^z \frac{dz^\prime}{H(z^\prime)}.
    \end{equation}
  \item[Proper Distance:] The proper distance is the simultaneous separation
    $d_p(z)$ between two objects.  From the FLRW metric with time interval
    $dt=0$, one can show that the proper distance is given by
    \begin{equation}
      \label{eq:proper_distance}
      d_p(z) = \frac{r(z)}{1 + z}.
    \end{equation}
  \item[Angular Diameter Distance:] The angular diameter distance $d_A(z)$
    is the
    ratio of the true size to the observed angular size (in radians) of
    an extended source.  From the FLRW metric with $dt = dr = 0$, one
    can show that the angular diameter distance is given by
    \begin{equation}
      \label{eq:angular_diameter_distance}
      d_A(z) = \frac{S_\kappa(r)}{1 + z}.
    \end{equation}
  \item[Luminosity Distance:] The luminosity distance $d_L(z)$ relates the
    emitted flux to the observed flux.  Taking into account the angular
    dilution of the flux, as well as the redshifted energy of each photon
    and time-delay of photon arrival leads to
    \begin{eqnarray}
      \label{eq:luminosity_distance}
      d_L(z) &=& (1 + z)^2\,d_A(z) \nonumber\\
      &=& (1 + z)\,S_\kappa(r)
    \end{eqnarray}
\end{description}
By measuring one of these distances as a function of redshift, we are
effectively constraining both the dependent and independent variable
in the dimensionless Friedmann equation, eqn.~\ref{eq:friedmann_dimensionless}.
In this way, it is possible to constrain the combinations of cosmological
parameters $\Omega_w$ which fit the data.

\section{Standard Candles: Cosmology via Luminosity Distance}
\label{sec:std_candles}
The earliest observations confirming the cosmic expansion (i.e. a positive
value of the hubble parameter $H_0$) were due to Edwin Hubble
\citep{hubble1929}.  Through observations of Cepheid-type variable stars
in nearby galaxies, Hubble found a linear correlation between the
luminosity distance to each galaxy and its apparent recessional velocity.
Although Hubble does not frame this measurement in terms of the Friedmann
equations, he cautiously mentions the potential relationship of the
observations to the ``de Sitter effect'', a particular solution of
Einstein's equations which is a special case of the FLRW metric and
Friedmann equations. Using our formalism above, we can show that for
nearby objects (such that $H(z) \approx H_0$),
the change in proper distance is approximated by
\begin{eqnarray}
  \frac{d}{dt}d_p(t) &=& H(t)\,\, d_p(t) \nonumber\\
                     &\approx& H_0\,\, d_L(t),
\end{eqnarray}
where the second line holds up to corrections of order $z$.  Because
Hubble used a sample with $z <\sim 0.005$, the linear fit is well within
observed errors.
Thus Hubble's observations can be seen as a first attempt at constraining
cosmological parameters in Equation~\ref{eq:friedmann_dimensionless} through
the simultaneous measurement of redshift and luminosity distance.

In the 70 years after Hubble's discovery, there were many attempts to
confirm and improve upon his discovery and use it to derive tighter
constraints on...

\comment{ Add some brief references to standard-candle-based constraints:
  Cepheid, MS fitting, TF relation, and type 1a supernovae.  Mention my
  work in this area: \citet{Kessler2009}}

The results summarized above are all related in that they are based on the
idea of a standard candle: if we can determine the intrinsic brightness
of an object as well as its redshift, then we can compare this to the
apparent brightness and constrain the hubble parameter $H(t)$.
Another path to this sort of constraint comes from standard {\it rulers}
rather than standard candles.  If we know the redshift as well as the
intrinsic size of an object, then we can use its apparent size to
constrain cosmological parameters.  One standard ruler is given by the
size-scales of structure in the universe.

\section{The Growth of Structure}
\label{sec:growth}
The above probes rely on standard or standardizable candles to measure the
luminosity distance as a function of redshift.  Other cosmological probes
depend on standard {\it rulers}, that is, objects whose true angular extent
is known, and can be used to determine the angular diameter distance as
a function of redshift.  Many of these probes depend on detailed modeling
of the growth of structure.

The distribution of density throughout the universe can be expressed
\begin{equation}
  \label{eq:overdensity}
  \rho(\myvec{x}, t) = \rho_b(t)[1 + \delta(\myvec{x}, t)]
\end{equation}
where we have taken out the background component,
\begin{equation}
  \rho_b(t) \propto \frac{1}{a(t)^3}.
\end{equation}

\subsection{Gravitational Instability}
We can proceed by treating matter as an ideal fluid with a
velocity field $\myvec{u}(\myvec{x}, t)$ and pressure $P(\myvec{x}, t)$
\citep{longair2008galaxy}.
In this case, it is governed by the
continuity equation, which describes conservation of mass,
\begin{equation}
  \label{eq:continuity}
  \left(\frac{\partial\rho}{\partial t}\right)_{\myvec{x}}
  + \rho\myvec{\nabla}_{\myvec{x}}\cdot\myvec{u} = 0,
\end{equation}
the Euler equation, which specifies conservation of momentum,
\begin{equation}
  \label{eq:euler}
  \left(\frac{\partial\myvec{u}}{\partial t}\right)_{\myvec{x}}
  + (\myvec{u}\cdot\myvec{\nabla}_{\myvec{x}})\myvec{u}
  + \frac{\myvec{\nabla}_{\myvec{x}}P}{\rho}
  = -\myvec{\nabla}_{\myvec{x}}\Phi,
\end{equation}
and the Poisson equation, which describes gravity in the Newtonian limit
\begin{equation}
  \label{eq:poisson}
  \nabla_{\myvec{x}}^2\Phi = 4\pi G\rho.
\end{equation}
Transforming to comoving coordinates $\myvec{r} = \myvec{x}/a(t)$,
defining comoving time derivatives
$d/dt \equiv \partial/\partial t + \myvec{u}\cdot\myvec{\nabla}$,
and expressing in terms of the dimensionless density contrast
$\delta(\myvec{r}, t)$ (eq.~\ref{eq:overdensity}) gives
\begin{equation}
  \label{eq:delta_evolution}
  \frac{d^2\delta}{d t^2} + 2 H \frac{d\delta}{d t}
  = \frac{c_s^2}{a^2}\nabla_{\myvec{r}}^2\delta + 4\pi G \rho_b\,\delta
\end{equation}
where $\nabla^2_{\myvec{r}}$ indicates the Laplacian with respect to comoving
coordinates \citep[for derivation see][\S11.2]{longair2008galaxy}.

It is convenient to look at wave-like solutions of the denstity contrast
$\delta(\myvec{r}, t) \propto \exp[i(\myvec{k}\cdot\myvec{r} - \omega t)]$
such that eqn.~\ref{eq:delta_evolution} for each fourier mode $\delta_k$
becomes
\begin{equation}
  \label{eq:delta_evolution_k}
  \ddot{\delta}_k + 2 H \dot{\delta}_k - \delta_k(4\pi G\rho_b - k^2c_s^2) = 0.
\end{equation}
Thus we see that the growth of perturbations $\delta$ with time depends
on the balance between the gravitational force through $4\pi G\rho_b$ and the
pressure through $k^2c_s^2$.  The scale where the left hand side of
Equation~\ref{eq:delta_evolution_k} is exactly zero is called the Jeans
length:
\begin{equation}
  \label{eq:jeans_length}
  \lambda_J \equiv c_s \sqrt{\frac{\pi}{G\rho_b}}.
\end{equation}
This is the approximate scale above which pressure cannot halt gravitational
collapse.  The length is directly proportional to the sound speed
$c_s = \sqrt{\partial P/\partial \rho}$ and so depends on the equation of
state of the total energy in the universe,
as well as the average density $\rho_b$.
The different components (radiation, matter, etc.) have different
equations of state (\S\ref{sec:friedmann}) and evolve
with different dependencies on the scale factor
$a$, and so the Jeans length also evolves through the course of cosmic
history.  Thus the scale of nonlinear
collapsed structure as a function of $z$ contains information which
be used to place constraints on the components which make up the Universe.

In particular, we can consider two relevant regimes: radiation dominance
and matter dominance.  In the regime where radiation dominates the energy
density, the equation of state $w=1/3$ leads to $c_s^2 = c^2/3$.  In a flat
universe $\rho_b$ is the critical density (eqn.~\ref{eq:critical_density})
and the Jeans length can be expressed
\begin{equation}
  \label{eq:jeans_radiation}
  \lambda_J^{(R)} = \frac{\pi c \sqrt{8}}{3 H}.
\end{equation}
This scale is on the same order as that of the horizon scale,
$\lambda_s \approx 2c / (H\sqrt{3})$.

In the matter-dominated regime, there are two possibilities: if radiation and
matter is coupled, the pressure comes from the radiation
even as the density is dominated by matter.  This gives
$c_s^2 \sim c^2 \Omega_r / \Omega_M \sim c^2 (1 + z) \Omega_{r,0}/\Omega_{m, 0}$.
Putting in numbers from WMAP, we find approximately
$c_s \approx 10^{-2} c \sqrt{1 + z}$.
If radiation and matter are decoupled, the pressure comes from the nonzero
temperature of matter itself:
$c_s^2 \sim kT/m_p$ where $m_p$ is the proton mass.  Assuming the matter
is in thermal equilibrium with the CMB, then temperature goes as
$T \propto (1 + z)$.  Putting in observational numbers, we find
$c_s \approx 10^{-7} c \sqrt{1 + z}$.

A related question is that of the rate of structure growth
on scales larger than the Jeans length. 
This can be addressed by defining the linear growth
factor $D(t)$ such that
\begin{equation}
  \label{eq:linear_growth_def}
  \delta(\myvec{r}, t) = \delta_0(\myvec{r})D(t).
\end{equation}
Using $\rho_b = \Omega_M\rho_c$ and assuming negligible pressure (i.e. scales
above the Jeans length), we can express eqn.~\ref{eq:delta_evolution} to find
\begin{equation}
  \label{eq:linear_growth_eqn}
  \ddot{D} + 2H\dot{D} - \frac{3}{2}\Omega_M H^2D = 0.
\end{equation}
This is a second-order differential equation, which will, in general,
admit a solution with a growing mode and a decaying mode:
\begin{equation}
  D(t) = A_1 D_1(t) + A_2 D_2(t).
\end{equation}
In a flat universe dominated by matter, the
Friedmann equation gives $H(t) = 2 / (3t)$, leading to solutions
\begin{equation}
  D(t) = A_1 t^{2/3} + A_2 t^{-1}.
\end{equation}
The first term quickly dominates the second, and we see that for a
matter-dominated universe, structure grows as
$\delta \propto t^{2/3} \propto a$.

In general, the growth factor for a flat universe is
\begin{equation}
  \label{eq:linear_growth}
  D(a) \propto \int_0^a [a^\prime H(a^\prime)]^{-3}da^\prime,
\end{equation}
where the normalization is usually chosen such that $D(a) = 1$ at the
present day.

A radiation-dominated universe presents a more complicated case:
eqn.~\ref{eq:linear_growth_eqn} assumes the pressure is negligible compared
to the gravitational force.  This approximation breaks down in
cases when $\Omega_M \to 0$.  For these cases, we need a more involved
perturbative treatment.

\subsection{Perturbation Treatment}
To explore the growth rate in a universe where $\Omega_M$ is small, we
will perform a perturbation analysis of Friedmann's equations.
A full discussion of this treatment can be found in
\citet{peebles1993principles}.
Here we will briefly outline a schematic approach from \citet{kolb_turner}
which leads to the same results.

By Birkhoff's theorem, a small spherical over-density can be treated as if
it were an independent homogeneous universe embedded within the background.
We'll assume the background is represented by a flat universe with
\begin{equation}
  H^2 = \frac{8\pi G}{3c^2}\varepsilon_0,
\end{equation}
and that a spherical perturbation has a small positive curvature
\begin{equation}
  H^2 = \frac{8\pi G}{3c^2}\varepsilon_1 - \frac{\kappa c^2}{a^2R_0^2}.
\end{equation}
The boundary requires that the expansion rate $H$ be equal between the
two; combining these we find
\begin{equation}
  \delta \equiv \frac{\varepsilon_1 - \varepsilon_0}{\varepsilon_0}
  = \frac{3\kappa c^4}{8\pi G R_0^2}\,\frac{1}{a^2 \varepsilon_0}.
\end{equation}
For a matter-dominated universe, $\varepsilon_0 \propto a^{-3}$
and we find $\delta \propto a$ as above.
For a radiation-dominated universe, $\varepsilon_0 \propto a^{-4}$
and we find $\delta \propto a^2$.

\subsection{Matter Power Spectrum}
In summary, the above results show that
\begin{description}
  \item[In the radiation-dominated regime], fluctuations on scales above
    $\lambda_J^{(R)} = (\pi c \sqrt{8})/(3 H)$ grow as
    $\delta(a) \propto a^2$.
  \item[In the matter-dominated regime], after decoupling, scales above
    $\lambda_J^{(M)}\approx 10^{-7} \lambda_J^{(R)} \sqrt{1 + z}$
    grow as $\delta(a) \propto a$.
\end{description}
The ratio of radiation density to matter density is
\begin{equation}
  \frac{\Omega_R(z)}{\Omega_M(z)} = (1 + z)\frac{\Omega_{R,0}}{\Omega_{M,0}}.
\end{equation}
So before the redshift of radiation-matter equality,
$z_{rm} \approx \Omega_{M,0}/\Omega_{R,0}$, radiation
dominates, while after this redshift matter dominates.
Thus the important scale is the horizon scale at $z_{rm}$.
Modes on length-scales $k > \lambda_{rm}$ will grow as $\delta \propto a^2$
for $a < a_{rm}$, and $\delta \propto a$ for $a > a_{rm}$.  Modes with
length-scales $k < \lambda_{rm}$ will grow as $\delta \propto a^2$ as long
as the horizon distance $d_{hor}(a) < k$, at which point the growth will
be suppressed by radiation pressure.  At $a > a_{rm}$, the Jeans length
shrinks by a factor of about $10^5$, and modes larger than $\lambda_J^{M}$
resume growth with $\delta \propto a$.

Thus, density modes with $k < k_{rm}$ are suppressed by a factor of
$(a_k / a_{rm})^2 \propto k^{-2}$.  This motivates use of the power
spectrum of density fluctuations:
\begin{equation}
  \label{eq:power_spectrum}
  P_k \equiv \langle|\delta_k|^2\rangle,
\end{equation}
in theory a measurable quantity, which will have a distinct break at
$k = k_{rm}$ for the reasons discussed above.

\subsection{Putting it all together}
The sum of the above discussion paints a general picture of the growth of
structure within the universe presents several concepts with readily
observable consequences:

\begin{itemize}
  \item The size and mass/length scale of clusters as a function of redshift
    depends on the length-scale of gravitational instability $\lambda_J$
    (eqn.~\ref{eq:jeans_length}),
    which in turn depends on the densities of matter and radiation in the
    universe.  Clustering also depends on the linear growth rate
    $D(z)$ (eqn.~\ref{eq:linear_growth}) and its nonlinear extentions,
    which also depend on the relative cosmic densities as a function of $z$.
  \item The power spectrum of density fluctuations (eq.~\ref{eq:power_spectrum})
    displays a length scale which is closely related to the horizon distance
    at the epoch of radiation-matter equality.  This scale acts as a standard
    ruler, such that the angular diameter distance can be estimated at a
    particular value of $z$, leading to cosmological constraints through the
    same means as the standard candle method discussed in
    \S\ref{sec:std_candles}.
  \item The linear growth factor (eqn.~\ref{eq:linear_growth}) affects the
    normalization of the power spectrum.  Therefore, measuring the power
    spectrum as a function of $z$ can lead to cosmological constraints
    through the dependence of $D(z)$ on cosmological parameters.
\end{itemize}

So we see that there are powerful cosmological constraints which can be
obtained through the observation of the density fluctuations and clustering
of matter through the universe.  There are several caveats, however:
the above discussion focuses on the linear approximation (that is, we
discuss the behavior of perturbations of order $\delta$, while ignoring
$\delta^2$).  This is sufficient for small $\delta$, but not for when
$\delta$ is much larger than 1.  At small scales, structure is well beyond
the regime where this approximation holds: for example, $\delta$ for our
galaxy is approximately $10^5$!  So nonlinear corrections are essential when
deriving constraints from clustering.  Even moderate-sized galaxy clusters
(i.e. a hypothetical cluster 2Mpc across, containing 50 Milky-way
sized galaxies) has $\delta$ of order $10^2-10^3$.

Secondly, the structure we are referring to here is that made up by the bulk
of the matter in the universe: collisionless dark matter.  Dark matter,
being non-luminous, cannot be obeserved directly through emitted light.
The power spectrum of luminous matter can be theoretically mapped to the
underlying mass power spectrum, but this mapping introduces systematic
errors which can be difficult to correct for.

In order to circumvent the error involved with mapping luminous matter to
the underlying dark matter, it would be preferable to observe the dark
matter directly.  This is where weak gravitational lensing can be a
useful tool.

\comment{ Mention measurements of LRG power spectra \& cosmic parameter estimates.}

\comment{ Should we also briefly mention WMAP anisotropies and the BAO signal?
  They're essentially very accurate standard rulers at different epochs.}

\section{Gravitational Lensing}
\label{sec:gravitational_lensing}
Einstein's theory of General Relativity predicts that photons will be deflected
in the presence of a gravitational field.  Under certain circumstances, this
deflection can be detected and used to learn about the nature of the
gravitating matter.

\subsection{Simplifying Assumptions}
\label{sec:lensing_simplification}
The propagation of light through a region of gravitational potential
$\Phi(\vec{r})$ is, in general, a very complicated\ problem, only analytically
solvable for potentials with various symmetries.  In cosmological contexts,
however, it is safe to assume that the universe is described by a
Robertson-Walker metric, with only small perturbations due to the density
fluctuations described by the potential $\Phi$.  In this case, the
gravitational deflection of a photon can be described by an effective
index of refraction given by 
\citep[see][and references therein]{narayan1996lectures}:
\begin{equation}
  n = 1-\frac{2}{c^2}\Phi 
\end{equation}
As in conventional optics, light rays are deflected in proportion to the
perpendicular gradient of the refraction index, such that the deflection angle
$\hat{\alpha}$ is given by
\begin{equation}
  \label{eq:alpha-def}
  \hat{\alpha} = -\int_0^{D_S} \vec{\nabla}_\perp n \,\,\dd D
  = \frac{2}{c^2}\int_0^{D_S} \vec{\nabla}_\perp\Phi\,\,\dd D
\end{equation}
where $D_S$ is the distance from the observer to the photon source.  

For a point-mass located at a distance $D_L$ and an impact parameter $b$, with $D_L \gg b$ and $D_S \approx 2D_L$, equation \ref{eq:alpha-def} can be integrated to give
\begin{equation}
  \hat{\alpha} = \frac{4GM}{bc^2}\Bigg[1 - \frac{1}{2}\bigg(\frac{b}{D_L}\bigg)^2 + \mathcal{O}\bigg[\frac{b}{D_L}\bigg]^3 \Bigg]
\end{equation}
The first-order approximation is twice the deflection predicted by Newtonian gravity for a particle of arbitrary mass moving at a speed $c$.  It is important to note here that to first order, the deflection does not depend on the distance to the lens or source.  That is, for a mass distribution located at a distance $D_L$, equation \ref{eq:alpha-def} can be approximated
\begin{equation}
  \label{eq:alpha-def-approx}
  \hat{\alpha} \approx \frac{2}{c^2}\int_{D_L-\delta D}^{D_L+\delta D} \vec{\nabla}_\perp\Phi\,\dd D
\end{equation}
for $\delta D$ sufficiently greater than the size scale of the mass-distribution in question.
%Also note that for a large lens distance $D_L$, the contribution to $\hat{\alpha}$ becomes vanishingly small, and can be neglected.
So, to a very good approximation, the incremental deflection $\delta \hat{\alpha}$ of a photon at a given point along its trajectory is entirely due to a sheet of matter with a width $2\,\delta D$, oriented perpendicular to the unperturbed photon trajectory.  

\subsection{Lensing Geometry}
\comment{Lensing diagram figure Here?}

For a mass-sheet located at a distance $D_L$, and a photon source located at a distance $D_S$ (with $D_{LS} = D_S - D_L$) geometric considerations in the small-angle approximation yield the relation
\begin{equation}
  \vec{\theta} = \vec{\beta} + \frac{D_{LS}}{D_S}\hat{\vec{\alpha}}
\end{equation}
where $\vec{\theta}$ and $\vec{\beta}$ are the observed and true positions of the source, respectively.  Rescaling $\hat{\vec{\alpha}}$ in more convenient units gives
\begin{equation}
  \label{eq:mapping}
  \vec{\theta} = \vec{\beta} + \vec{\alpha}
\end{equation}
where we have defined
\begin{equation}
  \label{eq:alpha-def2}
  \vec{\alpha} \equiv \frac{D_{LS}}{D_S}\hat{\vec{\alpha}}
\end{equation}

\subsection{Continuous Mass Distribution}
In the case of a continuous mass distribution, we can recall the remarks of section \ref{sec:lensing_simplification}, and define a surface-mass density for a mass-sheet located at a redshift $z_L$:
\begin{equation}
  \label{eq:sigma-def}
  \Sigma(\vec{\theta},z_L) 
  = \int_{D_L-\delta D}^{D_L+\delta D}\rho_M(\vec{\theta},D)\dd D 
  = \frac{1}{c^2}\int_{z_L - \delta z}^{z_L + \delta z} 
  \varepsilon_M(\vec{\theta},z)\frac{dD}{dz}\dd z
\end{equation}
where $\varepsilon_M \equiv \rho_M c^2$ is the energy density of matter,
$\vec{\theta}$ is the apparent angular position, and $z$ and $D(z)$ are the 
redshift and line-of-sight distance, respectively, with $D_S = D(z_s)$.  

A matter distribution $\rho_M(\vec{\theta},z)$, and its Newtonian potential $\Phi(\vec{\theta},z)$ are related by Poisson's equation:
\begin{equation}
  \label{eq:poisson}
  \nabla^2 \Phi(\vec{\theta},z) = 4\pi G \rho_M(\vec{\theta},z)
\end{equation}

It is convenient to define the unscaled lensing potential $\hat{\psi}$, 
given by
\begin{equation}
  \hat{\psi}(\vec{\theta},z_s) 
  = \int_{0}^{D(z_s)} \Phi(\vec{\theta},z(D))\,\,\dd D 
  = \int_{0}^{z_s}\Phi(\vec{\theta},z)\frac{dD}{dz}\,\,\dd z
\end{equation}
Using the approximation in equation \ref{eq:alpha-def-approx}, we can write this in terms of multiple mass-sheets, such that
\begin{equation}
  \hat{\psi} = \sum_i \delta \hat{\psi}_i 
  = \sum_i \int_{D_i-\delta D}^{D_i+\delta D}\Phi \dd D
\end{equation}
with $D_{i+1} = D_i + 2\delta D$.

The gradient of $\hat{\psi}$ with respect to 
$\vec{\xi} \equiv D_L\vec{\theta}$ is
\begin{equation}
  \vec{\nabla}_\xi\hat{\psi} 
  = \sum_i \vec{\nabla}_\xi\big(\delta\hat{\psi}_i\big) 
  = \sum_i \int_{D_i - \delta D}^{D_i+\delta D}\vec{\nabla}_\xi \Phi \dd D
\end{equation}
Comparing this with (\ref{eq:alpha-def-approx}) and (\ref{eq:alpha-def2}) gives the incremental deflection angle in terms of the lensing potential of a mass-sheet:
\begin{equation}
  \delta\vec{\alpha}_i = \frac{2}{c^2}\frac{D_{LS}}{D_S}\vec{\nabla}_\xi (\delta\hat{\psi}_i)
\end{equation}
Further simplification can be made by rescaling the lensing potential,
defining
\begin{equation}
  \delta\psi_i = \frac{2}{c^2}\frac{D_{LS}}{D_L D_S} \delta\hat{\psi}_i
\end{equation}
so that we are left with
\begin{equation}
  \label{eq:alpha-psi}
  \delta\vec{\alpha}_i(\vec{\theta},z_L) = \vec{\nabla}_\theta \Big( \delta\psi_i(\vec{\theta},z_L)\Big)
\end{equation}
Defining the total scaled lensing potential $\psi = \sum_i \delta\psi_i$, and the total deflection $\vec{\alpha} = \sum_i\delta\vec{\alpha}_i$, we obtain
\begin{equation}
  \vec{\alpha}(\vec{\theta},z_s) = \vec{\nabla}_\theta \psi(\vec{\theta},z_s)
\end{equation}

The Laplacian of $\delta\psi_i$ with respect to theta is given by
\begin{equation}
  \nabla_\theta^2 (\delta\psi_i) 
  = \frac{2}{c^2}\frac{D_{LS}D_L}{D_S}\int_{D_L-\delta D}^{D_L+\delta D} 
  \nabla_\xi^2\Phi \dd D
\end{equation}
Using (\ref{eq:sigma-def}) and (\ref{eq:poisson}) this becomes
\begin{equation}
  \label{eq:psi-sigma_init}
  \nabla_\theta^2(\delta\psi_i) = \frac{8\pi G}{c^2}\frac{D_{LS}D_L}{D_S}\Sigma(\vec{\theta},z_i)
\end{equation}

We now define the critical surface density,
\begin{equation}
  \Sigma_{c}(z) \equiv\frac{c^2 D_S}{4\pi G D_L(z) D_{LS}(z)}
\end{equation}
and the convergence
\begin{equation}
  \label{eq:kappa-sigma}
  \kappa(\vec{\theta},z_s) \equiv \sum \frac{\Sigma(\vec{\theta},z_i)}{\Sigma_c(z_i)}\ \forall\ z_i < z_s
\end{equation}
Now summing all the mass-sheets in (\ref{eq:psi-sigma_init}) gives the relation between the scaled lensing potential and the convergence
\begin{equation}
  \label{eq:psi-kappa-1}
  \nabla_\theta^2\psi(\vec{\theta},z_s) = 2\kappa(\vec{\theta},z_s)
\end{equation}

Solving this two-dimensional differential equation gives the effective potential in terms of the convergence:
\begin{equation}
  \label{eq:psi-kappa}
  \psi(\vec{\theta},z) 
  = \frac{1}{\pi}\int_{\mathbb{R}^2} \kappa(\vec{\theta}^\prime,z) 
  \ln|\vec{\theta} - \vec{\theta}^\prime|\dd^2\theta^\prime
\end{equation}

\section{Weak Gravitational Lensing}

The local properties of the mapping in (\ref{eq:mapping}) are contained in its Jacobian matrix, given by
\begin{equation}
  \label{eq:Jacobian_def}
  \mathcal{A} \equiv \frac{\partial \vec{\beta}}{\partial \vec{\theta}} = \Big(\delta_{ij} - \frac{\partial \alpha_i}{\partial \theta_j} \Big) = \Big(\delta_{ij} - \frac{\partial^2 \psi}{\partial \theta_i \partial \theta_j} \Big),
\end{equation}
where $i,j$ index the two components of the angular position.

Introducing the abbreviation
\begin{equation}
  \label{eq:psi_ij}
  \psi_{ij} = \frac{\partial^2 \psi}{\partial \theta_i \partial \theta_j}
\end{equation}
We can then rewrite the convergence $\kappa$ (eqn \ref{eq:psi-kappa-1}) and define the complex shear $\gamma \equiv \gamma_1 + i\gamma_2$ of the mapping:
\begin{equation}
  \label{eq:gamma-def}
  \begin{array}{lcl}
    \kappa & = & (\psi_{11} + \psi_{22})/2\\
    \gamma_1 & = & (\psi_{11} - \psi_{22})/2\\
    \gamma_2 & = & \psi_{21} = \psi_{12}
  \end{array}
\end{equation}
The local Jacobian matrix (\ref{eq:Jacobian_def}) of the lens mapping can then be written
\begin{equation}
  \label{eq:jacobian_kappa_gamma}
  \mathcal{A} = \left(
  \begin{array}{cc}
    1 - \kappa - \gamma_1 & -\gamma_2\\
    -\gamma_2             & 1-\kappa+\gamma_1
  \end{array}\right)
\end{equation}

Equations \ref{eq:psi-kappa}, \ref{eq:psi_ij} and \ref{eq:gamma-def} can be combined and simplified to yield the following relationship between the convergence and the shear, where for simplicity we define the complex angle
$\theta \equiv \theta_1 + i\theta_2$:
\begin{equation}
  \label{eq:gamma-kappa}
  \gamma(\theta) 
  = \frac{-1}{\pi}\int_{\mathbb{R}^2} \mathcal{D}(\theta - 
  \theta^\prime)\kappa(\theta^\prime) \dd^2\theta^\prime
\end{equation}
where 
\begin{equation}
  \label{eq:scriptD}
  \mathcal{D}(\theta) 
  = \frac{\theta_1^2 - \theta_2^2 + 2i\theta_1\theta_2}{(\theta_1^2+\theta_2^2)^2}
  = \frac{\theta^2}{|\theta|^4}
\end{equation}
is the Kaiser-Squires kernel \citep{Kaiser93}.
The lens mapping in eqn.~\ref{eq:jacobian_kappa_gamma} describes an
image transformation consisting of a magnification with magnitude 
given by the real convergence $\kappa$
and a distortion with magnitude and orientation given by the complex shear
$\gamma = \gamma_1 + i\gamma_2$.  This distortion results in a measurable
effect, at least in principle.  If the intrinsic shape, size, or brightness
of a distant image were known, then the observed shape, size, or brightness
could be observed to determine the shear and convergence at that point.
Unfortunately, the intrinsic shape and size of a galaxy cannot be known
{\it a priori}, but using well-founded assumptions about the statistics
of the {\it distribution} of shapes and sizes of sources can lead to
useful estimates of the shear and/or convergence across the sky.

\comment{add some references: using gamma \& kappa on galaxies, number counts,
  galaxy-galaxy lensing, quasar variability-luminosity relationship, etc.}

In the most typical approach to weak lensing,
the ellipticities of source galaxies are measured, 
giving a noisy estimate of the shear $\gamma(\theta)$.
These measurements can be utilized in a number of ways to learn about
fundamental physical principles; this work will focus on three areas:
\begin{description}
  \item[Direct mapping:] Having measured the shear $\gamma$ at locations
    across the sky, the convergence $\kappa$ can be estimated.  $\kappa$
    relates to the projected density via eqn.~\ref{eq:kappa-sigma}.  Thus
    the measured shear can be used to directly estimate a map of the
    distribution of dark matter in two dimensions.  Using redshift
    information for the lensed sources, there is the potential to extend
    this mapping to three dimensions. This is the subject of Chapter 3.
  \item[Peak statistics:] The two dimensional maps recovered as above
    represent a two-dimensional projection of the three-dimensional
    distribution of large scale structure, in particular massive galaxy
    clusters.  As discussed in \S\ref{sec:growth}, both the number of
    clusters and their mass distribution depend on the details of the
    geometry, expansion, and makeup of the universe.  By computing the
    statistics of observed lensing peaks to 
    that predicted by theory, it is possible to
    constrain cosmological parameters using the peaks alone.  This is
    the subject of Chapter 4.
  \item[Power spectrum:] The power spectrum of the shear is closely related
    to the power spectrum of the matter distribution which generates it.
    By measuring two point information of observed shear, it is possible
    to constrain cosmological parameters in a way which is complementary
    to the peak counts mentioned above.  This is the subject of Chapter 5.
\end{description}

To enable these three analyses, we will develop a bit further the basic
principles of $\kappa$-mapping and $\gamma$ power spectra.

\subsection{Mapping with Weak Lensing}
The typical weak lensing mapping problem can be computed using the following
steps:
\begin{enumerate}
  \item From the measured ellipticities and redshifts of photometrically
    observed galaxies, obtain noisy estimates of the shear
    $\gamma_{obs}(\theta, z)$.
  \item Using eqn.~\ref{eq:gamma-kappa}, recover a local estimate of
    $\kappa(\theta, z)$.  Note that due to the integral over the lensing
    kernel $\mathcal{D}(\theta)$, the convergence estimate is non-local:
    the value of $\kappa$ at a given location is related to the value of the
    $\gamma$ at {\it all other} locations.
  \item Using eqn.~\ref{eq:kappa-sigma}, determine the projected density
    $\Sigma(\theta, z)$.
  \item As a final step, it is possible in principle to use
    eqn.~\ref{eq:sigma-def} to recover the 3D mass density
    $\rho(\theta, z)$.  This is the subject of Chapter 3.
\end{enumerate}
To accomplish this, it is convenient to combine steps 3-4 and write 
the expression for $\kappa(\theta,z)$ in terms of $\rho(\theta,z)$
explicitly.  From (\ref{eq:sigma-def}) and (\ref{eq:kappa-sigma}),
approximating the sum as an integral, we find
\begin{equation}
  \kappa(\vec{\theta},z_s) 
  = 4\pi G \int_0^{z_s} 
  \frac{D^{(A)}(z)[D^{(A)}(z_s)-D^{(A)}(z)]}{D^{(A)}(z_s)} 
  \rho_M(\vec{\theta},z) \frac{dD^{(A)}(z)}{dz} \dd z
\end{equation}
The notation has been changed here to make clear that the distances in 
question are in fact angular diameter distance, the relevant distance 
in the context of lensing calculations.  Recall that angular diameter 
distance $D^{(A)}$ is related to the comoving distance $D$ by
\begin{equation}
  D^{(A)}(z) = a S_\kappa (D)
\end{equation}
where $S_\kappa(D) = D$ for a flat universe.  Assuming a flat universe,
converting to comoving distances, 
and writing this in terms of $\varepsilon = \rho c^2$, we find
\begin{equation}
  \label{kappa-epsilon-1}
  \kappa(\vec{\theta},z_s) 
  = \frac{4\pi G}{c^2} \int_0^{z_s} \dd z\frac{dD}{dz} 
  a^2\frac{D(D_S-D)}{D_S} \varepsilon_M(\vec{\theta},z),
\end{equation}
where we've used the shorthand $D \equiv D(z)$ and 
$D_S \equiv D(z_s)$.

To further progress, we can follow \S\ref{sec:growth} and write the
matter density $\varepsilon_M(\theta, z)$ in equation
\ref{kappa-epsilon-1} in terms of the density contrast $\delta$:
\begin{equation}
  \label{delta-def}
  \varepsilon_M(\vec{\theta},z) = \Omega_M(z) \varepsilon_c(z)\Big[1+\delta(\vec{\theta},z)\Big]
\end{equation}
where we have assumed a flat universe, such that the total density is
equal to the critical density $\varepsilon_c(z)$
(eqn.~\ref{eq:critical_density}).
We'll make use of two further algebraic substitutions:
from the definition of comoving distance (eq.~\ref{eq:comoving_distance}),
we can write
\begin{equation}
  \label{dDdz}
  \frac{dD}{dz} = \frac{c}{H(z)},
\end{equation}
and from the Friedmann equation (eqn.~\ref{eq:friedmann_dimensionless})
matter density fraction can be written
\begin{equation}
  \Omega_M(z) = \frac{H_0^2\Omega_{M,0}(1+z)^3}{[H(z)]^2}.
\end{equation}


Combining these equations gives
\begin{equation}
  \label{eq:kappa-delta}
  \kappa(z_s) = \frac{3cH_0^2\Omega_{M,0}}{2}\int_0^{z_s} \dd z \frac{(1+z)}{H(z)} \frac{D(D_S-D)}{D_S}\big[1+\delta(z)\big]
\end{equation}

Because of the mass-sheet degeneracy, $\kappa(z_s)$ can only be determined
up to an additive constant across a given redshift bin
\citep[see][for discussion]{seitz_schneider1996}.
Defining $\bar{\kappa}(z_s)$ to be the convergence due to the background
matter distribution in matter-dominated growth,
and $\Delta(z) \equiv \delta(z)/a$ we find
\begin{equation}
  \label{eq:kappa-delta-2}
  \kappa(z_s) \equiv \hat{\kappa}(z_s)-\bar\kappa(z_s) = 
  \frac{3cH_0^2\Omega_{M,0}}{2}\int_0^{z_s} \dd z 
  \frac{1}{H(z)} \frac{D(D_S-D)}{D_S}\Delta(z)
\end{equation}
where, to be explicit,
\begin{equation}
  \bar\kappa(z_s) = \frac{3cH_0^2\Omega_{M,0}}{2}\int_0^{z_s} \dd z 
  \frac{(1+z)}{H(z)} \frac{D(D_S-D)}{D_S}
\end{equation}
To be clear, here, $D$ is the comoving distance to a redshift $z$,
and $D_S$ is the comoving distance to the redshift $z_s$ of the photon
source.  Equation \ref{eq:kappa-delta-2} defines the mapping from $\kappa(z_s)$
to $\Delta(z)$ for $z<z_s$.

\subsection{Power Spectra}
Mass mapping can lead to deep astrophysical and cosmological insights through
the comparison of dark and luminous matter distributions
\citep[e.g.][]{Clowe2006}, through constraints on the mass profiles of
collapsed structures \citep[e.g.][]{Oguri2012}, or
through the comparison of observed mass peaks to
theoretical predictions (see Chapter 4).  Because of the noise inherent in
lensing observations, most of these localized analyses are limited to
very dense regions, far from the linear regime.

The linear regime, as well as the presence of nonlinear effects on small
scales, can be measured using power spectra of the weak lensing shear.  In
order to accomplish this, however, the power spectra of observed shear must
be related to the mass power spectra discussed in \S\ref{sec:growth}.

In \S\ref{sec:growth}, we defined the power spectrum
\begin{equation}
  P_\delta(k) = \langle |\hat\delta_k|^2 \rangle,
\end{equation}
where
The power spectrum 

\subsection{E and B modes}
In this section, we will outline the basic results of \citet{Schneider02b}.
We will start by defining the E/B decomposition of the shear field $\gamma$.
If the shear $\gamma$ and convergence $\kappa$ can be expressed as shown
in eqn.~\ref{eq:gamma-def}, then the gradient of $\kappa$ can be written
\begin{equation}
  \label{eq:u_def}
  \myvec{u} \equiv
  \nabla_\theta \kappa =
  \left(
  \begin{array}{l}
    \partial \kappa / \partial\theta_1\\
    \partial \kappa / \partial\theta_2 
  \end{array}
  \right) 
  =
  \left(
  \begin{array}{l}
    \partial \gamma_1 / \partial\theta_1 + \partial\gamma_2/\partial\theta_2\\
    \partial \gamma_2 / \partial\theta_1 - \partial\gamma_1/\partial\theta_2
  \end{array}
  \right).
\end{equation}
If $\kappa$ and $\gamma$ are due entirely to weak lensing, then the vector
$\myvec{u}$ should be a pure gradient field, and this will hold for the
quantity on the left hand side of eqn.~\ref{eq:u_def} as well.
If, however, other effects are involved (e.g.~shot noise, second-order effects,
intrinsic alignments, systematic errors, etc.)
then $\myvec{u}$ will not be a pure gradient field and will have a curl
component as well.  This ``curl-free'' condition can be compactly expressed
\begin{equation}
  \label{eq:curl-free}
  \myvec{\nabla} \times \myvec{u} = 0.
\end{equation}
When effects other than pure weak lensing are present, however, this condition
may not hold.
With this in mind, we will use an analogy from electrodynamics and decompose
$\kappa$ into a curl-free ``E-mode'' $\kappa_E$ and a divergence-free
``B-mode'' $\kappa_B$ such that
\begin{eqnarray}
  \nabla^2 \kappa_E &=& \myvec{\nabla} \cdot \myvec{u}\\
  \nabla^2 \kappa_B &=& \myvec{\nabla} \times \myvec{u}.
\end{eqnarray}
We'll also define the E-mode and B-mode lensing potential following
eqn.~\ref{eq:psi-kappa-1}:
\begin{equation}
  \nabla^2 \psi_{E, B} = 2\kappa_{E, B}.
\end{equation}
This allows us to define the E and B modes of $\gamma$ via
eqn.~\ref{eq:gamma-def}.  Explicitly,
\begin{equation}
  \gamma_{E,B} = \left(\frac{1}{2}\left[\frac{\partial^2}{\partial \theta_1\partial\theta_1}
  - \frac{\partial^2}{\partial \theta_2\partial\theta_2}\right]
  + i\frac{\partial^2}{\partial \theta_1\partial\theta_2}\right)
  \psi_{E,B}.
\end{equation}
Combining the convergence E and B modes as a complex linear combination
$\kappa = \kappa_E + i\kappa_B$, we find in analogy to
eqn.~\ref{eq:gamma-kappa},
\begin{equation}
  \label{eq:gamma-kappa}
  \left[\gamma_E(\theta) + i\gamma_B(\theta)\right]
  = \frac{-1}{\pi}\int_{\mathbb{R}^2} \mathcal{D}(\theta - 
  \theta^\prime)
  \left[\kappa_E(\theta^\prime) + i\kappa_B(\theta^\prime)\right]
  \dd^2\theta^\prime
\end{equation}


\begin{itemize}
  \item Weak gravitational lensing limit: $\gamma$ and $\kappa$
  \item Mass mapping from weak lensing surveys
  \item Power spectra of $\gamma$ and $\kappa$: how do these relate to
    the correlation functions from previous section?
  \item A note about the practical side: shape measurement, shot noise,
    etc.
\end{itemize}

{\it Notes:}
\begin{itemize}
  \item Mention work with the supernova collaboration \citep{Kessler2009}.
  \item We also should briefly mention here \cite{Jain2011} and alternatives
    to standard cosmological models \citep[also][]{Sollerman2009}.
\end{itemize}
