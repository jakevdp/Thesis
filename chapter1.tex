\chapter{Introduction to Cosmology}

By the early part of the 20th century, the general scientific concensus
was that the Universe comprised a collection of perhaps a few billion
stars in a ``bun-shaped'' distribution, occupying a static space
of perhaps infinite age \citep{smith2009expanding}.
That model was challenged by the work of many in the early years of the
twentieth century.
Harlow Shapley's observations of globular clusters showed that our solar
system is not near the center of our Galaxy. Edwin Hubble's discovery of
Henrietta Levitt's Cepheid variables in the so-called ``spiral nebulae''
showed their distances to be larger than we had ever imagined.  Hubble's
observational discovery of the seemingly uniform expansion of the
universe gave credence to the theoretical work of Einstein, Lemaitre,
Friedmann, Robinson, Walker, and other luminaries of Physical Cosmology.
Since then, a huge mass of theoretical and observational work has aided
our understanding of the dynamics of this expansion, as
well as it's implications on observations of Big Bang Nucleosynthesis
and the baryonic content of the universe,
recombination and the resulting cosmic background radiation,
gravitational instabilities and the formation of structure,
and the gravitational and dynamical effects postulated to be due to
previously unknown quantities dubbed
``dark matter'' and ``dark energy'', which together
make up over 95\% of the mass-energy content of the universe.

With such a wide and diverse field as Cosmology, we can't hope to offer
a complete introduction of the relevant theory.  For this purpose there
are several very well-written books available; much of the material
discussed below is taken from formalism developed more fully in these works
\citep[see, e.g.][]{ryden2003cosmology, peebles1993principles, peacock1999cosmological}

This chapter will cover the basic physical and mathematical background of
cosmology.  We will begin with a discussion of the
FLRW metric (named for Friedmann, Lemaitre,
Robinson, and Walker) which describes the geometry of space-time.
Next we'll move on to define the Friedmann Equations, which boil-down
the field equations of Einstein's General Relativity to the basic pieces
needed to describe the dynamics of a globally homogeneous and isotropic
universe.  We will briefly discuss the relevant theory behind gravitational
structure formation within this model, including the use of clustering
models and Fourier power spectra to relate observations to theory.
Finally, we will develop the equations describing gravitational lensing
in the weak limit, and show how weak lensing observations can be used
to gain insight into the parameters of our cosmological model.
Throughout, we'll point out the relevant observational work which supports
and constrains these theories.

\section{FLRW Metric}
The physical study of cosmology is based on the assumption of symmetry:
that the universe on the largest scales is homogeneous and isotropic.
Homogeneity is an expression of translational symmetry: the appearance of
the universe does not depend on the location of the observer.  Isotropy
is an expression of rotational symmetry: the appearance of the universe
does not change with respect to the orientation of the observer.
These assumptions are clearly incorrect at small scales -- our galaxy
has a much higher density of stars in the central bulge than in the
outer halo, for example -- but these assumptions appear to hold at
the largest scales.  At distance scales larger than the size of
typical galaxy clusters (about 10 Mpc or more), the distribution of
quasars and galaxies reflect the nearly homogeneous and isotropic
nature of large scale structure.  More importantly, the Cosmic Microwave
Background appears homogeneous and isotropic to within one part in
$10^5$, giving evidence that our assumptions of homogeneity and isotropy
are well-founded for the universe as a whole.

The most general metric for a homogeneous and isotropic space-time is due
to Howard Robertson and Arthur Walker, who showed that the space-time
distance $ds$ in spherical coordinates is given by
\begin{equation}
  \label{eq:FLRW_metric}
  ds^2 = -c^2 dt^2 + a(t)^2\left[dr^2 + S_\kappa^2(r)d\Phi^2\right]
\end{equation}
where $t$ is the time coordinate, $r$ and $\Phi$ are the spatial coordinate,
$a(t)$ describes the distance scale (which may be an arbitrary function
of $t$), and $S_\kappa^2(r)$ is the curvature term.  The curvature term
depends on the curvature, $\kappa$, which may be either positive, negative,
or zero:
\begin{equation}
  \label{eq:FLRW_curvature}
  S_\kappa(r) = \left\{
  \begin{array}{ll}
    R\,\sin(r/R) & \kappa = +1\\
    r & \kappa = 0\\
    R\,\sinh(r/R) & \kappa = -1
  \end{array}
  \right.
\end{equation}
where $R$ is the radius of curvature today.  Often, the curvature
sign $\kappa$ and radius $R$ are compactly expressed in a single curvature
parameter $k$, such that $\kappa = k/|k|$ and $R = |k|^{-1/2}$.

An interesting aspect of this metric is the scale factor $a(t)$.  A general
homogeneous and isotropic universe is not necessarily static: it can be
expanding or contracting with time.  The nature of this expansion cannot
be derived from purely geometric means.  The description of the dynamics
of cosmic expansion comes from the field equations of Einstein's theory
of General Relativity.

\section{The Friedmann Equations}
The Robertson-Walker metric shown above is a purely geometric result,
where the scale factor $a(t)$ is arbitrary and unspecified.
Friedmann and Lemaitre had earlier independently derived this expression
from Einstein's field equations, with the addition of certain dynamical
constraints on the scale factor.  For this reason, the Robertson-Walker
metric is often referred to as the Friedmann-Robertson-Walker metric
or the Friedmann-Lemaitre-Robertson-Walker (FLRW) metric.
The general relativistic constraints on the scale factor $a(t)$
are compactly expressed by the Friedmann
equations\footnote{For a GR-based derivation of the Friedmann
  equations, refer to \citet{peebles1993principles}}:
\begin{equation}
  \label{eq:friedmann_1}
  \left(\frac{\dot{a}}{a}\right)^2
  = \frac{8\pi G}{3c^2}\varepsilon
  + \frac{\Lambda}{3} - \frac{\kappa c^2}{a^2 R^2}
\end{equation}
\begin{equation}
  \label{eq:friedmann_2}
  \frac{\ddot{a}}{a}
  = -\,\frac{4\pi G}{3c^2}(\varepsilon + 3P) + \frac{\Lambda}{3}.
\end{equation}
Here the scale factor $a$ is understood to be a function of time, and the
dots represent derivatives with respect to time.
By convention, the scale factor at the present day is $a(t_0) = 1$.
$\varepsilon$ and $P$ are
the energy density and pressure of the mass-energy in the universe, and
$\Lambda$ represents the cosmological constant.
Equations~\ref{eq:friedmann_1} and \ref{eq:friedmann_2} are the first and
second Friedmann equations, respectively.  The third Friedmann equation
can be easily derived from the first two:
\begin{equation}
  \label{eq:friedmann_3}
  \dot{\varepsilon} = -3\,\frac{\dot{a}}{a}\,(\varepsilon + P).
\end{equation}
This expression is equivalent to the first law of thermodynamics
expressed for the universe as a whole.

We can simplify these further by writing the pressure $P$ and energy
density $\varepsilon$ in terms of an equation of state parameter
\begin{equation}
  \label{eq:w_EOS}
  w \equiv P / \varepsilon.
\end{equation}
Using this, the solution of eqn.~\ref{eq:friedmann_3} gives
\begin{equation}
  \varepsilon = \varepsilon_0\, a^{-3(1 + w)}
\end{equation}
for $w$ constant in time\footnote{the equivalent expression for
  non-constant $w(t)$ is slightly more complicated, but can be easily
  derived from eqn.~\ref{eq:friedmann_3}.}.
Here $\varepsilon_0 = \varepsilon(t_0)$ is the energy density today,
and we have used the standard convention $a(t_0) = 1$.
Given this parametrization, we can now
separate the various contributions to the mass-energy of the universe
and re-write eqn.~\ref{eq:friedmann_1} in terms of the equation of
state for each:
\begin{equation}
  \label{eq:friedmann_1_split}
  \left(\frac{\dot{a}}{a}\right)^2 = \frac{8\pi G}{3c^2}
  \sum_w \varepsilon_{w, 0} \,\, a^{-3(1 + w)}
\end{equation}
where $\varepsilon_{w,0}$ is the energy density of each species at present.
The various possible contributions are:
\begin{description}
  \item[Non-relativistic matter:] Non-relativistic matter (often known
    as {\it cold matter}) has kinetic energy much less than its rest mass;
    in other words $P \sim kT \ll \varepsilon$.  So non-relativistic matter
    has $w = 0$
  \item[Radiation:] Radiation has energy per particle
    proportional to the momentum times the speed of light.  From basic
    electrodynamics, one can show that for an ideal photon gas, each spatial
    degree of freedom contributes equally to the energy, so that the pressure
    is $P = dp/dt = \varepsilon / 3$.  So relativistic mass-energy has
    $w = 1/3$.
  \item[Vacuum energy:] The vacuum energy or cosmological constant has
    constant energy density.  So by Equation~\ref{eq:friedmann_3},
    $P = -\varepsilon$ and $w = -1$.
  \item[Curvature:] Though it may seem strange to think about the curvature
    of space as having an energy density, in General Relativity the curvature
    is in some sense a stand-in for gravitational potential energy.  Comparing
    eqns.~\ref{eq:friedmann_1} and \ref{eq:friedmann_1_split}, the dependence
    of the curvature term on scale factor means it has an effective equation
    of state parameter $w = -1/3$.  This makes it clear why curvature does
    not appear in eqn.~\ref{eq:friedmann_2}: for $w=-1/3$,
    $\varepsilon + 3P = 0$, and the presence of curvature cannot lead to
    a change in the expansion rate.
  \item[General Quintessence:] Quintessence is defined as any sort of matter
    which can balance the gravitational attraction, leading to accelerated
    expansion.  By Equation~\ref{eq:friedmann_2}, $\ddot{a}/a > 0$ only
    if $w < - 1/3$.  We see that the cosmological constant is a form of
    quintessence.
  \item[Relativistic Matter:] Relativistic matter has energy given by
    $E^2 = p^2c^2 + m^2 c^4$, where $p$ is the total momentum and $m$ is
    the rest-mass.  If $pc \ll mc^2$, we have the non-relativistic
    case above, and find $w \to 0$.  If $pc \gg mc^2$, we have the radiation
    case, and find $w \to 1/3$.  For general relativistic matter, then, we have
    $0 \le w \le 1/3$, with the exact value dependent on the energy density.
\end{description}

The first Friedmann equation (eqn.~\ref{eq:friedmann_1}) is commonly expressed
in terms of dimensionless parameters via the generalization in
eqn.~\ref{eq:friedmann_1_split}.  If we define the Hubble parameter
\begin{equation}
  \label{eq:hubble_parameter}
  H \equiv \frac{\dot{a}}{a},
\end{equation}
and let $H_0$ be the value of the hubble parameter today, then
eqn.~\ref{eq:friedmann_1_split} becomes
\begin{equation}
  \left(\frac{H}{H_0}\right)^2 = \frac{8\pi G}{3H_0^2c^2}
  \sum_w \varepsilon_{w, 0} \,\, a^{-3(1 + w)}.
\end{equation}
This motivates the definition of the critical density
\begin{equation}
  \label{eq:critical_density}
  \varepsilon_c \equiv \frac{3 H^2 c^2}{8\pi G},
\end{equation}
where, to be explicit, both the critical density $\varepsilon_c$ and
hubble parameter $H$ are functions of time.  With this definition,
and defining the dimensionless density parameter
\begin{equation}
  \label{eq:density_parameter}
  \Omega_w(t) \equiv \varepsilon_w(t) / \varepsilon_c(t)
\end{equation}
the Friedmann equation can be compactly expressed
\begin{eqnarray}
  \left(\frac{H}{H_0}\right)^2
  &=& \sum_w \Omega_w(t) \nonumber\\
  &=& \sum_w \Omega_{w, 0}\,\, a^{-3(1 + w)}.
\end{eqnarray}
where the subscript $0$ indicates the value of a time-dependent quantity
at the present.
The most important contributors to the density of the universe are
dark energy $(\Omega_\Lambda)$, matter $(\Omega_M)$, radiation $(\Omega_R)$,
and curvature $(\Omega_\kappa)$.  Neglecting other components gives the
familiar dimensionless form of the Friedmann Equation:
\begin{equation}
  \left(\frac{H}{H_0}\right)^2
  = \Omega_{M,0}\,\,a^{-3} + \Omega_{R,0}\,\,a^{-4}
  + \Omega_{\kappa,0}\,\,a^{-2} + \Omega_\Lambda
\end{equation}

In many ways the history of $20^{\rm th}$ century cosmology is an attempt to
understand the relative contributions of matter, radiation, curvature,
and dark energy to the hubble parameter, which measures the expansion
rate of the universe.

Talk about redshift and Hubble...

\begin{itemize}
  \item Einstein's blunder
  \item Standard candles
  \item Cepheids and Hubble's measurement
  \item Tully-Fisher, etc.
  \item supernovae
\end{itemize}

The results summarized above are all related in that they are based on the
idea of a standard candle: if we can determine the intrinsic brightness
of an object as well as its redshift, then we can compare this to the
apparent brightness and constrain the hubble parameter $H(t)$.
Another path to this sort of constraint comes from standard {\it rulers}
rather than standard candles.  If we know the redshift as well as the
intrinsic size of an object, then we can use its apparent size to
constrain cosmological parameters.  One standard ruler is given by the
size-scales of structure in the universe.


\section{The Growth of Structure}

\begin{itemize}
  \item Overdensities/underdensities and linear growth approximation
  \item Power spectra \& correlation functions \& the relationship to
    parameters from the previous section.
  \item nonlinear effects: clustering
  \item power spectra, BAO, WMAP anisotropies
\end{itemize}

\section{Gravitational Lensing}
\begin{itemize}
  \item Basic lensing geometry
  \item brief description of strong lensing, microlensing, weak lensing
\end{itemize}

\section{Weak Gravitational Lensing}
\begin{itemize}
  \item Weak gravitational lensing limit: $\gamma$ and $\kappa$
  \item Mass mapping from weak lensing surveys
  \item Power spectra of $\gamma$ and $\kappa$: how do these relate to
    the correlation functions from previous section?
  \item A note about the practical side: shape measurement, shot noise,
    etc.
\end{itemize}

{\it Notes:}
\begin{itemize}
  \item Mention work with the supernova collaboration \citep{Kessler2009}.
  \item We also should briefly mention here \cite{Jain2011} and alternatives
    to standard cosmological models \citep[also][]{Sollerman2009}.
\end{itemize}
