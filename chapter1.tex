\chapter{Introduction to Cosmology}

By the early part of the 20th century, the general scientific concensus
was that the Universe comprised a collection of perhaps a few billion
stars in a ``bun-shaped'' distribution, occupying a static space
of perhaps infinite age \citep{smith2009expanding}.
That model was challenged by the work of many in the early years of the
twentieth century.
Harlow Shapley's observations of globular clusters showed that our solar
system is not near the center of our Galaxy. Edwin Hubble's discovery of
Henrietta Levitt's Cepheid variables in the so-called ``spiral nebulae''
showed their distances to be larger than we had ever imagined.  Hubble's
observational discovery of the seemingly uniform expansion of the
universe gave credence to the theoretical work of Einstein, Lemaitre,
Friedmann, Robinson, Walker, and other luminaries of Physical Cosmology.
Since then, a huge mass of theoretical and observational work has aided
our understanding of the dynamics of this expansion, as
well as it's implications on observations of Big Bang Nucleosynthesis
and the baryonic content of the universe,
recombination and the resulting cosmic background radiation,
gravitational instabilities and the formation of structure,
and the gravitational and dynamical effects postulated to be due to
previously unknown quantities dubbed
``dark matter'' and ``dark energy'', which together
make up over 95\% of the mass-energy content of the universe.

With such a wide and diverse field as Cosmology, we can't hope to offer
a complete introduction of the relevant theory.  For this purpose there
are several very well-written books available; much of the material
discussed below is taken from formalism developed more fully in these works
\citep[see, e.g.][]{ryden2003cosmology, peebles1993principles, peacock1999cosmological}

This chapter will cover the basic physical and mathematical background of
cosmology.  We will begin with a discussion of the
FLRW metric (named for Friedmann, Lemaitre,
Robinson, and Walker) which describes the geometry of space-time.
Next we'll move on to define the Friedmann Equations, which boil-down
the field equations of Einstein's General Relativity to the basic pieces
needed to describe the dynamics of a globally homogeneous and isotropic
universe.  We will briefly discuss the relevant theory behind gravitational
structure formation within this model, including the use of clustering
models and Fourier power spectra to relate observations to theory.
Finally, we will develop the equations describing gravitational lensing
in the weak limit, and show how weak lensing observations can be used
to gain insight into the parameters of our cosmological model.
Throughout, we'll point out the relevant observational work which supports
and constrains these theories.

\section{FLRW Metric}

\section{The Friedmann Equations}
\begin{itemize}
  \item Just the basics
\end{itemize}

\begin{itemize}
  \item Discuss derivation and dimensionless representation
  \item Cepheids, MS fitting, Tully-Fisher relationship, supernovae
\end{itemize}

\section{The Growth of Structure}

\begin{itemize}
  \item Overdensities/underdensities and linear growth approximation
  \item Power spectra \& correlation functions \& the relationship to
    parameters from the previous section.
  \item nonlinear effects: clustering
  \item power spectra, BAO, WMAP anisotropies
\end{itemize}

\section{Gravitational Lensing}
\begin{itemize}
  \item Basic lensing geometry
  \item brief description of strong lensing, microlensing, weak lensing
\end{itemize}

\section{Weak Gravitational Lensing}
\begin{itemize}
  \item Weak gravitational lensing limit: $\gamma$ and $\kappa$
  \item Mass mapping from weak lensing surveys
  \item Power spectra of $\gamma$ and $\kappa$: how do these relate to
    the correlation functions from previous section?
  \item A note about the practical side: shape measurement, shot noise,
    etc.
\end{itemize}

{\it Notes:}
\begin{itemize}
  \item Mention work with the supernova collaboration \citep{Kessler2009}.
  \item We also should briefly mention here \cite{Jain2011} and alternatives
    to standard cosmological models \citep[also][]{Sollerman2009}.
\end{itemize}
