\chapter{Introduction to Cosmology}

By the early part of the 20th century, the general scientific concensus
was that the Universe comprised a collection of perhaps a few billion
stars in a ``bun-shaped'' distribution, occupying a static space
of perhaps infinite age \citep{smith2009expanding}.
That model was challenged by the work of many in the early years of the
twentieth century.
Harlow Shapley's observations of globular clusters showed that our solar
system is not near the center of our Galaxy. Edwin Hubble's discovery of
Henrietta Levitt's Cepheid variables in the so-called ``spiral nebulae''
showed their distances to be larger than we had ever imagined.  Hubble's
observational discovery of the seemingly uniform expansion of the
universe gave credence to the theoretical work of Einstein, Lemaitre,
Friedmann, Robinson, Walker, and other luminaries of Physical Cosmology.
Since then, a huge mass of theoretical and observational work has aided
our understanding of the dynamics of this expansion, as
well as it's implications on observations of Big Bang Nucleosynthesis
and the baryonic content of the universe,
recombination and the resulting cosmic background radiation,
gravitational instabilities and the formation of structure,
and the gravitational and dynamical effects postulated to be due to
previously unknown quantities dubbed
``dark matter'' and ``dark energy'', which together
make up over 95\% of the mass-energy content of the universe.

With such a wide and diverse field as Cosmology, we can't hope to offer
a complete introduction of the relevant theory.  For this purpose there
are several very well-written books available; much of the material
discussed below is taken from formalism developed more fully in these works
\citep[see, e.g.][]{ryden2003cosmology, peebles1993principles, peacock1999cosmological}

This chapter will cover the basic physical and mathematical background of
cosmology.  We will begin with a discussion of the
FLRW metric (named for Friedmann, Lemaitre,
Robinson, and Walker) which describes the geometry of space-time.
Next we'll move on to define the Friedmann Equations, which boil-down
the field equations of Einstein's General Relativity to the basic pieces
needed to describe the dynamics of a globally homogeneous and isotropic
universe.  We will briefly discuss the relevant theory behind gravitational
structure formation within this model, including the use of clustering
models and Fourier power spectra to relate observations to theory.
Finally, we will develop the equations describing gravitational lensing
in the weak limit, and show how weak lensing observations can be used
to gain insight into the parameters of our cosmological model.
Throughout, we'll point out the relevant observational work which supports
and constrains these theories.

\section{FLRW Metric}
\label{sec:FLRW}
The physical study of cosmology is based on the assumption of symmetry:
that the universe on the largest scales is homogeneous and isotropic.
Homogeneity is an expression of translational symmetry: the appearance of
the universe does not depend on the location of the observer.  Isotropy
is an expression of rotational symmetry: the appearance of the universe
does not change with respect to the orientation of the observer.
These assumptions are clearly incorrect at small scales -- our galaxy
has a much higher density of stars in the central bulge than in the
outer halo, for example -- but these assumptions appear to hold at
the largest scales.  At distance scales larger than the size of
typical galaxy clusters (about 10 Mpc or more), the distribution of
quasars and galaxies reflect the nearly homogeneous and isotropic
nature of large scale structure.  More importantly, the Cosmic Microwave
Background appears homogeneous and isotropic to within one part in
$10^5$, giving evidence that our assumptions of homogeneity and isotropy
are well-founded for the universe as a whole.

The most general metric for a homogeneous and isotropic space-time is due
to Howard Robertson and Arthur Walker, who showed that the space-time
distance $ds$ in spherical coordinates is given by
\begin{equation}
  \label{eq:FLRW_metric}
  ds^2 = -c^2 dt^2 + a(t)^2\left[dr^2 + S_\kappa^2(r)d\Phi^2\right]
\end{equation}
where $t$ is the time coordinate, $r$ and $\Phi$ are the spatial coordinate,
$a(t)$ describes the distance scale (which may be an arbitrary function
of $t$), and $S_\kappa^2(r)$ is the curvature term.  The curvature term
depends on the curvature, $\kappa$, which may be either positive, negative,
or zero:
\begin{equation}
  \label{eq:FLRW_curvature}
  S_\kappa(r) = \left\{
  \begin{array}{ll}
    R\,\sin(r/R) & \kappa = +1\\
    r & \kappa = 0\\
    R\,\sinh(r/R) & \kappa = -1
  \end{array}
  \right.
\end{equation}
where $R$ is the radius of curvature today.  Often, the curvature
sign $\kappa$ and radius $R$ are compactly expressed in a single curvature
parameter $k$, such that $\kappa = k/|k|$ and $R = |k|^{-1/2}$.

An interesting aspect of this metric is the scale factor $a(t)$.  A general
homogeneous and isotropic universe is not necessarily static: it can be
expanding or contracting with time.  The nature of this expansion cannot
be derived from purely geometric means.  The description of the dynamics
of cosmic expansion comes from the field equations of Einstein's theory
of General Relativity.

\section{The Friedmann Equations}
\label{sec:friedmann}
The Robertson-Walker metric shown above is a purely geometric result,
where the scale factor $a(t)$ is arbitrary and unspecified.
Friedmann and Lemaitre had earlier independently derived this expression
from Einstein's field equations, with the addition of certain dynamical
constraints on the scale factor.  For this reason, the Robertson-Walker
metric is often referred to as the Friedmann-Robertson-Walker metric
or the Friedmann-Lemaitre-Robertson-Walker (FLRW) metric.
The general relativistic constraints on the scale factor $a(t)$
are compactly expressed by the Friedmann
equations\footnote{For a GR-based derivation of the Friedmann
  equations, refer to \citet{peebles1993principles}}:
\begin{equation}
  \label{eq:friedmann_1}
  \left(\frac{\dot{a}}{a}\right)^2
  = \frac{8\pi G}{3c^2}\varepsilon
  + \frac{\Lambda}{3} - \frac{\kappa c^2}{a^2 R^2}
\end{equation}
\begin{equation}
  \label{eq:friedmann_2}
  \frac{\ddot{a}}{a}
  = -\,\frac{4\pi G}{3c^2}(\varepsilon + 3P) + \frac{\Lambda}{3}.
\end{equation}
Here the scale factor $a$ is understood to be a function of time, and the
dots represent derivatives with respect to time.
By convention, the scale factor at the present day is $a(t_0) = 1$.
$\varepsilon$ and $P$ are
the energy density and pressure of the mass-energy in the universe, and
$\Lambda$ represents the cosmological constant.
Equations~\ref{eq:friedmann_1} and \ref{eq:friedmann_2} are the first and
second Friedmann equations, respectively.  The third Friedmann equation
can be easily derived from the first two:
\begin{equation}
  \label{eq:friedmann_3}
  \dot{\varepsilon} = -3\,\frac{\dot{a}}{a}\,(\varepsilon + P).
\end{equation}
This expression is equivalent to the first law of thermodynamics
expressed for the universe as a whole.

We can simplify these further by writing the pressure $P$ and energy
density $\varepsilon$ in terms of an equation of state parameter
\begin{equation}
  \label{eq:w_EOS}
  w \equiv P / \varepsilon.
\end{equation}
Using this, the solution of eqn.~\ref{eq:friedmann_3} gives
\begin{equation}
  \varepsilon = \varepsilon_0\, a^{-3(1 + w)}
\end{equation}
for $w$ constant in time\footnote{the equivalent expression for
  non-constant $w(t)$ is slightly more complicated, but can be easily
  derived from eqn.~\ref{eq:friedmann_3}.}.
Here $\varepsilon_0 = \varepsilon(t_0)$ is the energy density today,
and we have used the standard convention $a(t_0) = 1$.
Given this parametrization, we can now
separate the various contributions to the mass-energy of the universe
and re-write eqn.~\ref{eq:friedmann_1} in terms of the equation of
state for each:
\begin{equation}
  \label{eq:friedmann_1_split}
  \left(\frac{\dot{a}}{a}\right)^2 = \frac{8\pi G}{3c^2}
  \sum_w \varepsilon_{w, 0} \,\, a^{-3(1 + w)}
\end{equation}
where $\varepsilon_{w,0}$ is the energy density of each species at present.
The various possible contributions are:
\begin{description}
  \item[Non-relativistic matter:] Non-relativistic matter (often known
    as {\it cold matter}) has kinetic energy much less than its rest mass;
    in other words $P \sim kT \ll \varepsilon$.  So non-relativistic matter
    has $w = 0$
  \item[Radiation:] Radiation has energy per particle
    proportional to the momentum times the speed of light.  From basic
    electrodynamics, one can show that for an ideal photon gas, each spatial
    degree of freedom contributes equally to the energy, so that the pressure
    is $P = dp/dt = \varepsilon / 3$.  So relativistic mass-energy has
    $w = 1/3$.
  \item[Vacuum energy:] The vacuum energy or cosmological constant has
    constant energy density.  So by Equation~\ref{eq:friedmann_3},
    $P = -\varepsilon$ and $w = -1$.
  \item[Curvature:] Though it may seem strange to think about the curvature
    of space as having an energy density, in General Relativity the curvature
    is in some sense a stand-in for gravitational potential energy.  Comparing
    eqns.~\ref{eq:friedmann_1} and \ref{eq:friedmann_1_split}, the dependence
    of the curvature term on scale factor means it has an effective equation
    of state parameter $w = -1/3$.  This makes it clear why curvature does
    not appear in eqn.~\ref{eq:friedmann_2}: for $w=-1/3$,
    $\varepsilon + 3P = 0$, and the presence of curvature cannot lead to
    a change in the expansion rate.
  \item[General Quintessence:] Quintessence is defined as any sort of matter
    which can balance the gravitational attraction, leading to accelerated
    expansion.  By Equation~\ref{eq:friedmann_2}, $\ddot{a}/a > 0$ only
    if $w < - 1/3$.  We see that the cosmological constant is a form of
    quintessence.
  \item[Relativistic Matter:] Relativistic matter has energy given by
    $E^2 = p^2c^2 + m^2 c^4$, where $p$ is the total momentum and $m$ is
    the rest-mass.  If $pc \ll mc^2$, we have the non-relativistic
    case above, and find $w \to 0$.  If $pc \gg mc^2$, we have the radiation
    case, and find $w \to 1/3$.  For general relativistic matter, then, we have
    $0 \le w \le 1/3$, with the exact value dependent on the energy density.
\end{description}

The first Friedmann equation (eqn.~\ref{eq:friedmann_1}) is commonly expressed
in terms of dimensionless parameters via the generalization in
eqn.~\ref{eq:friedmann_1_split}.  If we define the Hubble parameter
\begin{equation}
  \label{eq:hubble_parameter}
  H \equiv \frac{\dot{a}}{a},
\end{equation}
and let $H_0$ be the value of the hubble parameter today, then
eqn.~\ref{eq:friedmann_1_split} becomes
\begin{equation}
  \left(\frac{H}{H_0}\right)^2 = \frac{8\pi G}{3H_0^2c^2}
  \sum_w \varepsilon_{w, 0} \,\, a^{-3(1 + w)}.
\end{equation}
This motivates the definition of the critical density
\begin{equation}
  \label{eq:critical_density}
  \varepsilon_c \equiv \frac{3 H^2 c^2}{8\pi G},
\end{equation}
where, to be explicit, both the critical density $\varepsilon_c$ and
hubble parameter $H$ are functions of time.  With this definition,
and defining the dimensionless density parameter
\begin{equation}
  \label{eq:density_parameter}
  \Omega_w(t) \equiv \varepsilon_w(t) / \varepsilon_c(t)
\end{equation}
the Friedmann equation can be compactly expressed
\begin{equation}
  \left(\frac{H}{H_0}\right)^2
  = \sum_w \Omega_{w, 0}\,\, a^{-3(1 + w)},
\end{equation}
where the subscript $0$ indicates the value at present.
Alternatively, we can express the Friedmann equation as simply
\begin{equation}
  \sum_w \Omega_w(t) = 1.
\end{equation}
The most important contributors to the density of the universe are
dark energy $(\Omega_\Lambda)$, matter $(\Omega_M)$, radiation $(\Omega_R)$,
and curvature $(\Omega_\kappa)$.  Neglecting other components gives the
familiar dimensionless form of the Friedmann Equation:
\begin{equation}
  \label{eq:friedmann_dimensionless}
  \left(\frac{H}{H_0}\right)^2
  = \Omega_{M,0}\,\,a^{-3} + \Omega_{R,0}\,\,a^{-4}
  + \Omega_{\kappa,0}\,\,a^{-2} + \Omega_\Lambda
\end{equation}

\section{Redshift and distance measures}
\label{sec:redshift}
The FLRW metric of \S\ref{sec:FLRW} and the Friedmann equations of
\S\ref{sec:friedmann} lay the basic framework for the study of cosmology.
In many ways the history of $20^{\rm th}$ century cosmology surrounds various
attempts to understand the relative contributions of matter, radiation,
curvature, and dark energy to the hubble parameter, which measures
the expansion rate of the universe.  The exact nature of these various
contributions has far-reaching consequences,
determining how, when, and where galaxies, clusters and other structure
form and evolve; determining the age of the universe and its evolution
through time; determining cosmic abundances and 
the initial conditions of stellar evolution
and planet formation; and determining the nature of the universe's beginning,
and the possibility of its eventual end.

Eqn.~\ref{eq:friedmann_dimensionless} is simply a first-order differential
equation in $a$: For various choices of the density parameters $\Omega$, it
can be solved to yield a curve describing the scale factor $a$ as a function
of time $t$.  Placing observational constraints on the densities of
various components, then, would require simply measuring the value of $a$ at
several times $t$ and performing a multidimensional fit to these observed
data points.  But how can the scale factor $a$ be measured?
Conveniently, the nature of light allows straightforward determination
of the scale factor at the time that light was emitted.  

General Relativity tells us
that light always travels along null geodesics, that is, the space time
interval in eqn.~\ref{eq:FLRW_metric} satisfies $ds = 0$.  For a light
beam with no angular deflection $d\Omega$, this gives
\begin{equation}
  dr = \frac{c}{a(t)} dt.
\end{equation}
If a beam of light is emitted at time $t_e$ and travels
a comoving distance $r$, the
time $t_o$ that the light is observed can be found by solving
\begin{equation}
  r = \int_{t_e}^{t_o} \frac{c}{a(t)} dt
\end{equation}
If a second photon is emitted a short time later at time $t_e + \Delta t_e$,
and arrives at time $t_o + \Delta t_o$, this gives
\begin{eqnarray}
  r &=& 
  \int_{t_e + \Delta t_e}^{t_o + \Delta t_o} \frac{c}{a(t)} dt \nonumber\\
  &\approx& \int_{t_e}^{t_o} \frac{c}{a(t)} dt + \frac{c\Delta t_o}{a(t_o)}
  - \frac{c\Delta t_e}{a(t_e)},
\end{eqnarray}
where the second line is the first-order approximation.  Equating these
two expressions gives for small $\Delta t$:
\begin{equation}
  \label{eq:time_dialation}
  \Delta t_o = \Delta t_e \frac{a(t_o)}{a(t_e)}.
\end{equation}
If an atom emits light with a period 
$P_e = \Delta t_e = \lambda_e / c$, then the observed wavelength $\lambda_o$
and the emitted wavelength $\lambda_e$ are related by
\begin{equation}
  \lambda_o = \lambda_e \frac{a(t_o)}{a(t_e)}.
\end{equation}
The wavelength of light is lengthened due to the expansion of space.  For
historical reasons, this expansion is generally parametrized using the
redshift:
\begin{equation}
  1 + z \equiv \frac{a(t_o)}{a(t_e)}.
\end{equation}
Because we define $a(t_o) = 1$, we have
\begin{equation}
  a(t_e) = \frac{1}{1 + z}.
\end{equation}
Thus the redshift of a light source gives us a direct measurement of the
scale factor at the time that photon was emitted.  As such, it can be
substituted for $a$ as the dependent variable in the above equations
with the correct change-of-variables; we will switch between these two
conventions depending on what is convenient.

Measurement of the scale factor $a$ when the light was emitted
is only half the picture however: in order to constrain parameters in
eqn.~\ref{eq:friedmann_dimensionless}, we must be able to measure
information related to the time $t_e$ of emission.  This allows us to
constrain $a$ as a function of $t$, or the equivalent quantity in some
other parametrization.  To enable this, we'll introduce the concept
of distances in Cosmology.

Distance measures in cosmology are a potentially confusing subject.  An
excellent resource describing these can be found in \citet{hogg1999distance}.
We'll briefly define four relevant distance measures: the comoving distance,
the proper distance, the angular diameter distance, and the luminosity
distance.

\begin{description}
  \item[Comoving Distance:] The comoving distance is the distance $r$ which
    enters into the FLRW metric, eqn.~\ref{eq:FLRW_metric}.  This distance is
    constant for an object moving with the expansion of space.  Using the
    FLRW metric and setting $ds=0$, one can shown that the comoving
    distance for an object with redshift $z$ is given by
    \begin{equation}
      \label{eq:comoving_distance}
      r(z) = c\,\int_0^z \frac{dz^\prime}{H(z^\prime)}.
    \end{equation}
  \item[Proper Distance:] The proper distance is the simultaneous separation
    $d_p(z)$ between two objects.  From the FLRW metric with time interval
    $dt=0$, one can show that the proper distance is given by
    \begin{equation}
      \label{eq:proper_distance}
      d_p(z) = \frac{r(z)}{1 + z}.
    \end{equation}
  \item[Angular Diameter Distance:] The angular diameter distance $d_A(z)$
    is the
    ratio of the true size to the observed angular size (in radians) of
    an extended source.  From the FLRW metric with $dt = dr = 0$, one
    can show that the angular diameter distance is given by
    \begin{equation}
      \label{eq:angular_diameter_distance}
      d_A(z) = \frac{S_\kappa(r)}{1 + z}.
    \end{equation}
  \item[Luminosity Distance:] The luminosity distance $d_L(z)$ relates the
    emitted flux to the observed flux.  Taking into account the angular
    dilution of the flux, as well as the redshifted energy of each photon
    and time-delay of photon arrival leads to
    \begin{eqnarray}
      \label{eq:luminosity_distance}
      d_L(z) &=& (1 + z)^2\,d_A(z) \nonumber\\
      &=& (1 + z)\,S_\kappa(r)
    \end{eqnarray}
\end{description}
By measuring one of these distances as a function of redshift, we are
effectively constraining both the dependent and independent variable
in the dimensionless Friedmann equation, eqn.~\ref{eq:friedmann_dimensionless}.
In this way, it is possible to constrain the combinations of cosmological
parameters $\Omega_w$ which fit the data.

\section{Standard Candles: Cosmology via Luminosity Distance}
The earliest observations confirming the cosmic expansion (i.e. a positive
value of the hubble parameter $H_0$) were due to Edwin Hubble
\citep{hubble1929}.  Through observations of Cepheid-type variable stars
in nearby galaxies, Hubble found a linear correlation between the
luminosity distance to each galaxy and its apparent recessional velocity.
Although Hubble does not frame this measurement in terms of the Friedmann
equations, he cautiously mentions the potential relationship of the
observations to the ``de Sitter effect'', a particular solution of
Einstein's equations which is a special case of the FLRW metric and
Friedmann equations. Using our formalism above, we can show that for
nearby objects (such that $H(z) \approx H_0$),
the change in proper distance is approximated by
\begin{eqnarray}
  \frac{d}{dt}d_p(t) &=& H(t)\,\, d_p(t) \nonumber\\
                     &\approx& H_0\,\, d_L(t),
\end{eqnarray}
where the second line holds up to corrections of order $z$.  Because
Hubble used a sample with $z <\sim 0.005$, the linear fit is well within
observed errors.
Thus Hubble's observations can be seen as a first attempt at constraining
cosmological parameters in Equation~\ref{eq:friedmann_dimensionless} through
the simultaneous measurement of redshift and luminosity distance.

In the 70 years after Hubble's discovery, there were many attempts to
confirm and improve upon his discovery and use it to derive tighter
constraints on...

{\it Add some brief references to standard-candle-based constraints:
  Cepheid, MS fitting, TF relation, and type 1a supernovae.  Mention my
  work in this area: \citet{Kessler2009}}

The results summarized above are all related in that they are based on the
idea of a standard candle: if we can determine the intrinsic brightness
of an object as well as its redshift, then we can compare this to the
apparent brightness and constrain the hubble parameter $H(t)$.
Another path to this sort of constraint comes from standard {\it rulers}
rather than standard candles.  If we know the redshift as well as the
intrinsic size of an object, then we can use its apparent size to
constrain cosmological parameters.  One standard ruler is given by the
size-scales of structure in the universe.

\section{The Growth of Structure}
\label{sec:growth}
The above probes rely on standard or standardizable candles to measure the
luminosity distance as a function of redshift.  Other cosmological probes
depend on standard {\it rulers}, that is, objects whose true angular extent
is known, and can be used to determine the angular diameter distance as
a function of redshift.  Many of these probes depend on detailed modeling
of the growth of structure.

The distribution of density throughout the universe can be expressed
\begin{equation}
  \label{eq:overdensity}
  \rho(\myvec{x}, t) = \rho_b(t)[1 + \delta(\myvec{x}, t)]
\end{equation}
where we have taken out the background component,
\begin{equation}
  \rho_b(t) \propto \frac{1}{a(t)^3}.
\end{equation}
We can proceed by treating matter as an ideal fluid with a
velocity field $\myvec{u}(\myvec{x}, t)$ and pressure $P(\myvec{x}, t)$
\citep{longair2008galaxy}.
In this case, it is governed by the
continuity equation, which describes conservation of mass,
\begin{equation}
  \label{eq:continuity}
  \left(\frac{\partial\rho}{\partial t}\right)_{\myvec{x}}
  + \rho\myvec{\nabla}_{\myvec{x}}\cdot\myvec{u} = 0,
\end{equation}
the Euler equation, which specifies conservation of momentum,
\begin{equation}
  \label{eq:euler}
  \left(\frac{\partial\myvec{u}}{\partial t}\right)_{\myvec{x}}
  + (\myvec{u}\cdot\myvec{\nabla}_{\myvec{x}})\myvec{u}
  + \frac{\myvec{\nabla}_{\myvec{x}}P}{\rho}
  = -\myvec{\nabla}_{\myvec{x}}\Phi,
\end{equation}
and the Poisson equation, which describes gravity in the Newtonian limit
\begin{equation}
  \label{eq:poisson}
  \nabla_{\myvec{x}}^2\Phi = 4\pi G\rho.
\end{equation}
Transforming to comoving coordinates $\myvec{r} = \myvec{x}/a(t)$,
defining comoving time derivatives
$d/dt \equiv \partial/\partial t + \myvec{u}\cdot\myvec{\nabla}$,
and expressing in terms of the dimensionless density contrast
$\delta(\myvec{r}, t)$ (eq.~\ref{eq:overdensity}) gives
\begin{equation}
  \label{eq:delta_evolution}
  \frac{d^2\delta}{d t^2} + 2 H \frac{d\delta}{d t}
  = \frac{c_s^2}{a^2}\nabla_{\myvec{r}}^2\delta + 4\pi G \rho_b\,\delta
\end{equation}
where $\nabla^2_{\myvec{r}}$ indicates the Laplacian with respect to comoving
coordinates \citep[for derivation see][\S11.2]{longair2008galaxy}.

It is convenient to look at wave-like solutions of the denstity contrast
$\delta(\myvec{r}, t) \propto \exp[i(\myvec{k}\cdot\myvec{r} - \omega t)]$
such that eqn.~\ref{eq:delta_evolution} becomes
\begin{equation}
  \label{eq:delta_evolution_k}
  \ddot{\delta} + 2 H \dot{\delta} = \delta(4\pi G\rho_b - k^2c_s^2).
\end{equation}
Thus we see that the growth of perturbations $\delta$ with time depends
on the balance between the gravitational force through $4\pi G\rho_b$ and the
pressure through $k^2c_s^2$.  The scale where the left hand side of
Equation~\ref{eq:delta_evolution_k} is exactly zero is called the Jeans
length:
\begin{equation}
  \lambda_J \equiv c_s \sqrt{\frac{\pi}{G\rho_b}}.
\end{equation}
This is the approximate scale above which pressure cannot halt gravitational
collapse.  As we saw in \S\ref{

\newpage

\begin{itemize}
  \item Overdensities/underdensities and linear growth approximation
  \item Power spectra \& correlation functions \& the relationship to
    parameters from the previous section.
  \item nonlinear effects: clustering
  \item LRG power spectra, BAO, WMAP anisotropies
  \item Major uncertainty: mapping luminous matter to dark
\end{itemize}

\section{Gravitational Lensing}
\begin{itemize}
  \item Basic lensing geometry
  \item brief description of strong lensing, microlensing, weak lensing
\end{itemize}

\section{Weak Gravitational Lensing}
\begin{itemize}
  \item Weak gravitational lensing limit: $\gamma$ and $\kappa$
  \item Mass mapping from weak lensing surveys
  \item Power spectra of $\gamma$ and $\kappa$: how do these relate to
    the correlation functions from previous section?
  \item A note about the practical side: shape measurement, shot noise,
    etc.
\end{itemize}

{\it Notes:}
\begin{itemize}
  \item Mention work with the supernova collaboration \citep{Kessler2009}.
  \item We also should briefly mention here \cite{Jain2011} and alternatives
    to standard cosmological models \citep[also][]{Sollerman2009}.
\end{itemize}
