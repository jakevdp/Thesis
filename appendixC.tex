\chapter{Choice of KL Parameters}
\label{Choosing_Params}
The KL analysis outlined in Section~\ref{KL_Intro}
has only two free parameters: the number of
modes $n$ and the Wiener filtering level $\alpha$.  Each of these parameters
involves a trade-off: using more modes increases the amount of information
used in the reconstruction, but at the expense of a decreased signal-to-noise
ratio.  Decreasing the value of $\alpha$ to $0$ reduces the smoothing effect 
of the prior, but can lead to a nearly singular convolution matrix 
$\mymat{M}_{(n,\alpha)}$, 
which results in unrealistically large shear values in the poorly-constrained 
areas areas of the map (i.e.~masked regions).

To inform our choice of the number of modes $n$, we recall the trend of 
spatial scale with mode number seen in Figure~\ref{fig_bandpower}.  Our 
purpose in using KL is to allow interpolation in masked regions.  To this 
end, the angular scale of the mask should inform the choice of angular
scale of the largest mode used.  An eigenmode which probes scales much smaller
than the size of a masked region will not contribute meaningful information to
the reconstruction within that masked region.  Considering the pixels within
our mask, we find that 99.5\% of masked pixels are within 2 pixels of a
shear measurement.  This corresponds to an angular scale of $\ell=6140$.
Consulting Figure~\ref{fig_bandpower}, we see that modes larger than
about $n=900$ out of 4096 will probe length scales significantly 
smaller than the mask scale.  
Thus, we choose $n=900$ as an appropriate cutoff for our reconstructions.

To inform our choice of the Wiener filtering level $\alpha$, we examine the
agreement between histograms of \Map peaks for a noise-only DES field
with and without masking (see Section~\ref{Shear_Peaks}).  
We find that for large (small) values of $\alpha$, the number
of high-\Map peaks is underestimated (overestimated) in the masked 
case as compared to the unmasked case.  
Empirically, we find that the two agree at $\alpha = 0.15$; 
we choose this value for our analysis.  Note that this 
tuning is done on noise-only reconstructions, 
which can be generated for observed data by assuming that
shape noise dominates: 
\begin{equation}
  [\Noise_\gamma]_{ij} = \frac{\sigma_\epsilon}{n_i^2}\delta_{ij}.
\end{equation}
The $\alpha$-tuning can thus be performed on artificial noise realizations 
which match the observed survey characteristics.

We make no claim that $(n,\alpha) = (900,0.15)$ is the optimal
choice of free parameters for KL: determining this would involve a more
in-depth analysis.  They are simply well-motivated choices which we use to
make a case for further study.
