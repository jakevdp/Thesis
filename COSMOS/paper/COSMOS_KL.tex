% tutorial.tex  -  a short descriptive example of a LaTeX document
%
% For additional information see  Tim Love's ``Text Processing using LaTeX''
% http://www-h.eng.cam.ac.uk/help/tpl/textprocessing/
%
% You may also post questions to the newsgroup <b> comp.text.tex </b> 

%\documentclass[preprint]{aastex}
\documentclass[twocolumn]{emulateapj} 

\usepackage{amsfonts}
\usepackage{amsbsy}
\usepackage{natbib}
\usepackage{graphicx}
\usepackage{ulem}

\usepackage{color}

%change path to figures
\graphicspath{{fig/}}

\newcommand{\myvec}[1]{\boldsymbol{#1}}
\newcommand{\mymat}[1]{\boldsymbol{#1}}
\newcommand{\Map}{\ensuremath{M_{\rm ap}}\ }
\newcommand{\naive}{na\"{i}ve\ }
\newcommand{\Noise}{\mymat{\mathcal{N}}}

\newcommand{\comment}[1]{{\color{red} #1}}

\slugcomment{\textit{Draft Version, \today.} }
 
\alph{footnote}

\begin{document}

\title{Eigenmode Analysis Weak Lensing Surveys}

\author{J. T. VanderPlas}
\affil{Astronomy Department, University of Washington, Box 351580, 
  Seattle, WA 98195-1580}
\bibliographystyle{apj}

\begin{abstract}
  We explore the application of KL analysis to cosmological likelihood analysis
  in weak lensing surveys.  We apply this method to the COSMOS survey.
\end{abstract}

\keywords{
  gravitational lensing ---
  dark matter ---
  large-scale structure of universe  }

\section{Introduction}
\label{sec:introduction}
A decade ago, observations of distant supernovae pointed to an unexpected
acceleration in the expansion rate of the universe
\citep{Riess98, Perlmutter99}.  Since then, many independent observations
have confirmed this discovery: via cosmic microwave background radiation
[cites], baryon accoustic oscillations [cites], galaxy clusters [cites],
gravitational lensing [cites], and large-scale structure measurements [cites],
to name a few.  In the established cosmological picture, this anomalous
acceleration can be represented by a new component of cosmological
matter-energy, ``dark energy''.
Though the presence of dark energy is well-supported, its
nature remains a mystery.  
Cosmic shear -- the statistically observed weak gravitational lensing
signal from the cosmological matter distribution -- has the potential to
put particularly good constraints...

In this work we explore a new technique for the evaluation of cosmological
likelihoods using Karhunen-Lo\'{e}ve (KL) analysis of shear fields.  
We first explored the use of KL analysis in shear surveys in a previous work
(Vanderplas \textit{et al.} in preparation),
in which we focused on the ability of
KL modes to help fill-in missing information within the context of weak
lensing convergence mapping and studies of the peak statistics of the
resulting mass maps.
Here we follow a different approach:
we use KL analysis to aid in the calculation of cosmological likelihoods
using two-point statistics within a Bayesian framework.
This draws upon similar work done
previously to constrain cosmological parameters using number counts of
galaxy surveys \citep{Vogeley96, Pope04}.

In \S\ref{sec:lensing_intro} we review and discuss the strengths and
weaknesses of constraining cosmological quantities using two-point shear
statistics.
In \S\ref{sec:kl_intro} we review KL analysis and its application to 
shear surveys.
In \S\ref{sec:data} we describe the COSMOS shear data used in this analysis.
In \S\ref{sec:discussion} we discuss the results.
We conclude in \S\ref{sec:conclusion}.

\section{Two-point Statistics in Weak Lensing}
\label{sec:lensing_intro}
The large-scale structure of the universe provides a powerful probe of
cosmological parameters.  Through gravitational instability, the initial
matter fluctuations have grown to the nonlinear structure we see today.
This happens in a hierarchical manner, with the smallest structures
collapsing before the largest.  One of the most powerful probes of this
structure is the redshift-dependent power spectrum of matter density
fluctuations, $P(k,z)$, where $k$ which gives the amplitude of the
Fourier mode with wave-number $k$ at a redshift $z$. 
This approach has often been used to measure cosmological parameters
through optical tracers \citep{Tegmark06}.

Another way to get at the matter power spectrum is through weak gravitational
lensing.  As light travels from a distant source to an observer, its path is
perturbed by the gravitational force of the intervening matter distribution.
This cosmic shear signal is sensitive to both the gravitational potential
of the matter along the line of sight, and the angular diameter distance
to that matter.  As such, two-point measurements of cosmic shear have promise
to be powerful probes of cosmological parameters, including dark energy
\citep[see][]{Takada07}.  Recent results have shown the power of this
approach \citep{Ichiki09, Schrabback10}.

There are two approaches to measuring two-point information which are 
mathematically equivalent: the power spectrum $\mathcal{P}(\ell)$,
and its fourier transform $\xi(\theta)$.
The most common approach to measuring two-point information in practice
is through the correlation functions \citep[see][]{Schneider02}.
The main advantage of correlation functions is their ease of measurement:
they can be straightforwardly estimated from galaxy positions and shears,
even in very complicated survey geometries.  The disadvantage is that the
signal is highly correlated between different bins.  Accounting for this 
correlation is very important when computing cosmological likelihoods:
this often requires large suites of simulations to take into account.

Shear power spectra, on the other hand, have a number of nice properties.
They are simpler to map to theory than correlation functions.  They have
weaker correlations between different multipoles: on the largest scales,
where structure is close to gaussian, the scales are statistically independent.
Even on small scales where non-gaussianity leads to correlated errors,
these correlations have a relatively small effect \citep{Takada09}.
The disadvantage of shear power spectra as direct cosmological probes is
the difficulty of measuring them from data.  In particular, survey geometry
such as finite fields and masking effects can lead to mode-mixing on all
angular scales.  There have been a few attempts to correct for this
difficulty \citep{Brown03, Hikage11}.

The main problem with the power spectrum approach is the finite sky coverage:
spherical harmonics are orthogonal over the entire sky, but are not
necessarily orthogonal over a small patch of the sky.  This means that a
spherical harmonic decomposition on which power spectra are based is not
unique for partial sky coverage.  Thus the main difficulty with the power
spectrum method can be summed up as follows: the orthogonal modes are no
longer orthogonal.  It may be possible to construct a survey in order to
limit the magnitude of these effects
\citep[see][for some approaches]{Kilbinger04, Kilbinger06}.
One could imagine instead constructing a set of orthogonal modes for the
observed survey geometry.  Because the modes are orthogonal by construction,
one can skirt the difficulty of mode mixing.  We propose to take this latter
approach using Karhunen-Lo\'{e}ve (KL) analysis.

\section{KL for Parameter Estimation}
\label{sec:kl_intro}
KL analysis and the related Principal Component Analysis are well-known
statistical tools which have been applied in a wide variety of astrophysical
situations, from e.g. analysis of the spatial power of galaxy counts
\citep{Vogeley96, Szalay03, Pope04}
to characterization of stellar, galaxy, and QSO spectra
\citep{Connolly95, Connolly99, Yip04a, Yip04b},
to studies of noise properties of weak lensing surveys
\citep{Kilbinger06, Munshi06}, and a host of other situations too numerous
to mention here.  Informally, the power of KL/PCA rests in the fact that 
it allows a highly efficient representation of a set of data, highlighting
the components which are most important in the dataset as a whole.
The discussion of KL analysis below derives largely from \citet{Vogeley96},
reexpressed for application in cosmic shear surveys.

Any $D$-dimensional data point may be represented as a linear combination of 
$D$ orthogonal basis functions.  
For example, the data may be $N$ individual galaxy spectra, each with flux
measurements in $D$ wavelength bins.  Each spectrum can be thought of as a
single point in $D$-dimensional parameter space, where each axis corresponds
to the value within a single wavelength bin.  
Geometrically, there is nothing special about
this choice of axes: one could just as easily rotate and translate the axes
to obtain a different but equivalent representation of the same data.

In the case of of a shear survey, our single data vector is the set of
cosmic shear measurements across the sky.  We will divide the sky into $N$
cells in angular and redshift space, at coordinates
$\myvec{x}_i = (\theta_{x,i}, \theta_{y,i}, z_i)$
These cells may be spatially distinct, or they may overlap.
From the ellipticity of the galaxies within each cell, we
estimate the shear
$\gamma_i \equiv \gamma^o(\myvec{x}_i) = 
\gamma(\myvec{x}_i) + n_\gamma(\myvec{x}_i)$
where $\gamma(\myvec{x}_i)$ is the true underlying shear,
and $n_\gamma(\myvec{x}_i)$ is the measurement noise.
Our data vector is then
$\myvec{\gamma} = [\gamma_1, \gamma_2 \cdots \gamma_N]^T$.

We seek to express our set of measurements $\myvec{\gamma}$
as a linear combination of $N$ (possibly complex) 
orthonormal basis vectors
$\{\myvec{\Psi}_j(\myvec{x}_i, j=1,N)\}$ with complex coefficients
$a_j$:
\begin{equation}
  \label{eq:gamma_decomp}
  \gamma_i = \sum_{j=1}^{N} a_j \Psi_j(\myvec{x}_i)
\end{equation}
For conciseness, we'll create the matrix $\mymat{\Psi}$ whose columns are
the basis vectors $\myvec{\Psi}_j$, so that the above equation can be
compactly written $\myvec\gamma = \mymat\Psi\myvec{a}$.  Orthonormality
of the basis vectors leads to the property
$\mymat\Psi^\dagger\mymat\Psi = \mymat{I}$, where $\mymat{I}$ is the identity
matrix: that is, $\mymat\Psi$ is a unitary matrix with
$\mymat\Psi^{-1} = \mymat\Psi^\dagger$.  Observing this, we can easily compute
the coefficients for a particular data vector:
\begin{equation}
  \myvec{a} = \mymat\Psi^\dagger \myvec\gamma.
\end{equation}
We will be testing the likelihood of a particular set of coefficients
$\myvec{a}$.  
The statistical properties of these coefficients can be written in terms of
the covariance of the observed shear:
\begin{equation}
  \label{eq:a_cov}
  \left\langle \myvec{a}\myvec{a}^\dagger \right\rangle 
  =  \mymat\Psi^\dagger
  \left\langle \myvec\gamma\myvec\gamma^\dagger \right\rangle 
  \mymat\Psi
  \equiv \mymat\Psi^\dagger \myvec{\xi}  \mymat\Psi
\end{equation}
where we have defined the observed shear correlation matrix 
$\myvec{\xi} \equiv \left\langle 
\myvec\gamma\myvec\gamma^\dagger \right\rangle$, and angled braces
$\langle\cdots\rangle$ denote expectation value or ensemble average
of a quantity.

Because we hope to perform a likelihood analysis on the coefficients
$\myvec{a}$, it will be useful in likelihood estimation if they are
statistically orthogonal:
\begin{equation}
  \label{eq:a_cov_2}
  \left\langle \myvec{a}\myvec{a}^\dagger \right\rangle_{ij}
  = \left\langle a_i^2 \right\rangle \delta_{ij}
\end{equation}
Comparing Equations \ref{eq:a_cov} \& \ref{eq:a_cov_2} we see that the desired
basis functions are the solution of the eigenvalue problem
\begin{equation}
  \mymat\xi \myvec\Psi_j = \lambda_j \myvec\Psi_j
\end{equation}
where the eigenvalue $\lambda_j = \left\langle a_i^2 \right\rangle$.
By convention, we'll order the eigenvalue/eigenvector pairs such that
$\lambda_i \ge \lambda_{i+1} \forall i\in(1, N-1)$.
Expansion of the data $\myvec\gamma$ into this basis is the discrete form
of KL analysis.

A KL decomposition has a number of useful properties:
\begin{description}
  \item[Uniqueness] A KL decomposition is a unique representation of the data.
    That is, there only a single set of basis vectors which satisfy the above
    properties (up to degeneracies resulting from identical eigenvalues)
    This can be straightforwardly shown in a proof by contradiction
    \citep[e.g.][]{Vogeley96}.
  \item[Efficiency] A partial KL decomposition provides the
    optimal low-rank approximation of an observed data vector.  That is,
    for $n < N$, the partial reconstruction (cf. Eqn.~\ref{eq:gamma_decomp})
    \begin{equation}
      \myvec\gamma^{(n)} \equiv \sum_{j=1}^n a_j \Psi_j
    \end{equation}
    minimizes the average reconstruction error
    $\epsilon_n \equiv |\myvec\gamma - \myvec\gamma^{(n)}|^2$
    for any orthogonal basis set $\mymat\Psi$.  The proof can be easily
    obtained using Lagrangian multipliers \citep[again, see][]{Vogeley96}.
  \item[Signal-to-noise Optimization] As a consequence of the efficiency
    property, it is clear that for data with white noise\footnote{
    By \textit{white noise} we mean that the noise covariance satisfies
    $\mymat{\mathcal{N}}_{ij} \equiv 
    \langle\myvec{n_\gamma}\myvec{n_\gamma}^\dagger\rangle_{ij} 
    \propto \delta_{ij}$}, KL modes provide the
    maximum possible signal-to-noise ratio per mode.  The noise can be assured
    to be white through a judicious choice of binning, or alternatively
    the data can be artificially whitened (See \S\ref{sec:whitening}).
    If signal and noise are uncorrelated, then the covariance of the observed
    shear can be decomposed as
    \begin{equation}
      \mymat{\xi} = \mymat{\mathcal{S}} + \mymat{\mathcal{N}}
    \end{equation}
    Because the noise covariance $\mymat{\mathcal{N}} \equiv 
    \langle\myvec{n_\gamma}\myvec{n_\gamma}^\dagger\rangle$ is proportional
    to the identity by assumption, Diagonalization of $\mymat{\xi}$ results
    in a simultaneous diagonalization of both the signal $\mymat{\mathcal{S}}$
    and the noise $\mymat{\mathcal{N}}$.  Because of this signal-to-noise
    optimization property, KL modes can be proven to be the optimal basis
    for testing of spatial correlations \citep[see Appendix A of][]{Vogeley96}.
\end{description}

\subsection{Shear Noise Properties}
\label{sec:whitening}
The signal-to-noise properties of shear mentioned above are based on the 
requirement that noise be ``white'', that is, the noise covariance is
$\mymat{\mathcal{N}} \equiv 
\langle\myvec{n_\gamma}\myvec{n_\gamma}^\dagger\rangle
= \sigma^2 \mymat{I}$.  Noise in measured shear is affected mainly by the
intrinsic ellipticity and source density, but can also be prone to systematic
effects which lead to noise correlations between pixels.  When the survey
geometry leads to shear with more complicated noise characteristics, a
whitening transformation can be applied.

Given the measured data $\myvec\gamma$ and noise covariance
$\mymat{\mathcal{N}}$, we can define the whitened shear
\begin{equation}
  \myvec{\gamma}^\prime = \mymat{\mathcal{N}}^{-1/2} \myvec{\gamma}
\end{equation}
With this definition, the shear covariance matrix becomes
\begin{eqnarray}
  \mymat{\xi}^\prime 
  &=& \left\langle \myvec{\gamma}^\prime 
  \myvec{\gamma}^{\prime\dagger}\right\rangle \nonumber\\
  &=& \mymat{\mathcal{N}}^{-1/2}\mymat{\xi}
  \mymat{\mathcal{N}}^{-1/2} \nonumber\\
  &=& \mymat{\mathcal{N}}^{-1/2}\left[
    \mymat{\mathcal{S}} + \mymat{\mathcal{N}}
    \right]\mymat{\mathcal{N}}^{-1/2} \nonumber\\
  &=& \mymat{\mathcal{N}}^{-1/2}\mymat{\mathcal{S}}\mymat{\mathcal{N}}^{-1/2} + \mymat{I}
\end{eqnarray}
We see that the whitened signal is $\mymat{\mathcal{S}}^\prime = 
\mymat{\mathcal{N}}^{-1/2}\mymat{\mathcal{S}}\mymat{\mathcal{N}}^{-1/2}$
and the whitened noise is $\mymat{\mathcal{N}}^\prime = \mymat{I}$, the
identity matrix. So this transformation in fact whitens the data covariance,
so that the noise in each bin is constant and uncorrelated.  Given the
whitened measurement covariance $\mymat{\xi}^\prime$, we can find the KL
decomposition which satisfies the eigenvalue problem
\begin{equation}
  \mymat{\xi}^\prime \myvec{\Psi^\prime}_j = 
  \lambda^\prime_j \myvec{\Psi^\prime}_j
\end{equation}
With KL coefficients given by
\begin{equation}
  \myvec{a}^\prime = \mymat{\Psi}^{\prime\dagger}
  \mymat{\mathcal{N}}^{-1/2}\myvec\gamma
\end{equation}
Note that because $\langle\myvec\gamma\rangle = 0$,
the expectation value of the KL coefficients is
\begin{eqnarray}
  \langle\myvec{a}^\prime\rangle 
  &=& \mymat{\mathcal{N}}^{-1/2}\langle\myvec\gamma\rangle\nonumber\\
  &=& 0
\end{eqnarray}
For the remainder of this work, it will be assumed that we are working with
whitened quantities.  The primes will be dropped for notational simplicity.

\subsection{Constructing the Covariance Matrix}
In many applications, the data covariance matrix can be estimated
empirically, using the fact that
\begin{equation}
  \tilde{\myvec\xi} = \lim_{N\to\infty} \sum_{i=1}^N 
  \myvec{\gamma}_i \myvec{\gamma}_i^\dagger
\end{equation}
Unfortunately, in surveys of cosmic shear, we have only a single sky to
observe, so this approach does not work.  Instead, we can construct the
measurement covariance analytically by assuming a theoretical form of the
underlying matter power spectrum.

The measurement covariance $\mymat{\xi}_{ij}$ between two regions of the
sky $A_i$ and $A_j$ is given by
\begin{eqnarray}
  \label{eq:xi_analytic}
  \myvec{\xi}_{ij} 
  &=& \mymat{\mathcal{S}}_{ij} + \mymat{\mathcal{N}}_{ij} \nonumber\\
  &=& \left[\int_{A_i}d^2x_i\int_{A_j}d^2x_j 
    \xi_+(|\myvec{x_i}-\myvec{x_j}|)\right]
  + \mymat{\mathcal{N}}_{ij}
\end{eqnarray}
where $\xi_+(\theta)$ is the ``+'' shear correlation function. 
$\xi_+(\theta)$ is expressible as an integral over the shear power spectrum
weighted by the zeroth-order Bessel function
\citep[see, e.g.][]{Schneider02}:
\begin{equation}
  \label{eq:xi_plus_def}
  \xi_+(\theta) 
  = \frac{1}{2\pi} \int_0^\infty d\ell\ \ell P_\gamma(\ell) J_0(\ell\theta)
\end{equation}
The angular shear power spectrum $P_\gamma(\ell)$ can be expressed as a
weighted line-of-sight integral over the matter power 
spectrum \citep[see, e.g.][]{Takada04}:
\begin{equation}
  \label{eq:P_gamma}
  P_\gamma(\ell) = \int_0^{\chi_s}d\chi W^2(\chi)\chi^{-2}
  P_\delta\left(k=\frac{\ell}{\chi};z(\chi)\right)
\end{equation}
Here $\chi$ is the comoving distance, $\chi_s$ is the distance to the
source, and $W(\chi)$ is the lensing weight function,
\begin{equation}
  \label{eq:lensing_weight}
  W(\chi) = \frac{3\Omega_{m,0}H_0^2}{2a(\chi)}\frac{\chi}{\bar{n}_g}
  \int_{\chi}^{\chi_s}dz\ n(z) \frac{\chi(z)-\chi}{\chi(z)}
\end{equation}
where $n(z)$ is the empirical redshift distribution of galaxies.
The nonlinear mass fluctuation power spectrum $P_\delta(k, z)$ can be
predicted semianalytically: in this work we use the halo model of
\citet{Smith03}.  With this as an input, we can analytically
construct the measurement covariance matrix $\mymat\xi$ using 
Equations~\ref{eq:xi_analytic}-\ref{eq:lensing_weight}.

\subsection{Cosmological Likelihood Analysis with KL}
From the survey geometry and galaxy ellipticities, we measure the
shear $\myvec\gamma$, estimate the noise covariance
$\mymat{\mathcal{N}}$ (see \S\ref{sec:bootstrap}) and derive
the whitened covariance matrix $\mymat\xi$. 
From $\mymat\xi$ we compute the KL basis $\mymat\Psi$ and $\myvec\lambda$.
Using the KL basis, we compute the coefficients
$\myvec{a} = \mymat{\Psi}^\dagger \mymat{\mathcal{N}}^{-1/2} \myvec\gamma$.
Given these KL coefficients $\myvec{a}$, we use a Bayesian framework to
compute the posterior distribution of our cosmological parameters.

Given observations $D$ and prior information $I$, Bayes' theorem specifies the
posterior probability of a model described by the parameters $\{\theta_i\}$:
\begin{equation}
  \label{eq:bayes}
  P(\{\theta_i\}|DI) = P(\{\theta_i\}|I) \frac{P(D|\{\theta_i\}I)}{P(D|I)}
\end{equation}
The term on the LHS is the \textit{posterior} probability of the set of
model parameters $\{\theta_i\}$, which is the quantity we are interested in.

The first term on the RHS is the \textit{prior}.  It quantifies how our prior
information $I$ affects the probabilities of the model parameters.  The 
prior is where information from other surveys (e.g. WMAP, etc) can be
included. The likelihood function for the observed coefficients $\myvec{a}$
enters into the numerator $P(D|\{\theta_i\}I)$.  The denominator $P(D|I)$
is essentially a normalization constant, set so that the sum of probabilities
over the parameter space equals unity.

For a given model $\{\theta_i\}$, we can predict the expected distribution of model KL
coefficients $\myvec{a}_{\{\theta_i\}} \equiv \mymat{\Psi}^\dagger
\mymat{\mathcal{N}}^{-1/2}\myvec{\gamma}$:
\begin{eqnarray}
  \mymat{C}_{\{\theta_i\}}
  & \equiv & \langle\myvec{a}_{\{\theta_i\}}
  \myvec{a}_{\{\theta_i\}}^\dagger\rangle\nonumber\\
  &=& \mymat{\Psi}^\dagger \mymat{\mathcal{N}}^{-1/2} 
  \mymat{\xi}_{\{\theta_i\}}\mymat{\mathcal{N}}^{-1/2}\mymat{\Psi}
\end{eqnarray}
Using this, the measure of departure from the model $m$ is given by the
quadratic form
\begin{equation}
  \chi^2 = \myvec{a}^\dagger\mymat{C}_{\{\theta_i\}}^{-1}\myvec{a}
\end{equation}
The likelihood is then given by
\begin{equation}
  \label{eq:likelihood}
  \mathcal{L}(\myvec{a}|\{\theta_i\}) = 
  (2\pi)^{n/2} |\det(C_{\{\theta_i\}})|^{-1/2}
  \exp(-\chi^2/2)
\end{equation}
where $n$ is the number of degrees of freedom: that is, the number
of eigenmodes included in the analysis.  The likelihood given by
Equation~\ref{eq:likelihood} enters into Equation~\ref{eq:bayes} when
computing the posterior probability.

\section{COSMOS data}
\label{sec:data}
To test the KL likelihood formalism, we use a shear catalog derived from the
COSMOS survey\footnote{We are grateful to Tim Schrabback et al. for making 
this data available to us}.  The description of this catalog and detailed 
tests of its systematics are presented in
\citet[][hereafter S10]{Schrabback10}.  We will not repeat that information
here.  The catalog contains shape measurements of () source galaxies 
in a () square-degree field.  () of these source galaxies have well-defined
photometric redshifts from the COSMOS30 pipeline(ref), while () of the 
source galaxies have poorly fitted redshifts. $\cdots$

\subsection{Intrinsic Ellipticity estimation}
\label{sec:bootstrap}
In the KL analysis described above, it is important to have an accurate
determination of the noise for the observed shear.  Assuming systematic
errors are negligible, shape noise should be dominated by shot noise,
which scales as $\mymat{\mathcal{N}}_{ii} = \hat{\sigma}_\epsilon^2 / n_i$,
with $n_i$ representing the number of galaxies in bin $i$.

To test this assumption, we perform a bootstrap resampling of the observed
shear in 4 arcminute pixels.  Generating 1000 bootstrap samples within each
pixel, we find the variance shown in the top panel of
Figure~\ref{fig:bootstrap}.  These points are well-fit by the curve
\begin{equation}
  \sigma_\gamma^2 = \frac{0.393^2}{n_{\rm gal}}
\end{equation}
This best-fit curve is shown in the top panel of Figure~\ref{fig:bootstrap}.
The bottom panel shows the residuals to this fit, re-cast in terms of the
best-fit $\sigma$ for each point.

In this figure, we see that the fluctuation in shape noise from pixel to
pixel is only a few percent.  For the analysis below, we use for each pixel
the bootstrapped estimates derived here.  Because bootstrapping is inaccurate
for pixels with a small number of galaxies, if a pixel has fewer than 10
galaxies we use the best-fit estimate for the noise,
$\mymat{\mathcal{N}}_{ii} = \hat{\sigma}_\epsilon^2 / n_i$
with $\hat{\sigma}_\epsilon = 0.393$.  Pixels with zero galaxies (i.e.
masked pixels) are treated using the techniques developed in
Section~\ref{sec:kl_intro}.

\begin{figure*}
 \centering
 \plotone{sigma_calc.eps}
 \caption{Bootstrap estimates of the shape noise for each pixel.  The estimates
   reflect an intrinsic ellipticity of $0.393 \pm 0.013$.
   \label{fig:bootstrap}}
\end{figure*}

\begin{figure*}
 \centering
 \plotone{bright_eigenmodes.eps}
 \caption{
   The first nine 2D KL signal-to-noise eigenmodes
   for the COSMOS bright objects, with 2 arcminute pixels.
   \label{fig:eigenmodes}}
\end{figure*}

\begin{figure*}
 \centering
 \plotone{bright_coeff_hist.eps}
 \caption{
   The histogram of normalized coefficients $a_i / \sqrt{\lambda_i}$.
   If the shear is truly a gaussian random field, this distribution should
   be a gaussian with unit variance.
   \label{fig:coeff_hist}}
\end{figure*}

\begin{figure*}
 \centering
 \plotone{bright_eigenvalues.eps}
 \caption{
   The distribution of KL eigenvalues for the bright sample.
   \label{fig:eigenvalues}}
\end{figure*}

\begin{figure*}
 \centering
 \plotone{bright_bandpower.eps}
 \caption{
   The fourier power represented by each KL mode.  For each KL mode number,
   the vertical band shows the distribution of power with angular wavenumber
   $\ell$.  In general, the larger KL modes correspond to larger values of
   $\ell$, though there is a lot of mode mixing.
   \label{fig:bandpower}}
\end{figure*}

\section{Discussion}
\label{sec:discussion}
Discussion of estimated parameters.

\section{Conclusion}
\label{sec:conclusion}
Aaaaaand... we conclude.

\bibliography{COSMOS_KL}

%%%%%%%%%%%%%%%%%%%%%%%%%%%%%%%%%%%%%%%%%%%%%%%%%%%%%%%%%%%%%%%%%%%%%%%%%%%
%             begin appendix
\begin{appendix}
This is the appendix
\end{appendix}

\end{document}
