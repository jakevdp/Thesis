\chapter{Conclusion}
In the above chapters, we have developed \KL\ analysis as a useful tool
for several aspects of the analysis of present and future weak lensing
surveys.  In Chapter 2, we discussed the details of the KL formalism.
we showed that KL is a powerful technique which allows data to be represented
as a linear combination of orthogonal modes which are constructed such that
the modes are optimal representations of the signal-to-noise ratio.

In Chapter 3, we showed how KL can be used to construct an optimal linear
framework for the mapping of three dimensional structure from weak lensing
surveys.  The KL filtering leads to an algorithm which is orders of magnitude
faster than previously studied approaches, and allows quantitative constraints
on the effectiveness of mapping for given survey depths and geometries.

In Chapter 4, we showed how KL can be used to address a practical problem
of two and three dimensional mass mapping: the interpolation of shear signal
across masked regions of a given survey.  The reconstruction takes into
account theoretical expectations of the shear correlation, and results in
peak counts which are more consistent with those of the underlying
distribution.  The KL approach also results in a natural filtration of
low-magnitude noisy peaks, which has the potential to increase the
performance of cosmological likelihood calculations from peak statistics
of shear.

In Chapter 5, we show how KL can be used directly as a tool to derive
cosmological parameteter constraints from two-point information within
a Bayesian inference framework. Because KL can naturally account for 
arbitrary survey masks and geometries, it allows for robust determination
of likelihoods without the need for computationally expensive callibration
against N-body simulations.
As a proof-of-concept, we perform a
two-dimensional likelihood analysis to derive constraints on $\sigma_8$
and $\Omega_M$ which are consistent with those derived from conventional
correlation-function approaches using the same data.

In these three important areas of weak lensing analysis, the KL approach
has proven valuable in addressing the practical problems associated with
the science of weak lensing.  KL's robust, computationally efficient approach
has the potential to be very useful in many areas of future weak lensing
science.
